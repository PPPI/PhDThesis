\chapter{Conclusions}
\label{chapter:conclusions}

As projects continue, they accumulate bit-rot, technical debt, and generally
suffer an attrition to their health. Maintaining them in shape is indeed an
active process, oft forgone for implementing new features. One can hardly blame
the developer when the cost is typically a considerable time investment and the
reward is considerably delayed. To fix the situation, we need to fix either the
cost or the incentives. In this thesis, I consider reducing the cost of maintaining project health. I
present three tools and approaches that can aid developers, either directly or
via tool smiths, actively maintain project health. I present Aide-mémoire, a
tool for PR-Issue link prediction, Flexeme, a commit untangling tool, and POSIT,
a mixed-text segmentation and tagging tool. 

\section{Summary of Contributions}

In Aide-mémoire, we proposed the first online Pull-request-Issue linker. It aims
to help developers with suggestions when they are least likely to require a
context-switch, during Issue triaging and PR submission. This is key to
obtaining developer support, as the context-switch cost may make or break tools
that offer suggestions~\cite{ohearnKeynote2020}.

Flexeme untangles version histories for developers as they make history. When
committing a local branch to upstream, a developer is likely to still have the
context of their work in mind. This enables them to assess the correctness of
the suggestion with minimal cost. Further, atomic histories can benefit from a
wider range of tools: semantic version history slicing is expected to suffer
from less noise, while also making git bisect and other delta debugging
approaches over commits more attractive.

Through POSIT, we hope to have opened a new line of research. We formulated the
mixed-text tagging problem and demonstrated the benefits of considering both
natural and formal language channels when considering sources that mix both.
Although POSIT is closer to a meta-tool, it assists tool smiths make their tools
aware to the two channels intertwined in a mixed-text, POSIT can directly help
user of software fora such as Stack Overflow by detecting missed code
annotations.

Overall, this thesis demonstrates how the developer process can be augmented to
benefit both researchers as well as practitioners: a healthy project holds
promises for both. The areas in which we improve the state-of-the-art all serve
to provide cleaner development and project histories, or improve the tool
accessibility to developer communication over mixed-channels. 

\section{Summary of Future Work}
The work presented in this thesis identifies directions for further research. I discuss some of these directions in this section.

\textbf{Mixed-text comprehension}
POSIT is a step along the path to mixed-text comprehension. It offers the first
building blocks, an equivalent for Part of Speech tags for mixed-text. More
sophisticated models can build upon this to aid comprehension and parsing of
mixed text such that meaning is carried over from the Natural Language channel
to the algorithmic channel. Further, POSIT solves a reduced case of the
mixed-text parsing problem. Indeed, it assumes a single Formal and a single
Natural Language. 

\textbf{Graph Neural Network Approaches for Graph Similarity}
In Flexeme, we make use of graph kernels for graph Similarity computation;
however, recent research in Graph Neural Networks show promising results, and
future work may focus on replacing our graph kernels for a neural network
implementation, if a suitable training corpus is available.

\textbf{Multi-version PDG Static Analysis}
As we use the \deltaPDG for pairs of versions, we do not exploit the structure
to its full potential. Static analysis that relies on PDGs could be made version
aware such that the analysis can be carried out over multiple versions by simply
considering the \deltaPDG.

\textbf{Online Traceability}
In Aide-mémoire, we considered online traceability of PR-Issue links. This idea
could be extended to other forms of traceability, for example, the commit-issue
linking problem from which we started. While the research area of Just-In-Time
Traceability indeed exists and is active~\cite{hayes2003improving,
lin2006poirot, lucia2007recovering, cleland2007best}, it often uses offline
Information Retrieval techniques thus making the developer feedback delayed. An
online approach could serve developers better by being able to adapt faster to
changing requirements.