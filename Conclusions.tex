\chapter{Conclusions}
\label{chapter:conclusions}

\todo{We conclude on how the three papers advance the state of the art. How further advances could be made and future directions for project health.}

\section{Summary of Contributions}

\todo{summarise key takeaways of the three papers, sum them up under a single umbrella}

We proposed the first online PR-Issue linker, advanced the state of the art for commit untangling, and open the door to mixed-text analysis.

\section{Summary of Future Work}

\todo{more fluff}

\textbf{Mixed-text comprehension}
POSIT is a step along the path to mixed-text comprehension. It offers the first building blocks, an equivalent for Part of Speech tags for mixed-text. More sophisticated models can build upon this to aid comprehension and parsing of mixed text such that meaning is carried over from the Natural Language channel to the algorithmic channel.

\textbf{Graph Neural Network Approaches for Graph Similarity}
In Flexeme, we make use of graph kernels for graph Similarity computation; however recent research in Graph Neural Networks show promising results, and future work may focus on replacing our graph kernels for a neural network implementation if a suitable training corpus is present.

\textbf{Multi-version PDG Static Analysis}
In Flexeme, we propose a novel structure, the \deltaPDGN. While we demonstrate its use for commit untangling, this same structure could be employed for multi-version static analysis when the analysis depends on Program Dependency Trees. Alternatively, one could consider version sensitive slicing.