% UCL Thesis LaTeX Template
%  (c) Ian Kirker, 2014
% 
% This is a template/skeleton for PhD/MPhil/MRes theses.
%
% It uses a rather split-up file structure because this tends to
%  work well for large, complex documents.
% We suggest using one file per chapter, but you may wish to use more
%  or fewer separate files than that.
% We've also separated out various bits of configuration into their
%  own files, to keep everything neat.
% Note that the \input command just streams in whatever file you give
%  it, while the \include command adds a page break, and does some
%  extra organisation to make compilation faster. Note that you can't
%  use \include inside an \include-d file.
% We suggest using \input for settings and configuration files that
%  you always want to use, and \include for each section of content.
% If you do that, it also means you can use the \includeonly statement
%  to only compile up the section you're currently interested in.
% You might also want to put figures into their own files to be \input.

% For more information on \input and \include, see:
%  http://tex.stackexchange.com/questions/246/when-should-i-use-input-vs-include


% Formatting and binding rules for theses are here: 
%  https://www.ucl.ac.uk/students/exams-and-assessments/research-assessments/format-bind-and-submit-your-thesis-general-guidance

% This package goes first and foremost, because it checks all 
%  your syntax for mistakes and some old-fashioned LaTeX commands.
% Note that normally you should load your documentclass before 
%  packages, because some packages change behaviour based on
%  your document settings.
% Also, for those confused by the RequirePackage here vs usepackage
%  elsewhere, usepackage cannot be used before the documentclass
%  command, while RequirePackage can. That's the only functional
%  difference as far as I'm aware.
\RequirePackage[l2tabu, orthodox]{nag}

\PassOptionsToPackage{dvipsnames}{xcolor}


% ------ Main document class specification ------
% The draft option here prevents images being inserted,
%  and adds chunky black bars to boxes that are exceeding 
%  the page width (to show that they are).
% The oneside option can optionally be replaced by twoside if
%  you intend to print double-sided. Note that this is
%  *specifically permitted* by the UCL thesis formatting
%  guidelines.
%
% Valid options in terms of type are:
%  phd
%  mres
%  mphil
% \documentclass[12pt,phd,draft,a4paper,oneside]{ucl_thesis}
\documentclass[10pt,phd,a4paper,oneside]{ucl_thesis}

% Package configuration:
%  LaTeX uses "packages" to add extra commands and features.
%  There are quite a few useful ones, so we've put them in a 
%   separate file.
% -------- Packages --------

% This package just gives you a quick way to dump in some sample text.
% You can remove it -- it's just here for the examples.
\usepackage{blindtext}

% This package means empty pages (pages with no text) won't get stuff
%  like chapter names at the top of the page. It's mostly cosmetic.
\usepackage{emptypage}

% The graphicx package adds the \includegraphics command,
%  which is your basic command for adding a picture.
\usepackage{graphicx}

% The float package improves LaTeX's handling of floats,
%  and also adds the option to *force* LaTeX to put the float
%  HERE, with the [H] option to the float environment.
\usepackage{float}

% The amsmath package enhances the various ways of including
%  maths, including adding the align environment for aligned
%  equations.
\usepackage{amsmath}
\usepackage{amssymb}

% Use these two packages together -- they define symbols
%  for e.g. units that you can use in both text and math mode.
\usepackage{gensymb}
\usepackage{textcomp}
% You may also want the units package for making little
%  fractions for unit specifications.
%\usepackage{units}


% The setspace package lets you use 1.5-sized or double line spacing.
\usepackage{setspace}
\setstretch{1.5}

% That just does body text -- if you want to expand *everything*,
%  including footnotes and tables, use this instead:
%\renewcommand{\baselinestretch}{1.5}


% PGFPlots is either a really clunky or really good way to add graphs
%  into your document, depending on your point of view.
% There's waaaaay too much information on using this to cover here,
%  so, you might want to start here:
%   http://pgfplots.sourceforge.net/
%  or here:
%   http://pgfplots.sourceforge.net/pgfplots.pdf
%\usepackage{pgfplots}
%\pgfplotsset{compat=1.3} % <- this fixed axis labels in the version I was using

% PGFPlotsTable can help you make tables a little more easily than
%  usual in LaTeX.
% If you're going to have to paste data in a lot, I'd suggest using it.
%  You might want to start with the manual, here:
%  http://pgfplots.sourceforge.net/pgfplotstable.pdf
%\usepackage{pgfplotstable}

% These settings are also recommended for using with pgfplotstable.
%\pgfplotstableset{
%	% these columns/<colname>/.style={<options>} things define a style
%	% which applies to <colname> only.
%	empty cells with={--}, % replace empty cells with '--'
%	every head row/.style={before row=\toprule,after row=\midrule},
%	every last row/.style={after row=\bottomrule}
%}


% The mhchem package provides chemistry formula typesetting commands
%  e.g. \ce{H2O}
%\usepackage[version=3]{mhchem}

% And the chemfig package gives a weird command for adding Lewis 
%  diagrams, for e.g. organic molecules
%\usepackage{chemfig}

% The linenumbers command from the lineno package adds line numbers
%  alongside your text that can be useful for discussing edits 
%  in drafts.
% Remove or comment out the command for proper versions.
%\usepackage[modulo]{lineno}
% \linenumbers 


% Alternatively, you can use the ifdraft package to let you add
%  commands that will only be used in draft versions
%\usepackage{ifdraft}

% For example, the following adds a watermark if the draft mode is on.
%\ifdraft{
%  \usepackage{draftwatermark}
%  \SetWatermarkText{\shortstack{\textsc{Draft Mode}\\ \strut \\ \strut \\ \strut}}
%  \SetWatermarkScale{0.5}
%  \SetWatermarkAngle{90}
%}


% The multirow package adds the option to make cells span 
%  rows in tables.
\usepackage{multirow}


% Subfig allows you to create figures within figures, to, for example,
%  make a single figure with 4 individually labeled and referenceable
%  sub-figures.
% It's quite fiddly to use, so check the documentation.
%\usepackage{subfig}

% The natbib package allows book-type citations commonly used in
%  longer works, and less commonly in science articles (IME).
% e.g. (Saucer et al., 1993) rather than [1]
% More details are here: http://merkel.zoneo.net/Latex/natbib.php
%\usepackage{natbib}

% The bibentry package (along with the \nobibliography* command)
%  allows putting full reference lines inline.
%  See: 
%   http://tex.stackexchange.com/questions/2905/how-can-i-list-references-from-bibtex-file-in-line-with-commentary
\usepackage{bibentry} 

% The isorot package allows you to put things sideways 
%  (or indeed, at any angle) on a page.
% This can be useful for wide graphs or other figures.
%\usepackage{isorot}

% The caption package adds more options for caption formatting.
% This set-up makes hanging labels, makes the caption text smaller
%  than the body text, and makes the label bold.
% Highly recommended.
\usepackage[format=hang,font=small,labelfont=bf]{caption}

% If you're getting into defining your own commands, you might want
%  to check out the etoolbox package -- it defines a few commands
%  that can make it easier to make commands robust.
\usepackage{etoolbox}

% Needed for Author name, uses the correct comma below S and T for Romanian
\usepackage{combelow}  

% Add the papers as is, compiled on their own
\usepackage[enable-survey]{pdfpages}

% Subfigures, used in Aide-mémoire
\usepackage[caption=false,font=footnotesize,labelfont=sf,textfont=sf]{subfig}

% For code listing environments
\usepackage{listingsutf8}

% To import and not break relative links
\usepackage{import}

% Used in Global Macros for aesthetic et al. etc.
\usepackage{xspace}

% Used in macros for symbols and Profir's Name
\usepackage[utf8x]{inputenc} 

% for definition of new column type in AM
\usepackage{siunitx} 

% for new lines in table cells
\usepackage{makecell}

% aesthetically pleasing tables
\usepackage{booktabs}

% Allow Definition environments
\usepackage{amsthm}
\newtheorem{definition}{Definition}

% Switch to natbib to allow for citep and citet as used in papers
\usepackage{natbib}

% Some of the examples need tcolorbox
\usepackage{tcolorbox}

%allow valign of graphics
\usepackage[export]{adjustbox}

% For Flexeme's diagram
\usepackage{tikz}
\usetikzlibrary{calc,positioning}

% Used by some examples in Flexeme for specific Unicode
\usepackage[T1]{fontenc}

% Used for tables in Flexeme
\usepackage{tabularx}

% Used for the todo macro
\usepackage{ifthen}

% Used for better formatting of AM equations
\usepackage{multicol}


% Keep the main file clean, moved custom macros to the GlobalMacros file
\newcommand{\cf}{\hbox{\emph{cf.}}\xspace}
\newcommand{\deletia}{\ldots [deletia] \ldots}
\newcommand{\etal}{\hbox{\emph{et al.}}\xspace}
\newcommand{\eg}{\hbox{\emph{e.g.}}\xspace}
\newcommand{\ie}{\hbox{\emph{i.e.}}\xspace}
\newcommand{\scil}{\hbox{\emph{sc.}}\xspace} %scilicet: it is permitted to know
\newcommand{\st}{\hbox{\emph{s.t.}}\xspace}
\newcommand{\wrt}{\hbox{\emph{w.r.t.}}\xspace}
\newcommand{\etc}{\hbox{\emph{etc.}}\xspace}
\newcommand{\viz}{\hbox{\emph{viz.}}\xspace} %videlicet: it is permitted to see
\newcommand{\infinity}{\infty}
\newcommand{\green}{PineGreen}
% Gotta have todos
\newcommand{\todo}[1]{\textcolor{red}{TODO: #1}}

% Missing unicode chars, other brokenness in ucs/inputenc {{{1
\DeclareUnicodeCharacter{183}{\cdot}						% ·
\DeclareUnicodeCharacter{931}{\ensuremath\Sigma}			% Σ
\DeclareUnicodeCharacter{9001}{\ensuremath\langle}			% 〈
\DeclareUnicodeCharacter{9002}{\ensuremath\rangle}			% 〉
\DeclareUnicodeCharacter{9608}{\ensuremath\blacksquare}		% █
\DeclareUnicodeCharacter{1013}{\in}							% ϵ
\DeclareUnicodeCharacter{8213}{---}							% ―

\PrerenderUnicode{ț}

\theoremstyle{definition}
\newtheorem{defn}{Definition}[section]
\newtheorem{conj}{Conjecture}

% Sets up links within your document, for e.g. contents page entries
%  and references, and also PDF metadata.
% You should edit this!
%%
%% This file uses the hyperref package to make your thesis have metadata embedded in the PDF, 
%%  and also adds links to be able to click on references and contents page entries to go to 
%%  the pages.
%%

% Some hacks are necessary to make bibentry and hyperref play nicely.
% See: http://tex.stackexchange.com/questions/65348/clash-between-bibentry-and-hyperref-with-bibstyle-elsart-harv
\usepackage{bibentry}
\makeatletter\let\saved@bibitem\@bibitem\makeatother
\usepackage[pdftex,hidelinks]{hyperref}
% Needed for sane references, used in papers, must be after hyperref
\usepackage[nameinlink]{cleveref}
% Again, after hyperlink by requirement, used in Flexeme
\usepackage[noend]{algorithmic} % may not be installed
\usepackage{algorithm}
\makeatletter\let\@bibitem\saved@bibitem\makeatother
\makeatletter
\AtBeginDocument{
    \hypersetup{
        pdfsubject={Thesis Subject},
        pdfkeywords={Thesis Keywords},
        pdfauthor={Author},
        pdftitle={Title},
    }
}
\makeatother
    


% And then some settings in separate files.
% These settings are from:
%  http://mintaka.sdsu.edu/GF/bibliog/latex/floats.html

% They give LaTeX more options on where to put your figures, and may
%  mean that fewer of your figures end up at the tops of pages far
%  away from the thing they're related to.

% Alters some LaTeX defaults for better treatment of figures:
% See p.105 of "TeX Unbound" for suggested values.
% See pp. 199-200 of Lamport's "LaTeX" book for details.

%   General parameters, for ALL pages:
\renewcommand{\topfraction}{0.9}	% max fraction of floats at top
\renewcommand{\bottomfraction}{0.8}	% max fraction of floats at bottom

%   Parameters for TEXT pages (not float pages):
\setcounter{topnumber}{2}
\setcounter{bottomnumber}{2}
\setcounter{totalnumber}{4}     % 2 may work better
\setcounter{dbltopnumber}{2}    % for 2-column pages
\renewcommand{\dbltopfraction}{0.9}	% fit big float above 2-col. text
\renewcommand{\textfraction}{0.07}	% allow minimal text w. figs

%   Parameters for FLOAT pages (not text pages):
\renewcommand{\floatpagefraction}{0.7}	% require fuller float pages
% N.B.: floatpagefraction MUST be less than topfraction !!
\renewcommand{\dblfloatpagefraction}{0.7}	% require fuller float pages

% remember to use [htp] or [htpb] for placement,
% e.g. 
%  \begin{figure}[htp]
%   ...
%  \end{figure} % For things like figures and tables
\bibliographystyle{plainnat}   % For bibliographies

% Help avoid ambiguous destinations
\renewcommand{\theHsection}{\thepart.section.\thesection}
\renewcommand{\theHsubsection}{\thepart.subsection.\thesection}
\renewcommand{\theHsubsubsection}{\thepart.subsubsection.\thesection}

% These control how many number sections your subsections will take
%    e.g. Section 2.3.1.5.6.3
%  and how many of those will get put into the contents pages.
\setcounter{secnumdepth}{3}
\setcounter{tocdepth}{3}

\begin{document}

% At last, content! Remember filenames are case-sensitive and 
%  *must not* include spaces.
% Flags for content selection; 
% They should be before calls to \maketitle, \makedeclaration etc.
\setboolean{showconjoint}{false}
\setbool{useuclbanner}{true}
\uclbannerlocation{images/ucl-banner-a4.eps}

\title{Improving Software Project Health \\Using Machine Learning}
\author{Profir-Petru P\^ar\cb{t}achi}
\department{Department of Computer Science}


\extradeclaration{The work presented in this thesis %
is original work undertaken between September 2016 and July 2020 at %
University College London.\\%
\indent I list the papers that comprise this work in %
\Cref{chapter:introduction:sec:papers}. In the same section, I clearly detail %
my contributions towards each paper. They represent Chapers~\ref{chapter:am}, %
\ref{chapter:flexeme}, and \ref{chapter:posit} respectively.}

\maketitle
\makedeclaration

\begin{conjoint}
The thesis “Improving Project Health using Machine Learning” by Pârțachi
Profir-Petru, the undersigned, is composed of three main papers which were
written in collaboration with others. For each paper, I include the list of
authors herein and detail exactly my contributions. All authors have contributed
significantly to the writing of the papers; therefore, I detail only non-writing
work. Additionally, all work was done under the careful supervision of Earl
Barr, his insight has shaped all my work and our regular meetings were often
used to checkpoint and plan out my research tasks.

\begin{itemize}[leftmargin=*]
    \item[] \textbf{Aide-mémoire: Improving a Project’s Collective Memory via
Pull Request-Issue Links} 
    
    \noindent\emph{Authors: Profir-Petru Pârțachi, David R. White, Earl T. Barr}
    
    \noindent\emph{Venue: Submitted to ACM Transactions on Software Engineering
    and Methodology (TOSEM)}

    \noindent I have determined and mined the dataset of GitHub projects. I have
    implemented the proposed PR-Issue linker, performed the exploratory data
    analysis and feature engineering as well as wrote the evaluation and result
    analysis scripts. The experimental and EDA designs were done in close
    collaboration with David White who guided me patiently.

    \item[] \noindent\textbf{Flexeme: Untangling Commits Using Lexical Flows}
    
    \noindent\emph{Authors: Profir-Petru Pârțachi, Santanu Kumar Dash, Miltose
    Allamanis, Earl T. Barr}
    
    \noindent\emph{Venue: Proceedings of 28th ACM Joint European Software
    Engineering Conference and Symposium on the Foundations of Software
    Engineering, (ESEC/FSE 2020)}

    \noindent  I have implemented in full the construction of our new structure,
    though the idea of the structure was from Earl Barr, and the details of the
    specification were worked in close collaboration. I also implemented the
    necessary methods to construct the dataset, reproduced previous work in the
    area and implemented our proposed untangling algorithms as well as their
    evaluation on the constructed dataset. The RefiNym implementation was
    provided by Santanu Dash who helped me integrate it with Flexeme. The
    original PDG extraction implementation for C\# code was provided by Miltose
    Allamanis. Santanu Dash and I performed the manual evaluations.
    
    %%% !Remember to check if this is still needed if you edit text above it!
    \pagebreak
    
    \item[] \noindent\textbf{POSIT: Simultaneously Tagging Natural and
    Programming Languages} 
    
    \noindent\emph{Authors: Profir-Petru Pârțachi, Santanu Kumar Dash, Christoph
    Treude, Earl T. Barr}
    
    \noindent\emph{Venue: Proceedings of 42nd International Conference on
    Software Engineering (ICSE ’20)}

    \noindent I have implemented the preprocessing scripts, the Neural Network
    that realises POSIT, the adaptation of the previous State-of-the-art to our
    problem, and the necessary evaluation scripts. The Code Comments corpus was
    provided by Santanu Dash. The original implementation of TaskNav was
    provided by Christoph Treude. The informal proof of context-sensitivity of
    mixed-text was worked on in close collaboration with Earl Barr without whom
    the proof would have not been finished in a timely manner. Manual evaluation
    of POSIT was done together with Santanu Dash, while manual evaluations of
    TaskNav augmented with POSIT were done together with Christoph Treude.

\end{itemize}

\doublesignature{Earl T. Barr}
\end{conjoint}

\begin{abstract} % 300 word limit
In recent years, systems that would previously live on different platforms have
been integrated under a single umbrella. The increased use of GitHub, which
offers pull-requests, issue tracking and version history, and its integration
with other solutions such as Gerrit, or Travis, as well as the response from
competitors, created development environments that favour agile methodologies by
increasingly automating non-coding tasks: automated build systems, automated
issue triaging \etc In essence, source-code hosting platforms shifted to
continuous integration/continuous delivery(CI/CD) as a service. This facilitated
a shift in development paradigms, adherents of agile methodology can now adopt a
CI/CD infrastructure more easily. This has also created large, publicly
accessible sources of source-code together with related project artefacts:
GHTorrent and similar datasets now offer programmatic access to the whole of
GitHub.

Project health encompasses traceability, documentation, adherence to coding
conventions, tasks that reduce maintenance costs and increase accountability,
but may not directly impact features.  Over focus on health can slow velocity
(new feature delivery) so the Agile Manifesto suggests developers should travel
light --- forgo tasks focused on a project health in favour of higher feature
velocity. Obviously, injudiciously following this suggestion can undermine a
project's chances for success.

Simultaneously, this shift to CI/CD has allowed the proliferation of Natural
Language or Natural Language and Formal Language textual artefacts that are
programmatically accessible: GitHub and their competitors allow API access to
their infrastructure to enable the creation of CI/CD bots. This suggests that
approaches from Natural Language Processing and Machine Learning are now
feasible and indeed desirable. This thesis aims to (semi-)automate tasks for
this new paradigm and its attendant infrastructure by bringing to the foreground
the relevant NLP and ML techniques.

Under this umbrella, I focus on three synergistic tasks from this domain: (1)
improving the issue-pull-request traceability, which can aid existing systems to
automatically curate the issue backlog as pull-requests are merged; (2)
untangling commits in a version history, which can aid the beforementioned
traceability task as well as improve the usability of determining a fault
introducing commit, or cherry-picking via tools such as git bisect; (3)
mixed-text parsing, to allow better API mining and open new avenues for
project-specific code-recommendation tools.
\end{abstract}

\begin{impactstatement}
In the modern world, software is ubiquitous: from the personal computers most of
us use to the mission critical systems that require fine control. Software
quality is usually a result of a healthy software project, \ie it is not just a
function of the code, rather also of the process producing that code.
Further, projects are not just source-code; indeed, they contain a wealth of
documents written in English or other languages. Focusing strictly on
source-code tells us only half the story.

In this thesis, I propose the notion of project health; a notion that was
previously colloquially understood and that encompasses those processes that
help project succeed and flourish. Under this umbrella, I focus on three tasks
that could impede project success: pull-request-issue linking, commit separation
into atomic patches, and mixed-text segmentation and pre-processing. I focus on
tasks that would break assumptions made by researchers when proposing techniques
in the first two, while the latter presents a new way of handling data enabling
analysis which is aware of algorithmic and natural language information. By
borrowing techniques and methodology from Machine Learning, I propose prototype
systems to resolve these issues.

The work I present in this thesis follows the ethos of helping developers help
themselves and us. We, as researchers, help developers maintain project health
with its inherent benefits: easier onboarding, lower costs of maintenance \etc
This in turn can create better training data for researchers and may enable yet
better automation techniques and research avenues for Big Code. Further, all
tooling created during this thesis is Open Sourced and made available under the
MIT license for developer or researchers to use directly.
\end{impactstatement}

\begin{acknowledgements}   
A PhD is not an easy journey, but one I will look back on fondly. Along the way,
I have had the support of many for whom I wish to dedicate the following
paragraphs in thanks.

I would like to thank my supervisor, Dr Earl T. Barr, for the sage advice and
importantly patience with me especially as I was learning the art of
academic writing. I would also like to thank Earl for bearing with me as I fell
ill throughout the journey and offering support during those times. The
dialectic we had trying to formally prove a hypothesis with the clock ticking to
the deadline will be a fond memory for years despite the stress of the moment
then.

I would also like to thank Dr David R. White for the `Desk Compensation
Meetings' that set me on the path towards rigorous experimental design early in
my PhD. That formed the support beams for my empirical studies throughout my
doctoral work. I would also like to thank David for being there for me when I
needed someone to listen to me and helping navigate the academic landscape with
more confidence.

For the help and patience they showed, I would also like to thank my co-authors
during my PhD research. I would like to thank Dr Santanu Dash, for helping me
grok the rougher sides of Roslyn and enabling the work that followed from that.
I would also like to thank Santanu for helping me pursue an idea that later
became a paper by helping me with initial labelled data that would have been
impossible without his CLANG knowledge. His help and advice on the subtilties of
formal languages helped shape my work.

I extend similar gratitude to Miltose Allamanis and Christoph Treude, who have
made invaluable contributions and were a pleasure to work with as co-authors.
Miltose has helped me get a better start writing Machine Learning code, while
Christoph helped me better understand how to do qualitative and manual
quantitative analysis

I would like to thank my lab mates in CREST and SSE that made the journey more
fun: Leo Jeoffe, David Kelly, Bobby Bruce, DongGyun Han, Matheus Paix\~ao,
Carlos Gavidia, Giovani Guizzo, Vali Tawosi, Rebecca Mousa, Jie Zhang. The
informal chats were invaluable and they all helped me through the journey.
Discussing our work together helped me, and I hope them as well, better
understand how to do proper science, to better understand how and when
statistical techniques should be applied, how to formulate experiments, how to
thoroughly explore hypothesises. I also extend my gratitude to my close friend,
Tudor Haru\cb{t}a, with whom I oft shared my progress and who was always
available to chat when I needed to take my mind off of research.

Last but not by any means least, I would like to thank my family, my dear older
sister for always bearing with me and lending a ear to listen and a pair of eyes
to double check my English. My parents for supporting me through my PhD years
and even donating time on the home PC when I needed extra compute resources on a
tight deadline.

Our research is shaped by those in our network, those we talk to even causally,
thus the sum total of the contributions to this research is beyond any
quantifiable scope; I cannot hope to enumerate everyone who has contributed to
my work in the multitude of ways that I have been supported along on this
journey. To everyone, thank you.
\end{acknowledgements}

\setcounter{tocdepth}{2} 
\hypersetup{linkcolor = {black}}
\tableofcontents
\listoffigures
\listoftables
\hypersetup{linkcolor = {blue}}


\chapter{Introduction}
\label{chapter:introduction}

The software development process involves a multitude of aspects: the
specification and design, the architecture of the system, the code writing
itself, code version histories, issue tracking, code review. In an opensource
project external collaboration and their integration are also a part of it. Last
but not least, the developers themselves and the project culture further define
the development process. Some projects are vibrant; some moribund; others
implode after a few commits. \emph{Project health} determines project success.
Project health encompasses: technical, traceability, and social aspects of a
project. It is defined by the quality of code, documentation, version histories,
issue tracking, specifications and requirements. It is further defined by the
quality of the interlinking of these aspects. Finally, but not less important,
it is defined by the culture of the project, be that explicitly codified in a
file/manifesto or implicit in the interactions of developers. In this work, I
focus only on the first two aspects of project health: technical and
traceability. It is of no surprise that the artefacts produced by these
processes span a profusion of formats and both natural as well as programming
languages, even intermixed.

In recent years systems, that would previously live on different platforms have
been integrated under a single umbrella, thus the different aspects that can
impact project health can be centrally observed. The proliferation of GitHub,
which offers pull-requests, issue tracking and version history, and its
integration with other solutions such as Gerrit (for Code Reviews), Travis (for
continuous integration and delivery), as well as the response from competitors,
such as GitLab and Atlassian who offer equivalent offerings has led to leaner
and faster development cycles: GitHub advertises their CI offerings with a
promise to reduce time spent merging commits or debugging and increase time
spend writing code~\cite{GitHubCI}. This has also reduced the cost of entry and
created large, publicly accessible sources of source-code together with related
project artefacts, such as GHTorrent~\cite{GHTorrent},
CodeSearchNet~\cite{Husain2019} and similar datasets. Within this context,
developers can benefit from the automation of tasks, especially if they are oft
forgotten or ignored in favour of traveling light. 

This shift in tooling has also facilitated a shift in development paradigms,
developers now prefer a continuous integration/continuous delivery
infrastructure. This has led to more projects adopting a more agile development
process. Cleland-Huang~\etal~\cite{Cleland-Huang2014} and St{\aa}hl
\etal~\cite{Stahl2017} observe that, for traceability to be adopted, within an
agile framework, it must remain within the traveling light paradigm.
Simultaneously, the shift to continuous improvement/continuous development has
allowed for the proliferation of Natural Language or Natural Language and Formal
Language textual artefacts which are programmatically accessible. This suggests
that approaches from Natural Language Processing and Machine Learning are now
feasible and indeed desirable. This thesis aims to (semi-)automate tasks for
this new paradigm and its attendant infrastructure by bringing to the foreground
the relevant NLP and ML techniques.

\section{Problem Statements}
\label{chapter:introduction:sec:problem_statement}

Under Improving Project Health, I focus on three synergistic tasks: (1)
improving the issue-pull-request traceability, which can aid existing systems to
automatically curate the issue backlog as pull-requests are merged; (2)
untangling commits in a version history, which can aid the beforementioned
traceability task as well as improve the usability of determining a fault
introducing commit, or cherry-picking via tools such as git bisect; (3)
mixed-text parsing, to allow better API mining and open new avenues for
project-specific code-recommendation tools. 

Software development is a community task. The programming task itself is a
component of a much wider process. It is, however, a task that is viewed
exclusively when considered by developers at the cost of other administrative
tasks. This manifests itself as, for example, in the case of traceability, which
companies recognise as important, by organisations reaching an insufficient
level of traceability~\cite{mader2009motivation}. These administrative tasks
come with the promise of facilitating internal follow-ups, evaluation or
improvement of the development process~\cite{domges1998adapting}. Agile
practice, and especially Continuous Integration, by suggesting a lean approach
to documentation, reduces the direct applicability of existing approaches,
though work has been done to adapt to the new paradigm:
Ståhl~\etal~\cite{Stahl2017} provide the Eiffel framework, which allows
integrating different sources of traceability information in a light-weight
manner. Indeed, the conflict arises between the traditional assumptions of
Traceability and the agile team's tendencies of traveling quite
light~\cite{Cleland-Huang2014}. Indeed, \emph{traveling light} means forgoing
documentation, user stories not being punted to new sprints, or, more generally,
administrative tasks that distract from a developer coding being ignored. If
combined with backlog refinement~\cite{BacklogRefinement}, traveling light can
manifest itself as a form of project amnesia. This is a form of technical debt,
and will degrade project health in the long-run.

To help developers, tool smiths themselves require supporting artefacts and
utilities. One of these is access to high quality data that can be used both to
train a human intuition for a task as well as statistical approaches to enable
developers from less well maintained projects improve their situation. The
proliferation of web-sites such as GitHub, BitBucket/JIRA, GitLab, and others
that enable developers to have commits, issues, pull-request, code-reviews and
other development processes under one umbrella also opens a potential resource
to researchers and tool smiths that wish to study the properties of the
inter-relationships of such artefacts. In this context, traceability can help
developers maintain the collective memory of projects. When using such
resources, the promise of automatic triaging of issues as commits resolving them
are accepted into the mainline incentivises developers to maintain
code-review/issue to commit links. I have, however, observed a lack of such
links as far as GitHub is concerned, despite an encouragement of contributors to
do so in the contribution guidelines (\Cref{am:sec:eval:prelim}). Further,
developers only partially recording such link introduces bias in the datasets
researchers use.

\begin{tcolorbox}[title=Pull-request-Issue Traceability]
    Given a project, developers should receive traceability link suggestions
    when pertinent, \ie when they are already performing a task within which a
    link can be recorded. For a pull-request-issue traceability, suggestions are
    to be made at issue triage and PR submission.
\end{tcolorbox}

Additionally, links may not be clear if the fixes are presented as unfocused
patches, hiding bug fixing beyond many lines of refactoring.
Herzig~\etal~\cite{Herzig2016} (Herzig and Zeller~\cite{Herzig2013}) show that
tangled commits introduce bias in defect prediction. In our work, I aim to
provide a diff time solution for commit untangling to help developers reduce
this bias. The promise to developers is enabling tooling such as git bisect.

\begin{tcolorbox}[title=Commit Untangling]
    Given a commit, the patch should be decomposed into smaller patches such
    that each patch tackles a single task, \ie would be linked to a single issue
    or implements a single stakeholder concern. On atomic commits, this process
    should be the identity function.
\end{tcolorbox}

Despite the international nature of platforms such as GitHub, a significant
portion of software development information is in English. This leads to
projects in other natural languages to adopt English terms within their
processes as observed by Treude~\etal~\cite{Treude2015portuguese}. Further, such
natural language can mix any natural languages it may use with formal languages.
This same phenomenon is observable on mailing lists, such as the Linux Kernel
Mailing List, and developer fora, such as StackOverflow.  Further, despite
access to mark-up to help developers demarcate code from English, in practice,
even on StackOverflow submitters do forget to do so. This reduces the applicable
research tools which use such formatting cues.

\begin{tcolorbox}[title=Mixed-text Parsing]
    Given an artefact that freely mixes natural languages and formal languages,
    separate the languages. Further, for each token, provide the tag in the
    language from which the token is taken. In natural languages, this tags are
    part-of-speech tags; in formal languages, these tags are the AST tags of the
    parents of the terminals.
\end{tcolorbox}

\noindent The tasks range from more developer focused to more researcher
focused; however, all tasks aim to always provide a benefit to developers,
directly or indirectly.

Overall, the focus of this research focuses on the creation of tools and
meta-tools that (semi-)automate development tasks to enable project to remain
healthy. PR-issue traceability focuses directly on improving the development
process by reducing the administrative burden on the developers, while commit
untangling and mixed-text parsing focus on enabling a wider range of tools to be
applicable. The goal as a whole is an improvement of the development process by
allowing more time to be dedicated to the programming task rather than tasks
required to maintain the project health.

\section{Contributions}
\label{chapter:introduction:sec:contrib}

Under the three synergistic tasks, I have found the following ways of improving
the state-of-the-art, which can benefit both practitioners as well as
researchers; This thesis:

\begin{enumerate}
    \item Provides an online pull-request traceability link inference algorithm
    realised as Aide-mémoire. It solves an online, modern variant of the
    ``Missing Links'' problem proposed by Bachman~\etal~\cite{MissingLinks}. It
    departs from using commits in favour of pull-requests. Aide-mémoire suggests
    links when developers are already handling a PR or an issue.

    \item Forwards the state-of-the-art in commit untangling both in accuracy
    and in speed and thus enables the creation of better statistical tooling for
    developers indirectly, while directly allowing them to improve the health of
    their version histories. I realise this as Flexeme. Flexeme offers precise
    untangling suggestions within 10 seconds; fast enough to be viable in the
    code-review process.

    \item Formulates the mixed-text parsing problem, posing it as a simultaneous
    segmentation and tagging task. It proposes a solution, POSIT, which borrows
    from the Natural Language Processing and Linguistics literature that
    concerns itself with code-switching. It maps the borrowed results to the
    Software Engineering context and allows extending methods to languages other
    than English or resources that freely mix natural and formal languages.
\end{enumerate}

The solutions I propose can help developers directly, by improving the state of
the project artefacts, or indirectly by helping tool smiths. Further, these
solutions interact synergistically, the Flexeme and POSIT improve the quality of
the training data for Aide-mémoire, while also improving the general health of
developer version histories (Flexeme) or developer issue trackers and fora
(POSIT).

\section{List of Papers}
\label{chapter:introduction:sec:papers}

A long research project, like a PhD, is a close collaboration where it is
difficult, often impossible, to separate each collaborator's contributions. What
is clear is who took the lead on which aspects and tasks. The reader will also
notice a mixed use of `I' and `we', I continue to use `we' in the paper chapters
when referring to work done in close collaboration with my co-authors to
acknowledge their contribution. Further, the papers are presented as published
or submitted for review spare figures and tables that required editing to fit
into the thesis format. I now enumerate the tasks I lead in each of the papers
in this thesis.

\begin{itemize}[leftmargin=*]
    \item[]\noindent\textbf{1. Aide-mémoire: Improving a Project’s Collective Memory via Pull
Request-Issue Links} 

    \noindent\emph{Authors: Profir-Petru Pârțachi, David R. White, Earl T. Barr}

    \noindent\emph{Venue: Submitted to ACM Transactions on Software Engineering
    and Methodology (TOSEM)}

    \noindent I determined and mined the dataset of GitHub projects. I
    implemented the proposed PR-Issue linker, performed the exploratory data
    analysis and feature engineering as well as wrote the evaluation and result
    analysis scripts. The experimental and EDA designs were done in close
    collaboration with David White who guided me patiently.

    \item[]\noindent\textbf{2. Flexeme: Untangling Commits Using Lexical Flows}

    \noindent\emph{Authors: Profir-Petru Pârțachi, Santanu Kumar Dash, Miltos
    Allamanis, Earl T. Barr}

    \noindent\emph{Venue: Proceedings of 28th ACM Joint European Software
    Engineering Conference and Symposium on the Foundations of Software
    Engineering, (ESEC/FSE 2020)}

    \noindent I implemented in full the construction of our new structure,
    though the idea of the structure was from Earl Barr, and the details of its
    specification were worked out in close collaboration with him. I also
    implemented the necessary methods to construct the dataset, reproduced
    previous work in the area and implemented our proposed untangling algorithms
    as well as their evaluation on the constructed dataset. The RefiNym
    implementation was provided by Santanu Dash who helped me integrate it with
    Flexeme. The original PDG extraction implementation for C\# code was
    provided by Miltos Allamanis. Santanu Dash and I performed the manual
    evaluations.

    \item[]\noindent\textbf{3. POSIT: Simultaneously Tagging Natural and Programming
    Languages} 

    \noindent\emph{Authors: Profir-Petru Pârțachi, Santanu Kumar Dash, Christoph
    Treude, Earl T. Barr}

    \noindent\emph{Venue: Proceedings of 42nd International Conference on
    Software Engineering (ICSE ’20)}

    \noindent I implemented the preprocessing scripts, the Neural Network
    that realises POSIT, the adaptation of the previous state-of-the-art to our
    problem, and the necessary evaluation scripts. The Code Comments corpus was
    provided by Santanu Dash. The original implementation of TaskNav was
    provided by Christoph Treude. The informal proof of context-sensitivity of
    mixed-text was worked on in close collaboration with Earl Barr without whom
    the proof would have not been finished in a timely manner. The manual
    evaluation of POSIT was done together with Santanu Dash, while the manual
    evaluations of TaskNav augmented with POSIT were done together with
    Christoph Treude.
\end{itemize}

\section{Thesis Organisation}
\label{chapter:introduction:sec:organisation}

The remainder of the thesis is organised as follows.

\Cref{chapter:literature} first presents literature on the areas encompassed by
project health. It then focuses on the areas touched by the proposed tasks in
the order they are presented in the thesis. It first discusses traceability and
the ``Missing Links'' problem~\cite{MissingLinks}, it then focuses on the issues
caused by tangled commits and approaches to the commit untangling, it concludes
with an exploration of existing approaches to mixed-text: how do researchers
tackle code when applying natural language techniques and how the natural
language processing community handles the mixing of multiple natural languages.

\Cref{chapter:am} presents Aide-mémoire, a solution to online pull-request-issue
linking problem. It extends the ``Missing Links'' problem due to
Bachmann~\etal~\cite{MissingLinks} to pull-requests and proposes a solution that
can live along side the code-review process. Aide-mémoire is evaluated on an
extensive corpus of GitHub projects.

\Cref{chapter:flexeme} presents Flexeme, a solution to the commit untangling
problem. It shows that Flexeme improves the state-of-the-art while being fast
enough to be viable at during code-review. Flexeme is evaluated on an artificial
corpus that is rigorously constructed to mimic tangled commits
Herzig~\etal~\cite{Herzig2016} have observed developers make.

\Cref{chapter:posit} presents POSIT, where the mixed-text parsing problem is
introduced together with our proposed solution: POSIT. In it, we argue that the
mixed-text parsing problem is context-sensitive in the general case and hence
propose a distributional semantics approach, realised as a neural network, for
the problem. We show that POSIT can both directly improve software fora by
suggesting mixed code annotations, as well as aid other research tools: we show
POSIT to improve TaskNav's~\cite{Treude:2015:TTN:2819009.2819128} recall.

\Cref{chapter:conclusions} concludes the thesis and provides directions for
future work.
\chapter{Literature Review}
\label{chapter:literature}

In this chapter, I first present the wider literature that concerns developer
activities impacting project health in
\Cref{chapter:literature:sec:project_health}. I then focus on the subproblems
tackled in the thesis and start by presenting Traceability and Commit-Issue
linking literature in \Cref{chapter:literature:sec:am_rel_work}.
\Cref{chapter:literature:sec:flexeme_rel_work} presents the literature around
commit untangling. I conclude this chapter by describing the literature around
mixed-text in \Cref{chapter:literature:sec:posit_rel_work}.

\section{The Many Faces of Project Health}
\label{chapter:literature:sec:project_health}

As project health comprises different aspects of a project --- from technical,
through design, to social --- it is natural that it touches many areas of active
interest from researchers. Some of the more prominently researched areas are
technical debt, human aspects of software engineering. I will first present
technical debt literature and its relation to project health. I then will shift
focus to the area of software evolution and maintenance, with a primary focus on
how it helps projects remain active. I conclude this section by presenting
recent research in human aspects of software engineering, with a focus on
improving the development process by removing negative incentives.

Technical debt primarily arose as a concept from a metaphor relating it to
financial debt and serving to capture the consequences of poor software
development~\cite{cunningham1992wycash}. The core concept was that debt is not
necessarily bad, it can be used as a asset to speed up development at a crucial
point so long as it is repaid. While this provided an intuitive basis for the
concept, the lack of a formal definition left the boundaries fuzzy. 

Tom \etal~\cite{tom2012consolidated} found that this created an apparently
fragmented understanding from a number of both themes and anecdotal accounts
that fail to fully capture the phenomenon. Later, Tom \etal conduct a multivocal
literature review~\cite{tom2013exploration} within which they determine the
dimensions of technical debt, its attributes, and provide a taxonomy for it. Tom
\etal separate technical debt into: code debt, design and architectural debt,
environment debt, knowledge distribution and documentation debt, and testing
debt:
\begin{itemize}
    \item Code debt covers hacks or sloppy code that was written and would incur
    heavy maintenance costs.
    \item Design and architectural debt comes mainly in two flavours: an upfront
    design that did not factor maintenance or scale or piecemeal design that was
    done without the required refactoring; further, this debt could also arise
    as a consequence of a sub-optimal architectural solution, be it sub-optimal
    at the time of design or as technology progresses.
    \item Environmental debt covers software, hardware, dependencies, or even
    lack of automation. It manifests itself as suboptimal use of labour, reduced
    development velocity or even vulnerabilities.
    \item Knowledge distribution and documentation cost covers aspects related
    to having a high `bus factor', when a complex system is maintained by few
    knowledgeable individuals and the process is not documented, a company risks
    a sudden increase of technical debt should the individuals leave.
    \item Testing debt covers the need for manual quality assurance when the
    task can be automated. While it shares the automation aspect with
    Environmental debt, it has the associated risk of brand damage if a faulty
    patch is pushed through to production.
\end{itemize}  

As impact on project health, in the short term, technical debt allows a boost in
project velocity; and should this debt never come due, you may feel free to
incur high amounts~\cite{tom2013exploration}; however, in the long term, it can
create brittle code, increase the cost of new features or products by decreasing
reusability, and even increase production costs by decreasing long term project
velocity as bandwidth is taken by debt repayment or simply attrition from the
sub-optimal products and development pipelines. Tom \etal further found that
this can have an impact on not just productivity, quality, and risk, but also
developer morale as work time is taken on repaying debt and working in a brittle
system that is difficult to reason about.

While Tom \etal's~\cite{tom2013exploration} definition of technical debt covers
some of the impacts from developers on project health: such as low morale, low
documentation, sub-optimal tooling; it does not cover the scenario when the
development process itself has negative incentives. Indeed, a development or
maintenance process, such as bug triaging, may have hidden incentives that, when
each participant behaves optimally for themselves, causes sub-optimal outcomes
for the project, such as wasted development time. The nature of this issue lends
itself well to game-theoretic approaches. Gavidia-Calderon
\etal~\cite{gavidia2020game, gavidia2019assessor} propose that anomalies in the
software development process be modelled using Empirical Game Theory so that a
process intervention can be devised in a way that avoids negative incentives.
They demonstrate their approach using simulation based examples that cover bug
priority inflation, acquisition of technical debt in a simple fix or kugel
example, and the tragedy of the test suite (the accumulation of test cases due
to incorrect developer incentivisation). 

When technical debt cannot be avoided, a risk associated with technical debt is
its interest rate. The work I present in this thesis aims to, first, reduce the
need to accrue technical debt by (semi-)automating tasks that would otherwise be
forgone for a leaner project. When debt is acquired, the work herein aims to
reduce the interest rate. Recasting the perspective into the framework presented
by Gavidia-Calderon \etal, the thesis aims to reduce the cost of actions
associated with good long-term project health in the pay-off matrix, \ie to
ensure that short horizon optimal decisions are also optimal in a long
horizon setting.

\section{Bridging the Gap between Issues and History}
\label{chapter:literature:sec:am_rel_work}

A method of combating accruing knowledge distribution debt is by maintain a high
degree of traceability --- interlinking of development and design artefacts by a
`are related/implements/specifies etc.' link. In this section, I focus on
software traceability as an area, and the position of the commits to issues
traceability task, which we generalise and propose a solution for in
\Cref{chapter:am}, within this wider context.

\subsection{Modern Development}
\label{chapter:literature:sec:am_rel_work:modern_dev}

Modern development increasingly relies on tooling that integrates version
control, issue tracking, wikis, continuous integration and continuous deployment
under a single system. Notable examples are GitHub and Atlassian's JIRA. This
new development paradigm poses new problems and opportunities. Kalliamvakou
\etal~\cite{Kalliamvakou2014} elucidates these, using GitHub as their archetypal
example. A particular opportunity Kalliamvakou~\etal identify is this paradigm's
integration of Version Control and Issue Tracking with the potential to
interlink them. This brings in opportunities for those following in the
footsteps of Bachmann \etal's Missing Links Problem~\cite{MissingLinks}, as they
show that indeed this integration is not fully realised.

Lack of PR-issue links is an ongoing problem in modern software development.  So
much so, Agile practice specifies spring cleaning an issue (a user story in
Agile terminology) backlog. During backlog refinement, developers remove stale
stories and reprioritise and re-estimate remaining stories. When all stories are
stale, this practice discards all sprint-related artefacts --- issues, feature
requests, user stories, as well their links --- in favour of starting the next
sprint from a clean slate~\cite{BacklogRefinement}. Projects must resort to this
spring cleaning all too often~\cite{laurieTrattPersonalCommunciation}. This
practice loses user stories, documentation, issues, discussions, and other such
artefacts associated with a sprint by design; this loss of information comes at
the cost of higher maintenance costs, onboarding costs \etc as the information
has to be recreated or reverse engineered. The loss of documentation it entails
it just one example. This can exacerbate the accumulation of knowledge
distribution and documentation debt. Adoption of PR-issue inference tooling
promises to reduce the need to resort to this drastic measure by enabling
automatic issue triaging.

\subsection{Software Traceability}
\label{chapter:literature:sec:am_rel_work:tracability}

Requirements engineering (RE) focuses on stakeholders, decision makers, and
their artefacts: requirements, documentation, specification, and design or
architectural documents. These artefacts tend to be natural language, text or
speech, and often go unrecorded. When they are recorded, they exist in multiple
formats, including spreadsheets, email, figures, and printed material. They
further encompass developer artefacts, such as source code, pull-requests,
commits, and issues, but do not focus on them. Further, within an agile context,
documentation is often forgone in favour of staying lean and traveling light.
This further complicates an already difficult task and suggests that tooling
that would target such a paradigm should be light-weight~\cite{Stahl2017}.

\emph{Software traceability} seeks to infer \emph{traces} (\ie links) between
these heterogeneous artefacts~\cite{Cleland-Huang2014}. Missing or hard-to-parse
artefacts greatly complicate trace recovery, which is why much work on
traceability seeks to provide tooling to decision makers to capture or parse
these artefacts and persuade them to use it~\cite{Neumuller06, jiralinkdoc,
ghlinkdoc}. These tools must often record decision maker or developer
interactions, with each other or their tools. They must also avoid being either
disruptive, requiring the developer to switch contexts, or invasive, as when
they require developers to change their workflow or use instrumented
IDEs~\cite{TopicTraceability}, raising privacy and deployability concerns.
Inferring traces over these heterogeneous artefacts, as a consequence of their
heterogeneity, can only leverage abstract, generic features. 

As many of the artefacts within such systems are textual, work in this area
borrowed techniques from Information Retrieval, including ourselves. Borg
\etal~\cite{Borg2014} provide an excellent survey detailing the use of such
methods within traceability. Further, Mills
\etal~\cite{Mills:2017:ATL:3106237.3121280} propose reducing the human effort
required by IR based techniques via a classification step after the recommender
that filters the suggestions. Finally, St{\aa}hl \etal~\cite{Stahl2017} propose
a light-weight framework where different systems can emit traceability events,
facilitating a standardisation of how such links are recorded by minimising the
cost of integrating otherwise incompatible tools within a single eco-system,
which they call the Eiffel framework.

Asuncion and Taylor~\cite{Asuncion:2009:CCL:1556908.1557008} use record-replay
and hypermedia to tackle software traceability as an online problem. Their
online solution cannot employ offline information retrieval techniques (such as
VSM, tf-idf or topic modelling) commonly used in retrospective approaches to
traceability~\cite{4249808}. They focus on requirements and specifications,
architectural modules, source files, and test cases, making no mention of PRs or
issues beyond bug reports. Previous work required expertise in formal
modelling~\cite{Pohl96, Pinheiro96}. To eliminate this barrier to entry, they
instrument developer or decision maker tools, raising the usual deployability
and privacy concerns. Indeed, the solution they present invasively instrumented
specific versions of no less than five different tools, including general
purpose tools like Firefox and MS Word. Many of these tools are under constant
development; updating this instrumentation out-of-tree is extremely expensive,
suggesting that St{\aa}hl~\etal's light-weight approach is more likely to be
maintainable. Asuncion~\etal~\cite{TopicTraceability} in follow-up work
incorporate an LDA model into their framework to improve the interpretability of
their traces. To remain online, they recompute an LDA model on demand for each
graph visualisation and search query.

Additionally, Falessi~\etal~\cite{Falessi2017} have done work towards
quantifying the number of links that are left to be recovered. Their work
assumes an offline, closed-world setting; they provide a framework that can help
an analyst decide when to stop pursuing a traceability maintenance task. The aim
of this work is to provide a quantifiable signal of when the traceability task
should be finished as linking can be considered good enough.

Work in general traceability, due to its broader nature, deals with a profusion
of formats and has, so far, either resorted to intrusive
instrumentation~\cite{Asuncion:2009:CCL:1556908.1557008, TopicTraceability}, or
defining a protocol which tools must implement and communicate
via~\cite{Stahl2017}. The more limited scope of this work allows us to exploit
known structures that developers using platforms such as GitHub naturally
produce: namely issues and pull-requests. This enables Aide-mémoire to rely on
API access from GitHub, or, with minimal additional effort, JIRA, rather than
instrument the developer environment (\Cref{am:sec:imp:deployability}).

\subsection{Commit-Issue Link Inference}
\label{chapter:literature:sec:am_rel_work:cli}

The \emph{missing link problem} is the offline prediction problem of inferring
missing commit-issue links given a version history and issue tracker archive.
Bachman~\etal were the first to formulate and quantify this
problem~\cite{Bird2009,MissingLinks}. Their work aids developers indirectly by
helping researchers and tool-smiths avoid the bias introduced by missing links
that could undermine their techniques or tools. Specifically, they show that by
assuming recorded links to be representative of all links, tools are biased to
use code from more experienced developers, thus not learning from mistakes or
bugs introduced by less experienced contributors. 

Bachman~\etal together with Apache Commons developers manually supplied missing
links and published their Apache Commons corpus. Wo~\etal are the first to
attempt to automatically infer them. They propose ReLink~\cite{relink}, which
measures the similarity of change logs and bug reports with cosine similarity on
tf-idf vectors, learning a threshold for true links. Nguyen~\etal exploited
commit and issue tracker metadata in MLink~\cite{MLink} to improve recall over
ReLink. They evaluated MLink on the Apache Commons corpus, making it the de
facto standard. ReLink and MLink first consider only commit data to form an
initial set of commit-issue links that they then filter. Prechlet and
Pepper~\cite{prechelt2014bflinks} dispense with the initial commit-only stage,
and instead consider both commit and issue data from the start. They argue this
bi-directional inference is more sound. Their BFLinks proposes two link
predictors (based on bug and commit IDs) and a series of filters to reduce the
candidate set of links.

RCLinker~\cite{RCLinker} further improves recall. RCLinker relies on
ChangeScribe~\cite{ChangeScribe} to produce textual descriptions of commits,
especially those that lack commit messages. PRs have their own message and
aggregate multiple commits and their messages.  
This fact alleviates the problem of sparse commit descriptions when considering
issue-PR interlinking. PR descriptions for some projects can be of low quality,
impacting those that would rely on them. Liu \etal~\cite{liu2019automatic}
propose a tool, based on a bi-directional RNN with a copy network, to tackle PR
summarisation, which can help side-step the low-quality PR descriptions.
\Cref{am:sec:reproduction} provides a more detailed description of RCLinker.

Sun~\etal~\cite{FRLink} used non-source files in commits for commit-issue link
inference. They argue that these files are important for capturing developer
intent. They use the standard heuristics, such as checking for camelCase or
snake\_case, to determine the relevancy of a non-source file in a commit. As is
conventional, they implement these heuristics as regexes. They use the resulting
set of non-source files together with the co-committed source files to compute
textual features. They scan a preset and fixed list of the similarity thresholds
to find the maximum F1-score where Recall is at least 0.80. The procedure raises
two unanswered questions. First, how did the authors determine the threshold
granularity? Second, how does training FRLink on F1-score for a task whose
performance is measured in terms of F1-score avoid overfitting? They report
these choices allow FRLink to improve Recall over previous work while matching
or improving F1-score. 

Sun~\etal evaluate FRLink on a new corpus of GitHub projects. This corpus
differs from the Apache corpus used in prior work in the conventions governing
commit messages:  Apache messages tend to be descriptive~\cite{ApachePractice},
while FRLink's GitHub sample tends to contain exact matches, because copying
issue text is common practice~\cite{ruan2019deeplink}. Sun~\etal do not
investigate the effect of this differing practice on their results. They specify
their corpus in sufficient detail to reconstruct it. FRLink, the tool, however,
is not available. When we tried to reproduce FRLink, using its description in
Sun~\etal's paper, we were unable to reproduce the reported results.   
We contacted the authors for help explaining and correcting our reproduction
without response. More recently, Sun~\etal~\cite{PULink} treat existing links as
labels and reformulate the missing link problem into a semi-supervised problem.
As Bachmann~\etal found, existing links are biased; Sun~\etal do not discuss how
they coped with this bias. They report that their solution, PULink, outperforms
FRLink on FRLink's corpus. Like FRLink, PULink is not available.

Ruan~\etal~\cite{ruan2019deeplink} empirically studied the state of commit-issue
linking on GitHub Java projects and found only 42.2\% to be linked. They propose
DeepLink, a neural approach to the missing links problem. DeepLink trains a text
embedding for non-source artefacts and a code embedding for source artefacts
using the skip-gram model~\cite{mikolov2013efficient, mikolov2013distributed},
then passes each of these embeddings separately through an LSTM layer to obtain
the final vector representations. They use cosine similarity to compare the
vectors, choosing the maximum similarity to represent the score of the
commit-issue link. They show an improvement over FRLink in terms of F1-score,
and further show that pre-processing heuristics similar to previous work, such
as ReLink~\cite{relink} or MLink~\cite{MLink}, help DeepLink achieve a higher
F1-score. They also spot that the FRLink corpus had commit logs and issue titles
that are exact matches, introducing bias in the dataset which Ruan~\etal handle,
while FRLink does not. Since they did not evaluate DeepLink on Apache Commons or
against RCLinker and DeepLink is not available, we do not know its performance
relative to RCLinker.

Rath~\etal~\cite{1804.02433} also tackle the missing link problem, but from
within the requirement engineering community and without referencing the line of
work stemming from Bachmann~\etal The missing link inference work above uses a
vector space model over unigrams for textual features. They opt instead for a
n-gram model. They are the first to perform feature selection, using Weka's
feature auto-selection. They report promising results but on a different dataset
than Apache Commons.

Work focusing on the narrower traceability area of commit-issue linking has so
far been offline, focusing on fixing the situation after the fact, and working
at the commit level. The commit to issue mapping, however, is not
one-to-one~\cite{Kawrykow2011}. Indeed, current practice on GitHub suggests that
the one-to-one mapping be from pull-requests, that may contain a feature branch,
onto issues (\Cref{am:sec:eval:prelim}). Still, \Cref{am:sec:eval:prelim} also
shows that the state of pull-request-issue linking is non-ideal. To address
these issues, we focused Aide-mémoire as an online pull-request-issue linking
tool. Its online nature makes it complementary to offline approaches, while it
has a lower recall, it can still improve the training data of offline approaches
and allow for better link recovery. Its shift to pull-request focuses it on the
modern development practice that is now common on platforms such as GitHub.

\section{Towards Atomic Commits}
\label{chapter:literature:sec:flexeme_rel_work}

A potential source of inaccuracy and noise for the task of maintaining links
between commits and pull-requests is multi-concern commits. A developer, as part
of a bug fix or indeed to facilitate said bug fix, can and often do mix
refactoring changes with fixes. These are then committed as a single
patch. Such commits represent tangled commits. The ideal state of a version
history would be to contain only atomic commits, \ie those that tackle a single
task (would be linked to a single issue or implements a single stakeholder
concern). Changes unrelated to the fix may confuse statistical tools that strive
to maintain traceability links due to the added noise, thus approaches to (1)
detect multi-concern commits and (2) slice multi-concern commits into atomic
patches may aid reduce this source of noise. Further, this can have an added
benefit for developers since tools such as \lstinline+git bisect+ would also
benefit from such a segmentation of commits.

\subsection{Impact of Tangled Commits}
\label{chapter:literature:sec:flexeme_rel_work:impact}

Tao~\etal~\cite{Tao2012} were amongst the first to highlight the problem of
change decomposition in their study on programmer code comprehension; they
highlight the need for decomposition when many files are touched, multiple
features implemented, or multiple bug fixes committed. The last is diagnosed by
Murphy-Hill~\etal~\cite{Murphy-Hill2012} as a deliberate practice to improve
programmer productivity. Tao~\etal conclude that decomposition is required to
aid developer understanding of code changes.

Independently, Herzig~\etal~\cite{Herzig2013, Herzig2016} investigate the impact
of tangled commits on classification and regression tasks within software
engineering research. The authors manually classify a corpora of real-world
changesets as atomic, tangled or unknown, and find that the fraction of tangled
commits in a series of version histories ranges from 7\% to 20\%; they also find
that most projects contain a maximum of four tangled concerns per commit, which
is  consistent with previous findings by Kawrykow and
Robillard~\cite{Kawrykow2011}. They find that non-atomic commits
significantly impact the accuracy of classification and regression tasks such as
fault localisation.

In this thesis, we borrow the approach that Herzig~\etal~\cite{Herzig2016}
propose for generating a plausible dataset of tangled commits which we use to
compare previous approaches and our proposed approach to untangling commits
(\Cref{flex:sec:eval:corpus}).

\subsection{Untangling Commits into Atomic Patches}
\label{chapter:literature:sec:flexeme_rel_work:untangle}

Research on the impact of both tangled commits and non-essential code changes
prompted an investigation into changeset decomposition. Herzig
\etal~\cite{Herzig2013, Herzig2016} apply confidence voters in concert with
agglomerative clustering to decompose changesets with promising results,
achieving an accuracy of $0.58$-$0.80$ on an artificially constructed dataset
that mimics common causes of tangled commits. In contrast, Kirinuki
\etal~\cite{Kirinuki2014, Kirinuki2017} compile a database of atomic patterns to
aid the identification of tangled commits; they manually classify the resulting
decompositions as True, False, or Unclear, and find more than half of the
commits are correctly identified as tangled. The authors recognise that
employing a database introduces bias into the system and may necessitate
moderation via heuristics, such as ignoring changes that are too fine-grained or
add dependencies.

Other approaches rely on dependency graphs and use-define chains: Roover
\etal~\cite{Roover2017} use a slicing approach to segment commits across a
Program Dependency Graph, and correctly classify commits as (un)tangled in over
90\% of the cases for the systems studied, excluding some projects where they
are hampered by toolchain limitations. They propose, but do not implement, the
use of System Dependency Graphs to reduce some of the limitations of their
approach, such as being solely intraprocedural.

Barnett~\etal~\cite{Barnett2015} implement and evaluate a commit-untangling
prototype. This prototype projects commits onto def-use chains, clusters the
results, then classifies the clusters as trivial or non-trivial. A cluster is
trivial if its def-use chains all fall into the same method. Barnett~\etal
employ a mixed approach to evaluate their prototype. They manually investigated
results with few non-trivial clusters (0-1), finding that their approach
correctly separated 4 of 6 non-atomic commits,  or many non-trivial clusters (>
5), finding that, in all cases, their prototype's sole reliance on def-use
chains lead to excessive clustering. For results containing 2--5 clusters, they
conducted a user-study. They found that 16 out of the 20 developers surveyed
agreed that the presented clusters were correct and complete. This result is
strong evidence that their lightweight and elegant approach is useful,
especially to the tangled commits that Microsoft developers encounter
day-to-day. During the interviews, multiple developers agreed that the changeset
analysed did indeed tangle two different tasks, sometimes even confirming that
developers had themselves separated the commit in question after review. In
addition to validating their prototype, their interviews also found evidence for
the need for commit decomposition tools. Because they use def-use chains and
ignore trivial clusters, Barnett~\etal's approach can miss tangled concerns that
are in the same method. Barnett~\etal's user study itself shows that this can
matter: it reports that some developers disagreed with the classification of
some changesets as trivial.

Dias~\etal~\cite{Dias2015} take a more developer-centric approach and propose
the EpiceaUntangler tool. They instrument the Eclipse IDE and use confidence
voters over fine-grained IDE events that are later converted into a similarity
score via a Random Forest Regressor. This score is used similarly to Herzig
\etal~\cite{Herzig2016}'s metrics, \ie to perform agglomerative clustering. They
take an instrumentation-based approach to harvest information that would
otherwise be lost, such as changes that override earlier ones. This approach
also avoids relying on static analysis. They report a high median success rate
of 91\% when used by developers during a two-week study. While Dias~\etal
sidestep static analysis, they require developers to use an instrumented IDE.

Previous literature proposed approaches were either simple and, by virtue of the
simplicity, efficient~\cite{Barnett2015}, made use of heuristics to construct
their similarities and decompose-recluster commits~\cite{Herzig2016}, or relied
on heavy IDE instrumentation~\cite{Dias2015}. In \Cref{chapter:flexeme}, we
propose an approach that does not rely on heuristics, instead uses similarities
over a multi-version representation of code in a decompose-recluster method
similar to Herzig \etal While slower than Barnett \etal, it is within 10
seconds (\Cref{flex:sec:impl:deploy}), hence compatible with CI/CD, while being
more accurate (13\% accuracy vs 81\%). Further, Flexeme does not require project
instrumentation and its static analysis can be done incrementally as a
post-commit script.

\subsection{Multiversion Representations of Code}
\label{chapter:literature:sec:flexeme_rel_work:multiversion_repr}

Related work has considered multiversion representations of programs for static
analysis. \citet{kim2006program} investigate the applicability of different
techniques for matching elements between different versions of a program. They
examine different program representations, such as String, AST, CFG, Binary or a
combination of these as well as the tools that work on them on two hypothetical
scenarios. They only consider the ability of the tools to match elements across
versions and leave the compact representation of a multiversion structure as
future work. Some of the conclusions from the matching challenges presented by
\citet{kim2006program} are echoed in Flexeme as well, we make use of the UNIX
diff as it is stored within version histories; however, we also make use of
line-span hints from the compilers for each version of the application to better
facilitate matching nodes within a NFG.

Le~\etal~\cite{Le2014} propose a Multiversion Interprocedural Control Graph
(MVICFG) for efficient and scalable multiversion patch verification over systems
such as the PuTTY SSH client.
Alexandru~\etal~\citep{alexandru2019redundancy} generalise the Le~\etal MVICFG
construction to arbitrary software artefacts by constructing a framework that
creates a multiversion representation of concrete syntax trees for a git
project. They adopt a generic ANTLR parser, allowing them to be language
agnostic, and achieve scalability by state sharing and storing the
multi-revision graph structure in a sparse data structure. They show the
usefulness of such a framework by means of `McCabe’s Complexity', which they
implement in this framework such that it is language agnostic, does not repeat
computations unnecessarily and reuses the data stores in the sparse graph by
propagating from child to parent node. \citet{Sebastian2018} propose a compact,
multiversion AST that cleverly shares state across versions.

Previous work mostly considered multiversion representation of code for static
analysis or the compute of software metrics. We instead focus on multiversion
representations of code as a initial structure from which we can segment atomic
commits. To that end, we propose two new multiversion graph representations of
code, firstly, the \deltaPDG, which generalises Le~\etal's MVICFG to PDGs, and
the \deltaPDGN that further augments \deltaPDGNs with nameflow information
(\Cref{flex:sec:pdgn}), which we hoped to enable us to better approximate the
location of concerns in code via naming conventions.

\subsection{Semantic Slicing of Version Histories}
\label{chapter:literature:sec:flexeme_rel_work:sem_slice_vh}

Features in a system often co-evolve, which tangles the changes made for a one
high-level feature with others in a version history. To resurface
feature-specific changes, one can dynamically slice a target version, then walk
backwards in history while they can reverse the intra-version patch without
conflict; at each version they reach, they add any commit that contains a hunk
that touches the current slice to it. The goal of this semantic slicing of
version histories is to find a minimal slice of a version history that captures
the evolution of a feature. Li~\etal~\cite{Li2018, li2015semantic} first
formulated and introduced this problem. Semantic slicing is a form of commit
untangling backwards through history.  This retrospective framing is why they
treat the history as immutable. In this initial solution, Li~\etal treat commits
as atomic so their slices may contain noise introduced by tangled commits. To
reduce this noise, Li~\etal, in more recent work~\cite{Li2019}, unpack commits
into single-file commits into a private, local history.

Flexeme, in contrast, is static and online:  built from the ground up to rewrite
commits as developer make history.  
As such Flexeme and semantic slicing are complementary:  Flexeme would improve
the signal to noise ratio of semantic slicing. An interesting direction for
future work would be to use Flexeme to preprocess version histories prior to
semantically slicing them as with Definer~\cite{Li2019}.

\section{Discerning Text from Code}
\label{chapter:literature:sec:posit_rel_work}

One of the meta-task that can aid the traceability aspect of this research is
the segmentation of English, or natural languages in a more general sense, from
formal languages. While online fora perform a best effort attempt at maintaining
separate formatting for the two modalities of text, this signal remains
noisy~\cite{ponzanelli2014improving}. The work carried out in this thesis seeks
to improve the situation by framing the task as a code-switching phenomenon and
borrowing results from the relevant areas of the Natural Language Processing
(NLP) community (\Cref{chapter:posit}).

In this section, we will focus on first providing how similar techniques were
already employed within the Software Engineering (SE) community, followed by the
more relevant for this task research from the NLP community, concluding with
Ponzanelli~\etal's work on \SO which is closest to POSIT.

\subsection{Part of Speech Tagging in Software Engineering}
\label{chapter:literature:sec:posit_rel_work:pos4se}

In SE research, part-of-speech tagging has been directly applied for identifier
naming~\cite{Binkley2011}, code summarisation~\cite{Haiduc2010a, Haiduc2010b},
concept localisation~\cite{Abebe2010}, traceability-link
recovery~\cite{Capobianco2013}, and bug fixing~\cite{Tian2015}. The main
intuition behind the statistical approaches in this are are due to the
naturalness of code as discussed by Hindle~\etal~\cite{hindle2012naturalness}.

Operating directly on source code (not mixed text), Newman
\etal~\cite{Newman2017} created source code equivalents for lexeme categories
from natural languages. They looked at the behaviour exhibited by Proper Nouns,
Nouns, Pronouns, Adjectives, and Verbs and derived similar notions for
source-code from 1) Abstract syntax trees, 2) how the tokens impact memory, 3)
where they are declared, and 4) what type they have. They report the prevalence
of these categories in source-code. Their goal was to map these code categories
to PoS tags, thereby building a bridge for applying NLP techniques to code for
tasks such as program comprehension. 

Treude~\etal~\cite{Treude2015portuguese} described the challenges of analysing
software documentation written in Portuguese which commonly mixes two natural
languages (Portuguese and English) as well as code. They suggested the
introduction of a new part-of-speech tag called Lexical Item to capture cases
where the ``correct'' tag cannot be determined easily due to language switching.

Instead of adapting code to part-of-speech akin to Newman
\etal~\cite{Newman2017}, in this thesis, we instead wish to treat the problem as
a multi-lingual parsing problem, \ie we want to know what part of speech or part
of code each token is for all languages that we may consider. In POSIT, we
realised a first step towards this by considering the single natural language
and a single C-like source code case. Indeed, it is still an open problem to
tackle situations such as those presented by Treude
\etal~\cite{Treude2015portuguese}.

\subsection{Part of Speech Tagging in Code-Switched Natural Languages}
\label{chapter:literature:sec:posit_rel_work:nlcs}

NLP researchers are growing more interested in code-switching text and speech,
\ie that mixes multiple natural languages, among other reasons due to the higher
availability of corpora. These were made available either by dedicated
collections efforts, such as Miami Bangor~\cite{bangorTalk}, or due to social
media sites, such as twitter, that cause the mixing of English with other
languages~\cite{Vyas2014}. Previously, such data was scarce because
code-switching was stigmatised~\cite{Poplack1980}. 

Focusing specifically on part-of-speech tagging, Solorio and
Liu~\cite{Solorio2008} presented the first statistical approach to the task of
part-of-speech (PoS) tagging code-switching text. On a Spanglish corpus, one
that mixes Spanish and English, they heuristically combine PoS taggers trained
on larger monolingual corpora and obtain 85\% accuracy. We can think of this
approach as a mixtures of experts model. Jamatia~\etal~\cite{Jamatia2015},
working on an English-Hindi corpus gathered from Facebook and Twitter, recreated
Solorio's and Liu's tagger and additionally proposed a tagger built on
Conditional Random Fields. The mixtures of experts model performed better at
72\% vs 71.6\%. In 2018, Soto and Hirschberg~\cite{Soto2018} proposed a neural
network approach, opting to solve two related problems simultaneously:
part-of-speech tagging and Language ID tagging. The second task can be thought
of as a segmentation task, which in our English-code setting can be rendered
using formatting for code-blocks. They combined a biLSTM with a CRF network at
both outputs and fused the two learning targets by simply summing the respective
losses. This network achieves a test accuracy of 90.25\% on the inter-sentential
code-switched dataset from Miami Bangor~\cite{bangorTalk}.

The languages that previous literature considered for code-switching tasks may
have different tag distributions or even tag sets; although not as disjoint as
in our case with POSIT. This provided an initial starting point for promising
architectures. \Cref{posit:sec:impl} details how we combined these approaches to
enables us to solve a version of our joint tagging and segmentation task.

\subsection{Mixed-Text: When Natural Meets Formal}
\label{chapter:literature:sec:posit_rel_work:mixed}

In the context of natural languages mixed with source code and other formal
languages (mixed-text), Ponzanelli~\etal are the first to go beyond using
regular expressions to parse such mixed text. When customising
LexRank~\cite{Ponzanelli2015b}, a summarisation tool for mixed text, they
employed an island grammar that parses Java and stack-trace islands embedded in
natural language, which is relegated to water. They followed up LexRank with
\stormed, a tool that uses an island grammar to parse Java, JSON, and XML
islands in mixed text \SO posts, again relegating natural language to
water~\cite{Ponzanelli2015a}.  \stormed produces heterogeneous abstract syntax
trees (AST), which are ASTs decorated with natural language snippets. They make
both the tool and the resulting corpus available. \stormed, however, remains
reliant on either code-block annotations or heuristics to discern code from
English. POSIT instead learns this segmentation from data, and, after a manual
evaluation, we indeed observed it to be capable of segmenting \SO posts where
submitters forgotten to include the appropriate formatting
(\Cref{posit:sec:application}).
\chapter{Aide-mémoire: Improving a Project’s Collective Memory via Pull Request--Issue Links}
\label{chapter:am}

\paragraph{Paper Authors}\\
Profir-Petru~P\^ar\c{t}achi, Department of Computer Science, University College London, United Kingdom\\
David~R.~White, University of Sheffield, United Kingdom\\
Earl~T.~Barr, Department of Computer Science, University College London, United Kingdom

\subimport{../am-tex/}{../am-tex/paper_macros}

\paragraph{Abstract}
\subimport{../am-tex/}{../am-tex/abstract}

\subimport{../am-tex/}{../am-tex/introduction}
\subimport{../am-tex/}{../am-tex/example}
\subimport{../am-tex/}{../am-tex/implementation}
\subimport{../am-tex/}{../am-tex/feature_space}
\subimport{../am-tex/}{../am-tex/experiment}
\subimport{../am-tex/}{../am-tex/evaluation}
\subimport{../am-tex/}{../am-tex/related_work}
\subimport{../am-tex/}{../am-tex/conclusion}

\chapter{Flexeme: Untangling Commits Using Lexical Flows}
\label{chapter:flexeme}

% Copied here to avoid issues with the paper latex
\newcommand{\deltaPDG}{$\delta$-PDG\xspace}
\newcommand{\deltaPDGnCV}{$\delta$-PDG+CV\xspace}
\newcommand{\deltaPDGs}{$\delta$-PDGs\xspace}
\newcommand{\deltaPDGN}{$\delta$-NFG\xspace}
\newcommand{\deltaPDGNs}{$\delta$-NFGs\xspace}
\renewcommand{\algorithmicrequire}{\textbf{Input:}}
\renewcommand{\algorithmicensure}{\textbf{Output:}}


\newcommand{\ourapproach}{\textsc{flexeme}\xspace}
\newcommand{\Ourapproach}{\textsc{Flexeme}\xspace}
\renewcommand{\ourtool}{\textsc{heddle}\xspace}
\renewcommand{\Ourtool}{\textsc{Heddle}\xspace}
\newcommand{\projURL}{\url{https://github.com/PPPI/Flexeme}\xspace}

% I've moved our results here so they can be easily updated consistently...
\newcommand{\Ouraccuracy}{$0.81$\xspace}
\newcommand{\Ouraccuracyhigh}{$0.84$\xspace}
\newcommand{\OuraccuracyhighProject}{Nancy\xspace}
\newcommand{\Ourspeedup}{$32$\xspace}
\newcommand{\OurslowdownBarnett}{$9$\xspace}
\newcommand{\Ourabsolutespeedup}{$3'10''$\xspace}
\newcommand{\Ourruntime}{ten seconds\xspace}
\newcommand{\OurtoolsAtomicAccuracy}{$0.63$\xspace}
\newcommand{\OurtoolsAtomicLabelAccuracy}{$0.93$\xspace}

% Comparisons or other tools
\newcommand{\OurtoolVsHerzigAccuracy}{$0.14$\xspace}
\newcommand{\BarnettAccuracy}{$0.13$\xspace}
\newcommand{\HerzigConceptAccuracyDrop}{$0.07$\xspace}

% Scalability Results
\newcommand{\OurtoolsScalabilityRegression}{$t = 0.3371 - 0.0041 n + 0.0015 n^2$, $R^2=1.00$\xspace}
\newcommand{\HerzigScalabilityRegression}{$t = 0.8794 - 0.0528 n + 0.0019 n^2$, $R^2=0.99$\xspace}
\newcommand{\SpeedGainAtScale}{$68$ seconds\xspace}

% p-values
% Cross-tool
\newcommand{\HerzigDeltaOurtoolAccuracy}{$p<0.001$\xspace}
\newcommand{\HerzigDeltaOurtoolTime}{$p=0.51$\xspace}
% Cross-concepts
\newcommand{\BarnettConceptsAccuracy}{$p < 0.01$\xspace}
\newcommand{\HerzigConceptsAccuracy}{$p=0.28$\xspace}
\newcommand{\OurtoolConceptsAccuracy}{$p=0.76$\xspace}

\newcommand{\annequiv}{$\equiv_{a}$\xspace}
\newcommand{\nodeequiv}{$\equiv_{n}$\xspace}

\newcommand{\bra}{\langle}
\newcommand{\ket}{\rangle}

% ------------------------------------------------------- [ Colour Definitions ]
\definecolor{cb-green-sea}  {RGB}{  0, 146, 146}
\definecolor{cb-purple}     {RGB}{ 73,   0, 146}
\definecolor{cb-blue-sky}   {RGB}{109, 182, 255}
\definecolor{cb-burgundy}   {RGB}{146,   0,   0}

% Define diff language for lstlisting
\lstdefinelanguage{diff}{
  morecomment=[f][\color{cb-blue-sky}]{@@}, % group identifier
  morecomment=[f][\color{cb-burgundy}]-,    % deleted lines 
  morecomment=[f][\color{cb-green-sea}]+,   % added lines
  morecomment=[f][\color{cb-purple}]{---},  % Diff header lines (must appear after +,-)
  morecomment=[f][\color{magenta}]{+++},
}

\paragraph{Paper Authors}%
\begin{enumerate}
    \item[] Profir-Petru~P\^ar\c{t}achi, Department of Computer Science, University College London, United Kingdom
    \item[] Santanu~Kumar~Dash, University of Surrey, United Kingdom
    \item[] Miltiadis~Allamanis, Microsoft Research, Cambridge, United Kingdom
    \item[] Earl~T.~Barr, Department of Computer Science, University College London, United Kingdom
\end{enumerate}

\paragraph{Abstract}
\subimport{../untangle-tex/}{../untangle-tex/abstract}

\subimport{../untangle-tex/}{../untangle-tex/intro}
\subimport{../untangle-tex/}{../untangle-tex/example}
\subimport{../untangle-tex/}{../untangle-tex/definitions}
\subimport{../untangle-tex/}{../untangle-tex/implementation}
\subimport{../untangle-tex/}{../untangle-tex/evaluation}
\subimport{../untangle-tex/}{../untangle-tex/results}
\subimport{../untangle-tex/}{../untangle-tex/threats}
\subimport{../untangle-tex/}{../untangle-tex/rel_work}
\subimport{../untangle-tex/}{../untangle-tex/conclusion}
\chapter{POSIT: Simultaneously Tagging Natural and Programming Languages}
\label{chapter:posit}

%%%--------------------------------------------------------------------------%%%
%%% Macros from POSIT                                                        %%%
%%%--------------------------------------------------------------------------%%%

\renewcommand{\ourtool}{\textsc{POSIT}\xspace}
\renewcommand{\Ourtool}{\textsc{POSIT}\xspace}
\newcommand{\projurl}{\url{https://github.com/PPPI/POSIT}\xspace}

%%%--------------------------------------------------------------------------%%%
%%% END OF Macros from POSIT                                                 %%%
%%%--------------------------------------------------------------------------%%%

\paragraph{Paper Authors}
\begin{enumerate}
    \item[] Profir-Petru~P\^ar\cb{t}achi, Department of Computer Science, University College London, United Kingdom
    \item[] Santanu~Kumar~Dash, University of Surrey, United Kingdom
    \item[] Christoph~Treude, University of Adelaide, South Australia, Australia
    \item[] Earl~T.~Barr, Department of Computer Science, University College London, United Kingdom
\end{enumerate}

\paragraph{Abstract ---}
\subimport{../code_pos_tex/}{../code_pos_tex/abstract}

\subimport{../code_pos_tex/}{../code_pos_tex/intro}
\subimport{../code_pos_tex/}{../code_pos_tex/motivating_example}
\subimport{../code_pos_tex/}{../code_pos_tex/problem_statement}
\subimport{../code_pos_tex/}{../code_pos_tex/implementation}
\subimport{../code_pos_tex/}{../code_pos_tex/evaluation}
\subimport{../code_pos_tex/}{../code_pos_tex/application}
\subimport{../code_pos_tex/}{../code_pos_tex/discussion}
\subimport{../code_pos_tex/}{../code_pos_tex/rel_work}
\subimport{../code_pos_tex/}{../code_pos_tex/conclusion}
\chapter{Conclusions}
\label{chapter:conclusions}

\todo{We conclude on how the three papers advance the state of the art. How further advances could be made and future directions for project health.}

\section{Summary of Contributions}

\todo{summarise key takeaways of the three papers, sum them up under a single umbrella}

We proposed the first online PR-Issue linker, advanced the state of the art for commit untangling, and open the door to mixed-text analysis.

\section{Summary of Future Work}

\todo{more fluff}

\textbf{Mixed-text comprehension}
POSIT is a step along the path to mixed-text comprehension. It offers the first building blocks, an equivalent for Part of Speech tags for mixed-text. More sophisticated models can build upon this to aid comprehension and parsing of mixed text such that meaning is carried over from the Natural Language channel to the algorithmic channel.

\textbf{Graph Neural Network Approaches for Graph Similarity}
In Flexeme, we make use of graph kernels for graph Similarity computation; however recent research in Graph Neural Networks show promising results, and future work may focus on replacing our graph kernels for a neural network implementation if a suitable training corpus is present.

\textbf{Multi-version PDG Static Analysis}
In Flexeme, we propose a novel structure, the \deltaPDGN. While we demonstrate its use for commit untangling, this same structure could be employed for multi-version static analysis when the analysis depends on Program Dependency Trees. Alternatively, one could consider version sensitive slicing.

\hypersetup{linkcolor = {PineGreen}}
% Actually generates your bibliography. The fact that \include is 
% the last thing before this ensures that it is on a clear page.
\bibliography{full_bib}

% All done. \o/
\end{document}
