The thesis “Improving Project Health using Machine Learning” by Pârțachi
Profir-Petru, the undersigned, is composed of three main papers which were
written in collaboration with others. For each paper, I include the list of
authors herein and detail exactly my contributions. All authors have contributed
significantly to the writing of the papers; therefore, I detail only non-writing
work. Additionally, all work was done under the careful supervision of Earl
Barr, his insight has shaped all my work and our regular meetings were often
used to checkpoint and plan out my research tasks.

\begin{itemize}[leftmargin=*]
    \item[] \textbf{Aide-mémoire: Improving a Project’s Collective Memory via
Pull Request-Issue Links} 
    
    \noindent\emph{Authors: Profir-Petru Pârțachi, David R. White, Earl T. Barr}
    
    \noindent\emph{Venue: Submitted to ACM Transactions on Software Engineering
    and Methodology (TOSEM)}

    \noindent I have determined and mined the dataset of GitHub projects. I have
    implemented the proposed PR-Issue linker, performed the exploratory data
    analysis and feature engineering as well as wrote the evaluation and result
    analysis scripts. The experimental and EDA designs were done in close
    collaboration with David White who guided me patiently.

    \item[] \noindent\textbf{Flexeme: Untangling Commits Using Lexical Flows}
    
    \noindent\emph{Authors: Profir-Petru Pârțachi, Santanu Kumar Dash, Miltose
    Allamanis, Earl T. Barr}
    
    \noindent\emph{Venue: Proceedings of 28th ACM Joint European Software
    Engineering Conference and Symposium on the Foundations of Software
    Engineering, (ESEC/FSE 2020)}

    \noindent  I have implemented in full the construction of our new structure,
    though the idea of the structure was from Earl Barr, and the details of the
    specification were worked in close collaboration. I also implemented the
    necessary methods to construct the dataset, reproduced previous work in the
    area and implemented our proposed untangling algorithms as well as their
    evaluation on the constructed dataset. The RefiNym implementation was
    provided by Santanu Dash who helped me integrate it with Flexeme. The
    original PDG extraction implementation for C\# code was provided by Miltose
    Allamanis. Santanu Dash and I performed the manual evaluations.
    
    %%% !Remember to check if this is still needed if you edit text above it!
    \pagebreak
    
    \item[] \noindent\textbf{POSIT: Simultaneously Tagging Natural and
    Programming Languages} 
    
    \noindent\emph{Authors: Profir-Petru Pârțachi, Santanu Kumar Dash, Christoph
    Treude, Earl T. Barr}
    
    \noindent\emph{Venue: Proceedings of 42nd International Conference on
    Software Engineering (ICSE ’20)}

    \noindent I have implemented the preprocessing scripts, the Neural Network
    that realises POSIT, the adaptation of the previous State-of-the-art to our
    problem, and the necessary evaluation scripts. The Code Comments corpus was
    provided by Santanu Dash. The original implementation of TaskNav was
    provided by Christoph Treude. The informal proof of context-sensitivity of
    mixed-text was worked on in close collaboration with Earl Barr without whom
    the proof would have not been finished in a timely manner. Manual evaluation
    of POSIT was done together with Santanu Dash, while manual evaluations of
    TaskNav augmented with POSIT were done together with Christoph Treude.

\end{itemize}

\doublesignature{Earl T. Barr}