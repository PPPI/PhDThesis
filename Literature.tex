\chapter{Literature Review}
\label{chapter:literature}

\todo{Chapter Intro}
In this chapter I first present the wider literature that concerns developer
activities impacting project health in \Cref{todo}. I then focus on the
subproblems tackled in the thesis and start by presenting Traceability and
Commit-Issue linking literature in \Cref{chapter:literature:sec:am_rel_work}.
\Cref{chapter:literature:sec:flexeme_rel_work} presents the literature around
commit untangling. I conclude this chapter by describing the literature around
mixed-text in \Cref{chapter:literature:sec:posit_rel_work}.

\section{Bridging the Gap between Issues and History}
%TODO: Importance of metadata for our framework/system
\label{chapter:literature:sec:am_rel_work}

In this section, we focus on software traceability as an area, the position of
the commits to issues traceability task within this wider context, and our
proposed contribution within this.

\subsection{Modern Development}
\label{chapter:literature:sec:am_rel_work:modern_dev}

Modern development increasingly relies on tooling that integrates Version
Control, Issue Tracking, Wikis, Continuous Integration and Continuous Deployment
under a single system. Notable examples are GitHub and Atlassian's JIRA. This
new development paradigm poses new problems and opportunities. Kalliamvakou
\etal~\cite{Kalliamvakou2014} elucidates these, using GitHub as their archetypal
example. A particular opportunity Kalliamvakou~\etal identify is this paradigm's
integration of Version Control and Issue Tracking. Aide-mémoire rests on this
integration. Aide-mémoire also takes a step toward realising a promise they
identify:  interlinking developers, pull requests, issues and commits to offer a
comprehensive view of the software development process. 

Lack of PR-issue links is an ongoing problem in modern software development.  So
much so, Agile practice specifies spring cleaning an issue (a user story in
Agile terminology) backlog. During backlog refinement, developers remove stale
stories and reprioritise and re-estimate remaining stories. When all stories are
stale, this practice discards all sprint-related artefacts --- issues, feature
requests, user stories, as well their links --- in favour of starting the next
sprint from a clean slate~\cite{BacklogRefinement}. Projects must resort to this
spring cleaning all too often~\cite{laurieTrattPersonalCommunciation}. Clearly,
this practice loses valuable information. The loss of documentation it entails
it just one example. Adoption of PR-issue inference tooling, like Aide-mémoire,
promises to reduce the need to resort to this drastic measure by enabling
automatic issue triaging.

\subsection{Software Traceability}
\label{chapter:literature:sec:am_rel_work:tracability}

Requirements engineering (RE) focuses on stakeholders, decision makers, and
their artefacts: requirements, documentation, specification, and design or
architectural documents. These artefacts tend to be natural language, text or
speech, and often go unrecorded. When they are recorded, they exist in multiple
formats, including spreadsheets, email, figures, and printed material. They
further encompass developer artefacts, such as source code, pull-requests,
commits, and issues, but do not focus on them. Further, within an agile context,
documentation is often forgone in favour of staying lean and traveling light.
This further complicates an already difficult task and suggests that tooling
that would target such a paradigm should be light-weight~\cite{Stahl2017}.

\emph{Software traceability} seeks to infer \emph{traces} (\ie links) between
these heterogeneous artefacts~\cite{Cleland-Huang2014}. Missing or hard-to-parse
artefacts greatly complicate trace recovery, which is why much work on
traceability seeks to provide tooling to decision makers to capture or parse
these artefacts and persuade them to use it~\cite{Neumuller06, jiralinkdoc,
ghlinkdoc}. These tools must often record decision maker or developer
interactions, with each other or their tools. They must also avoid being either
disruptive, requiring the developer to switch contexts, or invasive, as when
they require developers to change their workflow or use instrumented
IDEs~\cite{TopicTraceability}, raising privacy and deployability concerns.
Inferring traces over these heterogeneous artefacts, as a consequence of their
heterogeneity, can only leverage abstract, generic features. As many of the
artefacts within such systems are textual, work in this area borrowed techniques
from Information Retrieval, including ourselves. An excellent survey detailing
the use of such methods within traceability has been done by Borg
\etal~\cite{Borg2014}. Further, Mills
\etal~\cite{Mills:2017:ATL:3106237.3121280} propose to reduce the human effort
required by IR based techniques by adding a classification step after the
recommender that filters the suggestions. Finally, St{\aa}hl
\etal~\cite{Stahl2017} propose a light-weight framework where different systems
can emit traceability events, facilitating a standardisation of how such links
are recorded by minimising the cost of integrating otherwise incompatible tools
within a single eco-system, which they call the eiffel framework.

Asuncion and Taylor~\cite{Asuncion:2009:CCL:1556908.1557008} use record-replay
and hypermedia to tackle software traceability, a prospective problem. Their
online solution cannot employ offline information retrieval techniques (such as
VSM, tf-idf or topic modelling) commonly used in retrospective approaches to
traceability~\cite{4249808}. They focus on requirements and specifications,
architectural modules, source files, and test cases, making no mention of PRs or
issues beyond bug reports. Previous work required expertise in formal
modelling~\cite{Pohl96, Pinheiro96}. To eliminate this barrier to entry, they
instrument developer or decision maker tools, raising the usual deployability
and privacy concerns. Indeed, the solution they present invasively instrumented
specific versions of no less than five different tools, including general
purpose tools like Firefox and MS Word. Many of these tools are under constant
development; updating this instrumentation out-of-tree is extremely expensive,
suggesting that St{\aa}hl \etal's light-weight approach is more likely to be
maintainable.

Asuncion \etal~\cite{TopicTraceability} in follow-up work incorporate an LDA
model into their framework to improve the interpretability of their traces. To
remain online, they recompute an LDA model on demand for each graph
visualisation and search query. Aide-mémoire does not rely on LDA and thus
scales better because it does not require per query model retraining.

Additionally, Falessi \etal~\cite{Falessi2017} have done work towards
quantifying the number of links that are left to be recovered. Their work
assumes an offline, closed-world setting; they provide a framework that can help
an analyst decide when to stop pursuing a traceability maintenance task. The aim
of this work is to provide a quantifiable signal of when the traceability task
should be finished as linking can be considered good enough.

\subsection{Commit-Issue Link Inference}
\label{chapter:literature:sec:am_rel_work:cli}

The \emph{missing link problem} is the offline prediction problem of inferring
missing commit-issue links given a version history and issue tracker archive.
Bachman~\etal were the first to formulate and quantify this
problem~\cite{Bird2009,MissingLinks}. Their work aids developers indirectly by
helping researchers and tool-smiths avoid the bias introduced by missing links
that could undermine their techniques or tools. Specifically, they show that by
assuming recorded links to be representative of all links, tools are biased to
use code from more experienced developers, thus not learning from mistakes or
bugs introduced by less experienced contributors. 

Bachman~\etal together with Apache Commons developers manually supplied missing
links and published their Apache Commons corpus. Wo~\etal are the first to
attempt to automatically infer them. They propose ReLink~\cite{relink}, which
measures the similarity of change logs and bug reports with cosine similarity on
tf-idf vectors, learning a threshold for true links. Nguyen~\etal exploited
commit and issue tracker metadata in MLink~\cite{MLink} to improve recall over
ReLink. They evaluated MLink on the Apache Commons corpus, making it the de
facto standard. ReLink and MLink first consider only commit data to form an
initial set of commit-issue links that they then filter. Prechlet and
Pepper~\cite{prechelt2014bflinks} dispense with the initial commit-only stage,
and instead consider both commit and issue data from the start. They argue this
bi-directional inference is more sound. Their BFLinks proposes two link
predictors (based on bug and commit IDs) and a series of filters to reduce the
candidate set of links.

Prior to Aide-mémoire, Li~\etal was the state of the art.
RCLinker~\cite{RCLinker} further improves recall. \Cref{am:sec:reproduction}
details RCLinker's realisation. RCLinker relies on
ChangeScribe~\cite{ChangeScribe} to produce textual descriptions of commits,
especially those that lack commit messages. PRs have their own message and
aggregate multiple commits and their messages.  
This fact alleviates the problem of sparse commit descriptions for Aide-mémoire.
Both tools would benefit from better PR summarisation and description. Liu
\etal~\cite{liu2019automatic} propose a tool, based on a bi-directional RNN with
a copy network, to tackle this task. 

Sun~\etal~\cite{FRLink} used non-source files in commits for commit-issue link
inference. They argue that these files are important for capturing developer
intent. They use the standard heuristics, such as checking for camelCase or
snake\_case, to determine the relevancy of a non-source file in a commit. As is
conventional, they implement these heuristics as regexes. They use the resulting
set of non-source files together with the co-committed source files to compute
textual features. They scan a preset and fixed list of the similarity thresholds
to find the maximum F1-score where Recall is at least 0.80. The procedure raises
two unanswered questions. First, how did the authors determine the threshold
granularity? Second, how does training FRLink on F1-score for a task whose
performance is measured in terms of F1-score avoid overfitting? They report
these choices allow FRLink to improve Recall over previous work while matching
or improving F1-score. 

Sun~\etal evaluate FRLink on a new corpus of GitHub projects. This corpus
differs from the Apache corpus used in prior work in the conventions governing
commit messages:  Apache messages tend to be descriptive~\cite{ApachePractice},
while FRLink's GitHub sample tends to contain exact matches, because copying
issue text is common practice~\cite{ruan2019deeplink}. Sun~\etal do not
investigate the effect of this differing practice on their results. They specify
their corpus in sufficient detail to reconstruct it. FRLink, the tool, however,
is not available. When we tried to reproduce FRLink, using its description in
Sun~\etal's paper, we were unable to reproduce the reported results.   
We contacted the authors for help explaining and correcting our reproduction
without response. More recently, Sun~\etal~\cite{PULink} treat existing links as
labels and reformulate the missing link problem into a semi-supervised problem.
As Bachmann~\etal found, existing links are biased; Sun~\etal do not discuss how
they coped with this bias. They report that their solution, PULink, outperforms
FRLink on FRLink's corpus. Like FRLink, PULink is not available.

Ruan~\etal~\cite{ruan2019deeplink} empirically studied the state of commit-issue
linking on GitHub Java projects and found only 42.2\% to be linked. They propose
DeepLink, a neural approach to the missing links problem. DeepLink trains a text
embedding for non-source artefacts and a code embedding for source artefacts
using the skip-gram model~\cite{mikolov2013efficient, mikolov2013distributed},
then passes each of these embeddings separately through an LSTM layer to obtain
the final vector representations. They use cosine similarity to compare the
vectors, choosing the maximum similarity to represent the score of the
commit-issue link. They show an improvement over FRLink in terms of F1-score,
and further show that pre-processing heuristics similar to previous work, such
as ReLink~\cite{relink} or MLink~\cite{MLink}, help DeepLink achieve a higher
F1-score. They also spot that the FRLink corpus had commit logs and issue titles
that are exact matches, introducing bias in the dataset which Ruan \etal handle,
while FRLink does not. Since they did not evaluate DeepLink on Apache Commons or
against RCLinker and DeepLink is not available, we do not know its performance
relative to RCLinker.

Rath~\etal~\cite{1804.02433} also tackle the missing link problem, but from
within the requirement engineering community and without referencing the line of
work stemming from Bachmann~\etal. The missing link inference work above uses a
vector space model over unigrams for textual features. They opt instead for a
n-gram model. They are the first to perform feature selection, using Weka's
feature auto-selection. They report promising results but on a different dataset
than Apache Commons. As previously noted (\Cref{am:sec:reproduction}),
Rath~\etal's work is difficult to compare with Aide-mémoire, because it uses
temporal features, such as a predicate that is true when a commit falls between
issue's creation and its resolution. In Aide-mémoire's online setting, this
predicate will sometimes be defined in terms of an event that is in the future
relative to the present query.

In contrast to previous work, Aide-mémoire exclusively focuses on the PR-issue
link inference problem. Good PR-issue linking accelerates development: PR-issue
links allow developers to more quickly understand why a pull-request was
submitted or how an issue was resolved in code; they also permit the use of
productivity enhancing techniques like automatic bug
localisation~\cite{BugLocBasedOnHistory, SoftChangeToBugLoc}, or automatic patch
generation tools such as R2Fix~\cite{R2Fix}. Version control and issue tracking
are almost ubiquitous in modern software development. PRs and issues are
plentiful, well-suited for machine learning which is data hungry, and their
format is well-known. Thus, unlike more general traceability tools, Aide-mémoire
does not have to contend with missing artefacts or a profusion of formats.
Solutions to this problem can exploit the structure encoded in PR meta-data. We
do not rely on change summarisers to produce a natural language description of
the source changes; we exploit the PR description instead. We designed
Aide-mémoire to seamlessly integrate into modern development practice. We
suggest links when a developer closes an issue or submits a PR, when this
information is pertinent, without intrusive instrumentation of developer tools
and the attendant privacy and deployability concerns. Further, the link
recording event can easily be emitted to the Eiffel framework by queueing an
appropriately formatted JSON.

\section{Towards Atomic Commits}
\label{chapter:literature:sec:flexeme_rel_work}

A potential source of inaccuracy and noise for the task of maintaining links
between commits and pull-requests is multi-concern commits. A developer, as part
of a bug fix or indeed to facilitate said bug fix, can and indeed do mix
refactoring changes together with fixes. These are then committed as a single
patch. Changes unrelated to the fix may confuse statistical tools that strive to
maintain traceability links due to the added noise, thus approaches to (1)
detect multi-concern commits and (2) slice multi-concern commits into atomic
patches may aid reduce this source of noise. Further, this can have an added
benefit for developers since tools such as \lstinline+git bisect+ would also
benefit from such a segmentation of commits.

We propose a new multi-version representation of code, the \deltaPDGN. It is
natural to think of identification of communities in it as a slicing problem
~\cite{slicingsurvey}. However, boundaries across concerns do not naturally map
to a slicing criteria; it is unclear on how to seed the slicing algorithm and
when to terminate it. This stems from concerns sharing edges making their
separation into disjoint entities using a slicing criteria difficult to specify.
In the rest of this section, we discuss the literature around the problem of
tangled commits and the theoretical foundations of our approach.

\subsection{Impact of Tangled Commits}
\label{chapter:literature:sec:flexeme_rel_work:impact}

Tao \etal~\cite{Tao2012} were amongst the first to highlight the problem of
change decomposition in their study on programmer code comprehension; they
highlight the need for decomposition when many files are touched, multiple
features implemented, or multiple bug fixes committed. The last is diagnosed by
Murphy-Hill \etal~\cite{Murphy-Hill2012} as a deliberate practice to improve
programmer productivity. Tao \etal conclude that decomposition is required to
aid developer understanding of code changes.

Independently, Herzig \etal~\cite{Herzig2013, Herzig2016} investigate the impact
of tangled commits on classification and regression tasks within software
engineering research. The authors manually classify a corpora of real-world
changesets as atomic, tangled or unknown, and find that the fraction of tangled
commits in a series of version histories ranges from 7\% to 20\%; they also find
that most projects contain a maximum of four tangled concerns per commit, which
is  consistent with previous findings by Kawrykow and
Robillard~\cite{Kawrykow2011}. They find discover non-atomic commits
significantly impact the accuracy of classification and regression tasks such as
fault localisation. 

\subsection{Untangling Commits into Atomic Patches}
\label{chapter:literature:sec:flexeme_rel_work:untangle}

It is natural to think of identification of communities in the \deltaPDGN as a
slicing problem~\cite{slicingsurvey}. However, boundaries across concerns do not
naturally map to a slicing criterion; it is unclear how to seed a slicing
algorithm and when to terminate it. This is because concerns are linked with
multiple edges which makes their separation difficult to specify with a slicing
criterion. In the rest of this section, we discuss the literature around the
problem of tangled commits and the theoretical foundations of Flexeme.

Research on the impact of both tangled commits and non-essential code changes
prompted an investigation into changeset decomposition. Herzig
\etal~\cite{Herzig2013, Herzig2016} apply confidence voters in concert with
agglomerative clustering to decompose changesets with promising results,
achieving an accuracy of $0.58$-$0.80$ on an artificially constructed dataset
that mimics common causes of tangled commits. In contrast, Kirinuki
\etal~\cite{Kirinuki2014, Kirinuki2017} compile a database of atomic patterns to
aid the identification of tangled commits; they manually classify the resulting
decompositions as True, False, or Unclear, and find more than half of the
commits are correctly identified as tangled. The authors recognise that
employing a database introduces bias into the system and may necessitate
moderation via heuristics, such as ignoring changes that are too fine-grained or
add dependencies.

Other approaches rely on dependency graphs and use-define chains: Roover
\etal~\cite{Roover2017} use a slicing approach to segment commits across a
Program Dependency Graph, and correctly classify commits as (un)tangled in over
90\% of the cases for the systems studied, excluding some projects where they
are hampered by toolchain limitations. They propose, but do not implement, the
use of System Dependency Graphs to reduce some of the limitations of their
approach, such as being solely intraprocedural. Flexeme tackles interprocedural
and cross-file dependencies by merging the $\delta$-PDGs of the files touched by
a commit.  

Barnett~\etal~\cite{Barnett2015} implement and evaluate a commit-untangling
prototype. This prototype projects commits onto def-use chains, clusters the
results, then classifies the clusters as trivial or non-trivial. A cluster is
trivial if its def-use chains all fall into the same method. Barnett~\etal
employ a mixed approach to evaluate their prototype. They manually investigated
results with few non-trivial clusters (0-1), finding that their approach
correctly separated 4 of 6 non-atomic commits,  or many non-trivial clusters (>
5), finding that, in all cases, their prototype's sole reliance on def-use
chains lead to excessive clustering. For results containing 2--5 clusters, they
conducted a user-study. They found that 16 out of the 20 developers surveyed
agreed that the presented clusters were correct and complete. This result is
strong evidence that their lightweight and elegant approach is useful,
especially to the tangled commits that Microsoft developers encounter
day-to-day. During the interviews, multiple developers agreed that the changeset
analysed did indeed tangle two different tasks, sometimes even confirming that
developers had themselves separated the commit in question after review. In
addition to validating their prototype, their interviews also found evidence for
the need for commit decomposition tools. Because they use def-use chains and
ignore trivial clusters, Barnett~\etal's approach can miss tangled concerns that
Flexeme can discern. Barnett~\etal's user study itself shows that this can
matter: it reports that some developers disagreed with the classification of
some changesets as trivial.

Dias~\etal~\cite{Dias2015} take a more developer-centric approach and propose
the EpiceaUntangler tool. They instrument the Eclipse IDE and use confidence
voters over fine-grained IDE events that are later converted into a similarity
score via a Random Forest Regressor. This score is used similarly to Herzig
\etal~\cite{Herzig2016}'s metrics, \ie to perform agglomerative clustering. They
take an instrumentation-based approach to harvest information that would
otherwise be lost, such as changes that override earlier ones. This approach
also avoids relying on static analysis. They report a high median success rate
of 91\% when used by developers during a two-week study. While Dias~\etal
sidestep static analysis, they require developers to use an instrumented IDE.
Heddle is complementary to EpiceaUntangler: it allows reviewers to propose
untanglings of code that may originate from development contexts where
instrumentation might not be possible.

\subsection{Multiversion Representations of Code}
\label{chapter:literature:sec:flexeme_rel_work:multiversion_repr}

Related work has considered multiversion representations of programs for static
analysis. \citet{kim2006program} investigate the applicability of different
techniques for matching elements between different versions of a program. They
examine different program representations, such as String, AST, CFG, Binary or a
combination of these as well as the tools that work on them on two hypothetical
scenarios. They only consider the ability of the tools to match elements across
versions and leave the compact representation of a multiversion structure as
future work. Some of the conclusions from the matching challenges presented by
\citet{kim2006program} are echoed in Flexeme as well, we make use of the UNIX
diff as it is stored within version histories; however, we also make use of
line-span hints from the compilers for each version of the application to better
facilitate matching nodes within a NFG.

Le~\etal~\cite{Le2014} propose a Multiversion Interprocedural Control Graph
(MVICFG) for efficient and scalable multiversion patch verification over systems
such as the PuTTY SSH client. Our $\delta$-PDG is a generalisation of this
approach to a more expressive data structure, with applications beyond
traditional static analysis. 

Alexandru~\etal~\citep{alexandru2019redundancy} generalise the Le~\etal MVICFG
construction to arbitrary software artefacts by constructing a framework that
creates a multiversion representation of concrete syntax trees for a git
project. They adopt a generic ANTLR parser, allowing them to be language
agnostic, and achieve scalability by state sharing and storing the
multi-revision graph structure in a sparse data structure. They show the
usefulness of such a framework by means of `McCabe’s Complexity', which they
implement in this framework such that it is language agnostic, does not repeat
computations unnecessarily and reuses the data stores in the sparse graph by
propagating from child to parent node. \citet{Sebastian2018} propose a compact,
multiversion AST that cleverly shares state across versions. Flexeme, in
contrast, rests on PDGs and is well-suited for the untangling tasks, as our
evaluation demonstrates.

\subsection{Semantic Slicing of Version Histories}
\label{chapter:literature:sec:flexeme_rel_work:sem_slice_vh}

Features in a system often co-evolve, which tangles the changes made for a one
high-level feature with others in a version history. To resurface
feature-specific changes, they dynamically slice a target version, then walk
backwards in history while they can reverse the intra-version patch without
conflict; at each version they reach, they add any commit that contains a hunk
that touches the current slice to it. The goal of this semantic slicing of
version histories is to find a minimal slice of a version history that captures
the evolution of a feature. Li~\etal~\cite{Li2018, li2015semantic} first
formulated and introduced this problem. Semantic slicing is a form of commit
untangling backwards through history.  This retrospective framing is why they
treat the history as immutable. In this initial solution, Li~\etal treat commits
as atomic so their slices may contain noise introduced by tangled commits. To
reduce this noise, Li~\etal, in more recent work~\cite{Li2019}, unpack commits
into single-file commits into a private, local history. Flexeme, in contrast, is
static and online:  built from the ground up to rewrite commits as developer
make history.  
As such Flexeme and semantic slicing are complementary:  Flexeme would improve
the signal to noise ratio of semantic slicing. An interesting direction for
future work would be to use Flexeme to preprocess version histories prior to
semantically slicing them as with Definer~\cite{Li2019}.

\subsection{Graph Segmentation, Graph Kernels, and Community Detection}
\label{chapter:literature:sec:flexeme_rel_work:graph_seg}

\todo{Update to be Graph Kernels as the focus.}

An active area of research is the inference of labels from a graph structure,
and a common benchmark used for the methods proposed is the Stochastic Block
Model (SBM)~\cite{HOLLAND1983}. This problem can be formulated in an
unsupervised, semi-supervised, or supervised fashion. Early solutions focused on
using graph operators and their spectral properties to infer labels, such as
using the graph Laplacian or the non-Backtracking operator~\cite{Krzakala2013}.
These achieved good results at a high computational cost, the latter able to
detect communities down to the theoretical limit imposed by the Signal-to-Noise
ration for graphs sampled from an SBM. Saade \etal~\cite{Saade2014} later
propose using the Bethe Hessian or deformed Laplacian to compute the community
labels. They relate this operator to the non-Backtracking operator and that it
can also achieve community detection down the Signal-to-Noise floor.

In software engineering, spectral clustering has been used by Zhang
\etal~\cite{Zhang2016} to perform cross-project defect prediction in an
unsupervised manner, achieving results comparable to supervised classifiers in
both within- and cross-project settings. To our knowledge, such segmentation
techniques have not been employed for other tasks within SE.

Another family of approaches considered for graph prediction tasks are the graph
kernels~\cite{vishwanathan2010graph}. In domains where relational data is
naturally occurring, such as bioinformatics, social networks and others,
considering approaches where information is represented as a graph is natural.
Within such frameworks it is also natural to ask questions such as: `How similar
are two nodes in the same/different graphs?' or `How similar are two graphs?'.
Kernels offer to learn a notion of similarity~\cite{smola1998learning}, and
graph kernels offer to answer this for structured data. In our case, we consider
the \deltaPDGN representation of methods and use this approach to learn a
similarity function such that edits that are from the same atomic patch are more
similar.

A further approach, that is considered as a stretch goal, is to make use of the
recent advances in Graph Neural Networks~\cite{scarselli2008graph} for learning
a similarity function. Specifically, Graph Matching
Networks~\cite{DBLP:journals/corr/abs-1904-12787} offer to jointly learn
embeddings of two or more graphs for the purposes of a similarity function. Both
this approach, or the graph kernels approach would nevertheless still require an
agglomerative clustering step at the end to materialise the actual atomic commit
suggestions.

\section{Discerning Text from Code}
\label{chapter:literature:sec:posit_rel_work}

One of the meta-task that can aid the traceability aspect of this research is
the segmentation of English, or natural languages in a more general sense, from
formal languages. While online fora perform a best effort attempt at maintaining
separate formatting for the two modalities of text, this signal remains
noisy~\cite{ponzanelli2014improving}. The work carried out for our tool seeks to
improve the situation by framing the task as a code-switching phenomenon and
borrowing results from the relevant areas of the Natural Language Processing
(NLP) community.

In this section, we will focus on first providing how similar techniques were
already employed within the Software Engineering (SE) community, followed by the
more relevant for this task research from the NLP community, concluding with
Ponzanelli \etal's work on \SO which is closest to our work.

In SE research, part-of-speech tagging has been directly applied for identifier
naming~\cite{Binkley2011}, code summarisation~\cite{Haiduc2010a, Haiduc2010b},
concept localisation~\cite{Abebe2010}, traceability-link
recovery~\cite{Capobianco2013}, and bug fixing~\cite{Tian2015}. The main
intuition behind the statistical approaches in this are are due to the
naturalness of code as discussed by Hindle \etal~\cite{hindle2012naturalness}.

Operating directly on source code (not mixed text), Newman
\etal~\cite{Newman2017} created source code equivalents for lexeme categories
from natural languages. The looked at the behaviour exhibited by Proper Nouns,
Nouns, Pronouns, Adjectives, and Verbs and derived similar notions for
source-code from 1) Abstract syntax trees, 2) how the tokens impact memory, 3)
where they are declared, and 4) what type they have. They report the prevalence
of these categories in source-code. Their goal was to map these code categories
to PoS tags, thereby building a bridge for applying NLP techniques to code for
tasks such as program comprehension. Treude \etal~\cite{Treude2015portuguese}
described the challenges of analysing software documentation written in
Portuguese which commonly mixes two natural languages (Portuguese and English)
as well as code. They suggested the introduction of a new part-of-speech tag
called Lexical Item to capture cases where the ``correct'' tag cannot be
determined easily due to language switching.

We first review natural language processing (NLP) research on code-switching,
the natural language analogue of the mixed text problem.  This is work on which
we based POSIT.  Then we discuss initial efforts to establish analogues for
parts of speech categories for code and use them to tag code tokens.  We close
with the pioneering work on \stormed, the first context-free work to
automatically tackle the mixed text tagging problem.

NLP researchers are growing more interested in code-switching text and speech,
\ie that mixes multiple natural languages, among other reasons due to the higher
availability of corpora. These were made available either by dedicated
collections efforts, such as Miami Bangor~\cite{bangorTalk}, or due to social
media sites, such as twitter, that cause the mixing of English with other
languages~\cite{Vyas2014}. Previously, such data was scarce because
code-switching was stigmatised~\cite{Poplack1980}. 

Focusing specifically on part-of-speech tagging, Solorio and
Liu~\cite{Solorio2008} presented the first statistical approach to the task of
part-of-speech (PoS) tagging code-switching text. On a Spanglish corpus, one
that mixes Spanish and English, they heuristically combine PoS taggers trained
on larger monolingual corpora and obtain 85\% accuracy. We can think of this
approach as a mixtures of experts model. Jamatia \etal~\cite{Jamatia2015},
working on an English-Hindi corpus gathered from Facebook and Twitter, recreated
Solorio's and Liu's tagger and additionally proposed a tagger built on
Conditional Random Fields. The mixtures of experts model performed better at
72\% vs 71.6\%. In 2018, Soto and Hirschberg~\cite{Soto2018} proposed a neural
network approach, opting to solve two related problems simultaneously:
part-of-speech tagging and Language ID tagging. The second task can be thought
of as a segmentation task, which in our English-code setting can be rendered
using formatting for code-blocks. They combined a biLSTM with a CRF network at
both outputs and fused the two learning targets by simply summing the respective
losses. This network achieves a test accuracy of 90.25\% on the inter-sentential
code-switched dataset from Miami Bangor~\cite{bangorTalk}.

In the context of natural languages mixed with source code, Ponzanelli \etal are
the first to go beyond using regular expressions to parse such mixed text. When
customising LexRank~\cite{Ponzanelli2015b}, a summarisation tool for mixed text,
they employed an island grammar that parses Java and stack-trace islands
embedded in natural language, which is relegated to water. They followed up
LexRank with \stormed, a tool that uses an island grammar to parse Java, JSON,
and XML islands in mixed text \SO posts, again relegating natural language to
water~\cite{Ponzanelli2015a}.  \stormed produces heterogeneous abstract syntax
trees (AST), which are ASTs decorated with natural language snippets. They make
both the tool and the resulting corpus available. \stormed, however, remains
reliant on either code-block annotations or heuristics to discern code from
English. POSIT instead learns this segmentation from data, and, after a manual
evaluation, we indeed observed it to be capable of segmenting \SO posts where
submitters forgotten to include the appropriate formatting.

In extending this work from the single-natural-language and
single-formal-language case implemented as POSIT, we can look to the more recent
advances in NLP that are due to the adoption of the Transformers
model~\cite{vaswani2017attention, wang2019language}. They have been shown to
learn long-range dependencies, suggesting that they would be able to relate
single token code-occurrences to their repeated appearance in longer snippets.
Further, since in a setting where multiple natural languages and formal
languages are mixed, we expect each respective language to be low-resource; thus
we expect Transformers to offer a better promise of solving the task as they
have been shown to fare better in such scenarios~\cite{lakew2017multilingual}.