%%
%% This file uses the hyperref package to make your thesis have metadata embedded in the PDF, 
%%  and also adds links to be able to click on references and contents page entries to go to 
%%  the pages.
%%

% Some hacks are necessary to make bibentry and hyperref play nicely.
% See: http://tex.stackexchange.com/questions/65348/clash-between-bibentry-and-hyperref-with-bibstyle-elsart-harv
\usepackage{bibentry}
\makeatletter\let\saved@bibitem\@bibitem\makeatother
\usepackage[pdftex,colorlinks]{hyperref}
\usepackage{bookmark}
\hypersetup{bookmarksnumbered = true,
            hypertexnames=false,
            unicode = true,
            psdextra = true, 
            breaklinks = true,
            citecolor = {\green}, 
            urlcolor = {blue},		
            filecolor = {blue}, 
            linkcolor = {blue},}
% Needed for sane references, used in papers, must be after hyperref
\usepackage[nameinlink]{cleveref}
% Define label for Definition environments
\creflabelformat{defn}{Definition\textup{#1}}
% Again, after hyperlink by requirement, used in Flexeme
\usepackage[noend]{algorithmic} % may not be installed
\usepackage{algorithm}
\makeatletter\let\@bibitem\saved@bibitem\makeatother
\makeatletter
\AtBeginDocument{
    \hypersetup{
        pdfsubject={Computer Science},
        pdfkeywords={project health, traceability, mixed-text, commit untangling, graph similarity, biLSTM, mondrian forests},
        pdfauthor={Profir-Petru P\^arțachi},
        pdftitle={Improving Software Project Health Using Machine Learning},
    }
}
\makeatother
    
