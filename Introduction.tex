\chapter{Project Health}
\label{chapter:introduction}

Some projects are vibrant;  some moribund; others implode after a few commits.
Project health determines project success.  Project health encompasses:
\todo{Definition}. These tasks often involve a mixture of artefacts, both in
Natural and in Formal languages. 

In recent years systems that would previously live on different platforms have
been integrated under a single umbrella. The proliferation of GitHub, which
offers pull-requests, issue tracking and version history, and its integration
with other solutions such as Gerrit, for Code Reviews, Travis, for continuous
integration and delivery, as well as the response from competitors such as
GitLab and Atlassian to offer equivalent offerings has led to leaner and faster
development cycles. This has also reduced the cost of entry and created large,
publicly accessible sources of source-code together with related project
artefacts. Within this context, developers can benefit from the automation of
tasks, especially if they are oft forgotten or ignored in favour of traveling
light. 

This shift in tooling has also facilitated a shift in development paradigms,
developers now prefer a continuous integration/continuous delivery
infrastructure. This has led to more projects adopting a more agile development
process. Cleland-Huang \etal~\cite{Cleland-Huang2014} and St{\aa}hl
\etal~\cite{Stahl2017} observe that for traceability to be adopted within an
agile framework, it must remain within the traveling light paradigm.
Simultaneously, this shift has allowed for the proliferation of Natural Language
or Natural Language and Formal Language textual artefacts which are
programmatically accessible. This suggests that approaches from Natural Language
Processing and Machine Learning are now feasible and indeed desirable. This
thesis aims to (semi-)automate tasks for this new paradigm and its attendant
infrastructure by bringing to the foreground the relevant NLP and ML techniques.

Under this umbrella, we focus on three synergistic tasks from this domain: (1)
improving the issue-pull-request traceability, which can aid existing systems to
automatically curate the issue backlog as pull-requests are merged; (2)
untangling commits in a version history, which can aid the beforementioned
traceability task as well as improve the usability of determining a fault
introducing commit, or cherry-picking via tools such as git bisect; (3)
mixed-text parsing, to allow better API mining and open new avenues for
project-specific code-recommendation tools. 

The tasks range from more developer focused to more researcher focused; however,
all tasks aim to, at worst, provide benefits to developers as a side-product if
not directly helping them.

Under the three synergistic tasks, we have found the following ways of improving
the state-of-the-art:

Towards making traceability attractive for agile development, Ståhl \etal [2]
provide the Eiffel framework, which allows integrating different sources of
traceability information in a light-weight manner. AIDE-MEMOIRE seeks to provide
issue-pull-request traceability information, directly helping developers by
allowing features for automatic triaging that already exist to work more
efficiently. This fits within the future directions proposed by Ståhl \etal and
enabled by Eiffel, while the problem itself represents a recast of the Bachman
\etal’s “Missing Links” problem~\cite{MissingLinks} to modern development
practices. 

Traceability, importantly for this research, and other tasks such as fault
localisation or patch synthesis must contend with bias in the training data
whenever a statistical approach is considered. This bias is influenced by a
multitude of factors among which the mentioned “Missing Links” problem as well
as tangled commits as observed by Herzig \etal~\cite{Herzig2016} (Herzig and
Zeller~\cite{Herzig2013}). FLEXEME aims to forward the state-of-the-art in
commit untangling and thus enable the creation of better statistical tooling for
developers indirectly, while directly allowing them to improve the health of
their version histories. 

Despite the international nature of platforms such as GitHub, a significant
portion of software development information is in English. This leads to
projects in other natural languages to adopt English terms within their
processes as observed by Treude \etal~\cite{Treude2015portuguese}. Further, such
natural language can mix any natural languages it may use with formal languages.
This same phenomenon is observable on mailing lists, such as the Linux Kernel
Mailing List, and developer fora, such as StackOverflow. POSIT borrows from the
NLP literature that concerns itself with code-switching to provide solutions to
this problem within the Software Engineering context and allow extending methods
to languages other than English or resources that freely mix natural and formal
languages.

This thesis aims to improve the state-of-the-art of (semi-)automation of tasks
and the attending infrastructure of Continuous Integration and Development
projects. It proposes to do so by advancing the state-of-the-art for three
problems: (1) issue-pull-request traceability, (2) commit untangling, and (3)
the parsing of mixed-text containing formal languages. All solutions should be
targeting a continuous integration, continuous delivery context. Further, these
solutions interact synergistically, the latter two improve the quality of the
training data for the first, while also improving the general health of
developer version histories (2) or developer issue trackers and fora (3).

