\chapter{Introduction}
\label{chapter:introduction}

The software development process involves a multitude of aspects: the
specification and design, the architecture of the system, the code writing
itself, code version histories, issue tracking, code review. In an opensource
project, external collaboration and their integration are also a part of it.
Last but not least, the developers themselves and the project culture further
define the development process. Some projects are vibrant; some moribund; others
implode after a few commits. \emph{Project health} determines project success.
Project health encompasses: social, technical, and traceability aspects of a
project. It is defined by the quality of code, documentation, version histories,
issue tracking, specifications and requirements. It is further defined by the
quality of the interlinking of these aspects. Finally, but not less important,
it is defined by the culture of the project, be that explicitly codified in a
file/manifesto or implicit in the interactions of developers. In this work, I
focus only on the technical and traceability aspects of project health. It is of
no surprise that these processes produce artefacts that span and intermix a
profusion of formats, natural languages, and programming languages.

In recent years systems, which would previously live on different platforms,
have been integrated under a single umbrella, thus the different aspects that
can impact project health can be centrally observed. The proliferation of
GitHub, which offers pull-requests, issue tracking and version history, and its
integration with other solutions such as Gerrit (for Code Reviews), Travis (for
continuous integration and delivery), as well as the response from competitors,
such as GitLab and Atlassian who offer equivalent offerings, has led to leaner
and faster development cycles: GitHub advertises their CI offerings with a
promise to reduce time spent merging commits or debugging and increase time
spend writing code~\cite{GitHubCI}. This has also reduced the cost of entry and
created large, publicly accessible sources of source-code together with related
project artefacts, such as GHTorrent~\cite{GHTorrent},
CodeSearchNet~\cite{Husain2019} and similar datasets. Within this context,
developers can benefit from the automation of tasks, especially if they are
often forgotten or ignored in favour of traveling light. 

This shift in tooling has also facilitated a shift in development paradigms,
developers can now more easily adopt a continuous integration/continuous
delivery infrastructure. This has led to more projects adopting a more agile
development process. Cleland-Huang~\etal~\cite{Cleland-Huang2014} and St{\aa}hl
\etal~\cite{Stahl2017} observe that, for traceability to be adopted, within an
agile framework, it must remain within the traveling light paradigm.
Simultaneously, the shift to continuous improvement/continuous development has
allowed for the proliferation of Natural Language or Natural Language and Formal
Language textual artefacts that are programmatically accessible. This suggests
that approaches from Natural Language Processing and Machine Learning are now
feasible and indeed desirable. This thesis aims to (semi-)automate tasks for
this new paradigm and its attendant infrastructure by bringing to the foreground
the relevant NLP and ML techniques.

\section{Problem Statements}
\label{chapter:introduction:sec:problem_statement}

Under Improving Project Health, I focus on three synergistic tasks: (1)
improving the issue-pull-request traceability, which can aid existing systems to
automatically curate the issue backlog as pull-requests are merged; (2)
untangling commits in a version history, which can aid the beforementioned
traceability task as well as improve the usability of determining a fault
introducing commit, or cherry-picking via tools such as git bisect; (3)
mixed-text parsing, to allow better API mining and open new avenues for
project-specific code-recommendation tools. 

Software development is a community task. The programming task itself is a
component of a much wider process. It is, however, a task that is viewed
exclusively when considered by developers at the cost of other administrative
tasks. This manifests itself as, for example, in the case of traceability, which
companies recognise as important, by organisations reaching an insufficient
level of traceability~\cite{mader2009motivation}. These administrative tasks
come with the promise of facilitating internal follow-ups, evaluation or
improvement of the development process~\cite{domges1998adapting}. Agile
practice, and especially Continuous Integration, by suggesting a lean approach
to documentation, reduces the direct applicability of existing approaches,
though work has been done to adapt to the new paradigm:
Ståhl~\etal~\cite{Stahl2017} provide the Eiffel framework, which allows
integrating different sources of traceability information in a light-weight
manner. Indeed, the conflict arises between the traditional assumptions of
Traceability and the agile team's tendencies of traveling quite
light~\cite{Cleland-Huang2014}. Indeed, \emph{traveling light} means forgoing
documentation, user stories not being punted to new sprints, or, more generally,
administrative tasks that distract from a developer coding being ignored. If
combined with backlog refinement~\cite{BacklogRefinement}, traveling light can
manifest itself as a form of project amnesia. This is a form of technical debt,
and will degrade project health in the long-run.

To help developers, tool smiths themselves require supporting artefacts and
utilities. One of these is access to high quality data that can be used both to
train a human intuition for a task as well as statistical approaches to enable
developers from less well maintained projects improve their situation. The
proliferation of web-sites such as GitHub, BitBucket/JIRA, GitLab, and others
that enable developers to have commits, issues, pull-request, code-reviews and
other development processes under one umbrella also opens a potential resource
to researchers and tool smiths that wish to study the properties of the
inter-relationships of such artefacts. In this context, traceability can help
developers maintain the collective memory of projects. When using such
resources, the promise of automatic triaging of issues as commits resolving them
are accepted into the mainline incentivises developers to maintain
code-review/issue to commit links. I have, however, observed a lack of such
links as far as GitHub is concerned, despite an encouragement of contributors to
do so in the contribution guidelines (\Cref{am:sec:eval:prelim}). Further,
developers only partially recording such link introduces bias in the datasets
researchers use.

\begin{tcolorbox}[title=Pull-request-Issue Traceability]
    Given a project, developers should receive traceability link suggestions
    when pertinent, \ie when they are already performing a task within which a
    link can be recorded. For a pull-request-issue traceability, suggestions are
    to be made at issue triage and PR submission.
\end{tcolorbox}

Additionally, links may not be clear if the fixes are presented as unfocused
patches, hiding bug fixing beyond many lines of refactoring.
Herzig~\etal~\cite{Herzig2016} (Herzig and Zeller~\cite{Herzig2013}) show that
tangled commits introduce bias in defect prediction. In our work, I aim to
provide a diff time solution for commit untangling to help developers reduce
this bias. The promise to developers is enabling tooling such as git bisect.

\begin{tcolorbox}[title=Commit Untangling]
    Given a commit, the patch should be decomposed into smaller patches such
    that each patch tackles a single task, \ie would be linked to a single issue
    or implements a single stakeholder concern. On atomic commits, this process
    should be the identity function.
\end{tcolorbox}

Despite the international nature of platforms such as GitHub, a significant
portion of software development information is in English. This leads to
projects in other natural languages to adopt English terms within their
processes as observed by Treude~\etal~\cite{Treude2015portuguese}. Further, such
natural language can mix any natural languages it may use with formal languages.
This same phenomenon is observable on mailing lists, such as the Linux Kernel
Mailing List, and developer fora, such as StackOverflow.  Further, despite
access to mark-up to help developers demarcate code from English, in practice,
even on StackOverflow submitters do forget to do so. This reduces the applicable
research tools which use such formatting cues.

\begin{tcolorbox}[title=Mixed-text Parsing]
    Given an artefact that freely mixes natural languages and formal languages,
    separate the languages. Further, for each token, provide the tag in the
    language from which the token is taken. In natural languages, these tags are
    part-of-speech tags; in formal languages, these tags are the AST tags of the
    parents of the terminals.
\end{tcolorbox}

\noindent The tasks range from more developer focused to more researcher
focused; however, all tasks aim to always provide a benefit to developers,
directly or indirectly.

Overall, the focus of this research is the creation of tools and
meta-tools that (semi-)automate development tasks to enable project to remain
healthy. PR-issue traceability focuses directly on improving the development
process by reducing the administrative burden on the developers, while commit
untangling and mixed-text parsing focus on enabling a wider range of tools to be
applicable. The goal as a whole is an improvement of the development process by
allowing more time to be dedicated to the programming task rather than tasks
required to maintain the project health.

\section{Contributions}
\label{chapter:introduction:sec:contrib}

Under the three synergistic tasks, I have found the following ways of improving
the state-of-the-art, which can benefit both practitioners as well as
researchers; This thesis:

\begin{enumerate}
    \item Provides an online pull-request traceability link inference algorithm
    realised as Aide-mémoire. It solves an online, modern variant of the
    ``Missing Links'' problem proposed by Bachman~\etal~\cite{MissingLinks}. It
    departs from using commits in favour of pull-requests. Aide-mémoire suggests
    links when developers are already handling a PR or an issue.

    \item Forwards the state-of-the-art in commit untangling both in accuracy
    and in speed and thus enables the creation of better statistical tooling for
    developers indirectly, while directly allowing them to improve the health of
    their version histories. I realise this as Flexeme. Flexeme offers precise
    untangling suggestions within 10 seconds; fast enough to be viable in the
    code-review process.

    \item Formulates the mixed-text parsing problem, posing it as a simultaneous
    segmentation and tagging task. It proposes a solution, POSIT, which borrows
    from the Natural Language Processing and Linguistics literature that
    concerns itself with code-switching. It maps the borrowed results to the
    Software Engineering context and allows extending methods to languages other
    than English or resources that freely mix natural and formal languages.
\end{enumerate}

The solutions I propose can help developers directly, by improving the state of
the project artefacts, or indirectly by helping tool smiths. Further, these
solutions interact synergistically, Flexeme and POSIT improve the quality of
the training data for Aide-mémoire, while also improving the general health of
developer version histories (Flexeme) or developer issue trackers and fora
(POSIT).

\section{List of Papers}
\label{chapter:introduction:sec:papers}

A long research project, like a PhD, is a close collaboration where it is
difficult, often impossible, to separate each collaborator's contributions. What
is clear is who took the lead on which aspects and tasks. The reader will also
notice a mixed use of `I' and `we', I continue to use `we' in the paper chapters
when referring to work done in close collaboration with my co-authors to
acknowledge their contribution. Further, the papers are presented as published
or submitted for review spare figures and tables that required editing to fit
into the thesis format. I now enumerate the tasks I lead in each of the papers
in this thesis.

\begin{itemize}[leftmargin=*]
    \item[]\noindent\textbf{1. Aide-mémoire: Improving a Project’s Collective Memory via Pull
Request-Issue Links} 

    \noindent\emph{Authors: Profir-Petru Pârțachi, David R. White, Earl T. Barr}

    \noindent\emph{Venue: Submitted to ACM Transactions on Software Engineering
    and Methodology (TOSEM)}

    \noindent I determined and mined the dataset of GitHub projects. I
    implemented the proposed PR-Issue linker, performed the exploratory data
    analysis and feature engineering as well as wrote the evaluation and result
    analysis scripts. The experimental and EDA designs were done in close
    collaboration with David White who guided me patiently.

    \item[]\noindent\textbf{2. Flexeme: Untangling Commits Using Lexical Flows}

    \noindent\emph{Authors: Profir-Petru Pârțachi, Santanu Kumar Dash, Miltos
    Allamanis, Earl T. Barr}

    \noindent\emph{Venue: Proceedings of 28th ACM Joint European Software
    Engineering Conference and Symposium on the Foundations of Software
    Engineering, (ESEC/FSE 2020)}

    \noindent I implemented in full the construction of our new structure,
    though the idea of the structure was from Earl Barr, and the details of its
    specification were worked out in close collaboration with him. I also
    implemented the necessary methods to construct the dataset, reproduced
    previous work in the area and implemented our proposed untangling algorithms
    as well as their evaluation on the constructed dataset. The RefiNym
    implementation was provided by Santanu Dash who helped me integrate it with
    Flexeme. The original PDG extraction implementation for C\# code was
    provided by Miltos Allamanis. Santanu Dash and I performed the manual
    evaluations.

    \item[]\noindent\textbf{3. POSIT: Simultaneously Tagging Natural and Programming
    Languages} 

    \noindent\emph{Authors: Profir-Petru Pârțachi, Santanu Kumar Dash, Christoph
    Treude, Earl T. Barr}

    \noindent\emph{Venue: Proceedings of 42nd International Conference on
    Software Engineering (ICSE ’20)}

    \noindent I implemented the preprocessing scripts, the Neural Network
    that realises POSIT, the adaptation of the previous state-of-the-art to our
    problem, and the necessary evaluation scripts. The Code Comments corpus was
    provided by Santanu Dash. The original implementation of TaskNav was
    provided by Christoph Treude. The informal proof of context-sensitivity of
    mixed-text was worked on in close collaboration with Earl Barr without whom
    the proof would have not been finished in a timely manner. The manual
    evaluation of POSIT was done together with Santanu Dash, while the manual
    evaluations of TaskNav augmented with POSIT were done together with
    Christoph Treude.
\end{itemize}

\section{Thesis Organisation}
\label{chapter:introduction:sec:organisation}

The remainder of the thesis is organised as follows.

\Cref{chapter:literature} first presents literature on the areas encompassed by
project health. It then focuses on the areas touched by the proposed tasks in
the order they are presented in the thesis. It first discusses traceability and
the ``Missing Links'' problem~\cite{MissingLinks}, it then focuses on the issues
caused by tangled commits and approaches to the commit untangling, it concludes
with an exploration of existing approaches to mixed-text: how do researchers
tackle code when applying natural language techniques and how the natural
language processing community handles the mixing of multiple natural languages.

\Cref{chapter:am} presents Aide-mémoire, a solution to online pull-request-issue
linking problem. It extends the ``Missing Links'' problem due to
Bachmann~\etal~\cite{MissingLinks} to pull-requests and proposes a solution that
can live along side the code-review process. Aide-mémoire is evaluated on an
extensive corpus of GitHub projects.

\Cref{chapter:flexeme} presents Flexeme, a solution to the commit untangling
problem. It shows that Flexeme improves the state-of-the-art while being fast
enough to be viable during code-review. Flexeme is evaluated on an artificial
corpus that is rigorously constructed to mimic tangled commits
Herzig~\etal~\cite{Herzig2016} have observed developers make.

\Cref{chapter:posit} presents POSIT, where the mixed-text parsing problem is
introduced together with our proposed solution: POSIT. In it, we argue that the
mixed-text parsing problem is context-sensitive in the general case and hence
propose a distributional semantics approach, realised as a neural network, for
the problem. We show that POSIT can both directly improve software fora by
suggesting mixed code annotations, as well as aid other research tools: we show
POSIT to improve TaskNav's~\cite{Treude:2015:TTN:2819009.2819128} recall.

\Cref{chapter:conclusions} concludes the thesis and provides directions for
future work.