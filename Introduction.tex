\chapter{Introduction}
\label{chapter:introduction}

\todo{Intro for Intro}

\section{Project Health}
\label{chapter:introduction:sec:proj_health}

Some projects are vibrant;  some moribund; others implode after a few commits.
Project health determines project success.  Project health encompasses:
\todo{Definition}. These tasks often involve a mixture of artefacts, both in
Natural and in Formal languages. 

In recent years systems that would previously live on different platforms have
been integrated under a single umbrella. The proliferation of GitHub, which
offers pull-requests, issue tracking and version history, and its integration
with other solutions such as Gerrit, for Code Reviews, Travis, for continuous
integration and delivery, as well as the response from competitors such as
GitLab and Atlassian to offer equivalent offerings has led to leaner and faster
development cycles. This has also reduced the cost of entry and created large,
publicly accessible sources of source-code together with related project
artefacts. Within this context, developers can benefit from the automation of
tasks, especially if they are oft forgotten or ignored in favour of traveling
light. 

This shift in tooling has also facilitated a shift in development paradigms,
developers now prefer a continuous integration/continuous delivery
infrastructure. This has led to more projects adopting a more agile development
process. Cleland-Huang \etal~\cite{Cleland-Huang2014} and St{\aa}hl
\etal~\cite{Stahl2017} observe that for traceability to be adopted within an
agile framework, it must remain within the traveling light paradigm.
Simultaneously, this shift has allowed for the proliferation of Natural Language
or Natural Language and Formal Language textual artefacts which are
programmatically accessible. This suggests that approaches from Natural Language
Processing and Machine Learning are now feasible and indeed desirable. This
thesis aims to (semi-)automate tasks for this new paradigm and its attendant
infrastructure by bringing to the foreground the relevant NLP and ML techniques.

Under this umbrella, we focus on three synergistic tasks from this domain: (1)
improving the issue-pull-request traceability, which can aid existing systems to
automatically curate the issue backlog as pull-requests are merged; (2)
untangling commits in a version history, which can aid the beforementioned
traceability task as well as improve the usability of determining a fault
introducing commit, or cherry-picking via tools such as git bisect; (3)
mixed-text parsing, to allow better API mining and open new avenues for
project-specific code-recommendation tools. 

The tasks range from more developer focused to more researcher focused; however,
all tasks aim to, at worst, provide benefits to developers as a side-product if
not directly helping them.

Under the three synergistic tasks, we have found the following ways of improving
the state-of-the-art:

Towards making traceability attractive for agile development, Ståhl \etal [2]
provide the Eiffel framework, which allows integrating different sources of
traceability information in a light-weight manner. AIDE-MEMOIRE seeks to provide
issue-pull-request traceability information, directly helping developers by
allowing features for automatic triaging that already exist to work more
efficiently. This fits within the future directions proposed by Ståhl \etal and
enabled by Eiffel, while the problem itself represents a recast of the Bachman
\etal’s “Missing Links” problem~\cite{MissingLinks} to modern development
practices. 

Traceability, importantly for this research, and other tasks such as fault
localisation or patch synthesis must contend with bias in the training data
whenever a statistical approach is considered. This bias is influenced by a
multitude of factors among which the mentioned “Missing Links” problem as well
as tangled commits as observed by Herzig \etal~\cite{Herzig2016} (Herzig and
Zeller~\cite{Herzig2013}). FLEXEME aims to forward the state-of-the-art in
commit untangling and thus enable the creation of better statistical tooling for
developers indirectly, while directly allowing them to improve the health of
their version histories. 

Despite the international nature of platforms such as GitHub, a significant
portion of software development information is in English. This leads to
projects in other natural languages to adopt English terms within their
processes as observed by Treude \etal~\cite{Treude2015portuguese}. Further, such
natural language can mix any natural languages it may use with formal languages.
This same phenomenon is observable on mailing lists, such as the Linux Kernel
Mailing List, and developer fora, such as StackOverflow. POSIT borrows from the
NLP literature that concerns itself with code-switching to provide solutions to
this problem within the Software Engineering context and allow extending methods
to languages other than English or resources that freely mix natural and formal
languages.

This thesis aims to improve the state-of-the-art of (semi-)automation of tasks
and the attending infrastructure of Continuous Integration and Development
projects. It proposes to do so by advancing the state-of-the-art for three
problems: (1) issue-pull-request traceability, (2) commit untangling, and (3)
the parsing of mixed-text containing formal languages. All solutions should be
targeting a continuous integration, continuous delivery context. Further, these
solutions interact synergistically, the latter two improve the quality of the
training data for the first, while also improving the general health of
developer version histories (2) or developer issue trackers and fora (3).

\section{Problem Statements}
\label{chapter:introduction:sec:problem_statement}

\todo{what am I solving?}

Software development is a community task. The programming task itself is a
component of a much wider process. It is, however, a task that is viewed
exclusively when considered by developers at the cost of other administrative
tasks. This manifests itself as, for example, in the case of traceability which
companies to recognise as important by organisations reaching an insufficient
level of traceability~\cite{mader2009motivation}. These administrative tasks
come with the promise of facilitating internal follow-ups, evaluation or
improvement of the development process~\cite{domges1998adapting}. Agile
practice, and especially Continuous Integration, by suggesting a lean approach
to documentation, reduces the direct applicability of existing approaches,
though work has been done to adapt to the new paradigm~\cite{Stahl2017}.

To help developers, tool-smiths themselves require supporting artefacts and
utilities. One of these is access to high quality data that can be used both to
train a human intuition for a task as well as statistical approaches to enable
developers from less well maintained projects improve their situation. The
proliferation of web-sites such as GitHub, BitBucket/JIRA, GitLab, and others
that enable developers to have commits, issues, pull-request, code-reviews and
other development processes under one umbrella also open a potential resource to
researchers and tool-smiths that wish to study the properties of the
inter-relationships of such artefacts. In this context, traceability can help
developers maintain the collective memory of projects. When using such
resources, developers are incentivised to maintain code-review/issue to commit
links by the promise of automatic triaging of issues as commits resolving them
are accepted into the mainline. We have, however, observed a lack of such links
as far as GitHub is concerned, despite an encouragement of contributors to do so
in the contribution guidelines. Additionally, links may not be clear if the
fixes are presented as unfocused patches, hiding bug fixing beyond many lines of
refactoring. In our work, we try to make these task simpler to facilitate both
further research and adoption by developers.

Further, despite access to mark-up to help developers demarcate code from
English, in practice, even on StackOverflow, an active question-answering
community for coding related tasks, submitters do forget to do so. This reduces
the applicable research tools which use such formatting cues. To further
increase our model options, we also propose a new tagging and segmentation
approach to automate this task.

Overall, the focus of this research focuses on the creation of tools and
meta-tools that (semi-)automate development tasks within a Continuous
Integration project paradigm. The former focus directly on improving the
development process by reducing the administrative burden on the developers,
while the latter focus on enabling a wider range of tools to be applicable. The
goal as a whole is a improvement of the development process by allowing more
time to be dedicated to the programming task rather than tasks required to
maintain the project health.

\section{Goals and Objectives}
\label{chapter:introduction:sec:goals}

\todo{what do I want to achieve?}

\section{Contributions}
\label{chapter:introduction:sec:contrib}

\todo{How do I contribute to the general advancement of knowledge?}

\section{List of Papers}
\label{chapter:introduction:sec:papers}

\todo{Check with Earl that this is fine, otherwise, a we/I substitution might be needed}
A long research project, like a PhD, is a close collaboration where it is
difficult, often impossible, to separate each collaborator's contributions. What
is clear is who took the lead on which aspects and tasks. The reader will also
notice a mixed use of `I' and `we', I continue to use `we', including outside
the paper chapters, when referring to work done in close collaboration with my
co-authors as to acknowledge their contribution. Further, the papers are
presented as published or submitted for review spare figures and tables that
required editing to fit into the thesis format. I now enumerate the tasks I lead
on each of the papers included in this thesis.

\noindent\textbf{1. Aide-mémoire: Improving a Project’s Collective Memory via Pull
Request-Issue Links} 

\noindent\emph{Authors: Profir-Petru Pârțachi, David R. White, Earl T. Barr}

\noindent\emph{Venue: Under submission to ACM Transactions on Software
Engineering and Methodology (TOSEM)}

\noindent I have determined and mined the dataset of GitHub projects. I have
implemented the proposed PR-Issue linker, performed the exploratory data
analysis and feature engineering as well as wrote the evaluation and result
analysis scripts. The experimental and EDA designs were done in close
collaboration with David White who made sure I perform these correctly and
guided me patiently.

\noindent\textbf{2. Flexeme: Untangling Commits Using Lexical Flows}

\noindent\emph{Authors: Profir-Petru Pârțachi, Santanu Kumar Dash, Miltose
Allamanis, Earl T. Barr}

\noindent\emph{Venue: Proceedings of 28th ACM Joint European Software
Engineering Conference and Symposium on the Foundations of Software Engineering,
(ESEC/FSE 2020)}

\noindent  I have implemented in full the construction of our new structure,
though the idea of the structure was from Earl Barr, and the details of the
specification were worked in close collaboration. I also implemented the
necessary methods to construct the dataset, reproduced previous work in the area
and implemented our proposed untangling algorithms as well as their evaluation
on the constructed dataset. The RefiNym implementation was provided by Santanu
Dash who helped me integrate it with Flexeme. The original PDG extraction
implementation for C\# code was provided by Miltose Allamanis. Santanu Dash and
I performed the manual evaluations.

\noindent\textbf{3. POSIT: Simultaneously Tagging Natural and Programming
Languages} 

\noindent\emph{Authors: Profir-Petru Pârțachi, Santanu Kumar Dash, Christoph
Treude, Earl T. Barr}

\noindent\emph{Venue: Proceedings of 42nd International Conference on Software
Engineering (ICSE ’20)}

\noindent I have implemented the preprocessing scripts, the Neural Network that
realises POSIT, the adaptation of the previous State-of-the-art to our problem,
and the necessary evaluation scripts. The Code Comments corpus was provided by
Santanu Dash. The original implementation of TaskNav was provided by Christoph
Treude. The informal proof of context-sensitivity of mixed-text was worked on in
close collaboration with Earl Barr without whom the proof would have not been
finished in a timely manner. Manual evaluation of POSIT was done together with
Santanu Dash, while manual evaluations of TaskNav augmented with POSIT were done
together with Christoph Treude.

\section{Thesis Organisation}
\label{chapter:introduction:sec:organisation}

\todo{Lazy enum or drop this section}
