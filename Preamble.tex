% I may change the way this is done in a future version, 
%  but given that some people needed it, if you need a different degree title 
%  (e.g. Master of Science, Master in Science, Master of Arts, etc)
%  uncomment the following 3 lines and set as appropriate (this *has* to be before \maketitle)
% \makeatletter
% \renewcommand {\@degree@string} {Master of Things}
% \makeatother

\title{Improving Software Project Health using Machine Learning}
\author{Profir-Petru P\^ar\cb{t}achi}
\department{Department of Computer Science}

\maketitle
\makedeclaration

\begin{abstract} % 300 word limit
In recent years systems that would previously live on different platforms
have been integrated under a single umbrella. The proliferation of GitHub,
which offers pull-requests, issue tracking and version history, and its
integration with other solutions such as Gerrit, or Travis, as well as the
response from competitors has led to leaner and faster development cycles.
This has also reduced the cost of entry and created large, publicly
accessible sources of source-code together with related project artefacts.

This shift in tooling has also facilitated a shift in development paradigms,
developers now prefer a continuous integration/continuous delivery
infrastructure. This has led to more projects adopting a more agile
development process. Developers often forgo tasks that may aid project
health so that they can instead travel light. However, project health
determines project success.  Project health encompasses traceability,
documentation, adherence to coding conventions, in short tasks that allow
for lower maintenance costs and higher accountability.

Simultaneously, this shift has allowed for the proliferation of Natural
Language or Natural Language and Formal Language textual artefacts which are
programmatically accessible. This suggests that approaches from Natural
Language Processing and Machine Learning are now feasible and indeed
desirable. This thesis aims to (semi-)automate tasks for this new paradigm
and its attendant infrastructure by bringing to the foreground the relevant
NLP and ML techniques.

Under this umbrella, we focus on three synergistic tasks from this domain:
(1) improving the issue-pull-request traceability, which can aid existing
systems to automatically curate the issue backlog as pull-requests are
merged; (2) untangling commits in a version history, which can aid the
beforementioned traceability task as well as improve the usability of
determining a fault introducing commit, or cherry-picking via tools such as
git bisect; (3) mixed-text parsing, to allow better API mining and open new
avenues for project-specific code-recommendation tools.	
\end{abstract}

\begin{impactstatement}
The work presented as part of this thesis opens the door to further research in applications of machine learning towards aiding the process of software development and making the profession more accessible. Directly, it helps developers maintain projects in a state of lower technical debt and can, over time, enable more sophisticated analysis by researchers as the quality of the data improves. It also further paves the way towards multi-channel analysis of coding artefacts where signal from both code and natural language are considered. \todo{more stuff here about how this is impactful.}
\end{impactstatement}

\begin{acknowledgements}
\todo{Reduce repetitiveness}
I would like to thank my supervisor, Dr Earl T. Barr for the sage advice and
importantly patience with me especially as I was navigating the landscape of
academic writing. I would also like to thank Earl for bearing with me as I fell
ill throughout the journey and offering support during those times. The
dialectic we had trying to formally prove a hypothesis with the clock ticking to
the deadline will be a fond memory for years despite the stress of the moment
then.

I would like to thank Dr David R. White for the `Desk Compensation Meetings'
that set me on the path towards rigorous experimental design early in my PhD and
that formed the support beams for my empirical studies throughout my doctoral
work. I would also like to thank David for being there for me when I needed
someone to listen to me and helping navigate the academic landscape with more
confidence.

I would like to thank Dr Santanu Dash, for helping me navigate the rougher sides
of Roslyn and enabling the work that followed from that. I would also like to
thank Santanu for helping me pursue an idea that later became a paper by helping
me with initial labelled data that would have been impossible without his CLANG
knowledge. The help and advice navigating formal languages helped shape my work.

I would like to thank my lab mates that made the journey more fun: Leo Jeoffe,
David Kelly, Bobby Bruce, DongGyun Han, Matheus Paix\~ao, Carlos Gavidia,
Giovani Guizzo, Vali Tawosi, Rebecca Mousa, Jie Zhang. The informal chats were
invaluable and they all helped me through the journey. Discussing our work
together helped me, and I hope them as well, better understand how to do proper
science, to better understand how and when statistical techniques should be
applied, how to formulate experiments, how to thoroughly explore hypothesises.

\pp{I would like to thank my family: my parents and my elder sister.}

As the bulk of this thesis is composed of papers that have been published or are
under review, I would now like to carefully detail the contributions of the
authors so that my own work is also clear. Unless otherwise stated, the
engineering work was always performed by myself.

Aide-m\'emoire: This paper has gone through several iterations making writing
contributions difficult to trace as all three authors, Earl Barr, David White,
and I, have contributed throughout the paper. Initially, I contributed the
Feature Selection, Tool implementation, Results, and Conclusion sections as well
as the initial seed for the Related Work. David contributed the Motivating
Example, and Evaluation Set-up sections. Earl contributed the introduction as
well as the polished Related Work. This initial draft was then reworked by all
three authors several times as part of the journal submission process, as such
it is difficult to say that the current versions has sections that belong to a
certain person in particular. On the experimental side, David has significantly
shaped the experimental set-up, hypothesis and helped chose the appropriate
performance measures for our setting.

Flexeme: This paper has gone through two iterations before acceptance.
Engineering-wise, Santanu has contributed help debugging Roslyn code and help
setting up Refinym to work in our setting. Miltose has contributed the original
code for PDG extraction implemented in Roslyn, significant time helping me
navigate the Roslyn infrastructure, as well as compute time to generate our
dataset for later use. Writing wise, in the initial iteration, Earl focused on
the introduction and related work. Earl also dedicated significant time
improving the introduction and guiding me on how to better write the other
sections. Santanu, additional to his engineering contributions, also helped
create the figures and tighten the formal description of our proposed structure.
In the second iteration, Santanu further improved and updated the figures and
provided help editing throughout the paper. Earl has rewritten an updated
introduction and spent time teaching me how to better write academic English by
co-writing the remainder of the sections with me. As this was chronologically
after POSIT, I was given more solo writing tasks followed by co-editing passes
as learning opportunities. Miltose has helped re-write the formal parts of the
paper, making the definitions clearer and crisper ensuring that a careful reader
will understand them.

POSIT: This paper has gone through two iterations, similar to Flexeme, however,
unlike it, the second iterations could be considered a re-write from scratch.
The original paper was mostly written by Santanu, who focused on Introduction,
Neural Network description and figures, and me followed by a heavy editorial
pass from Earl, with a heavy focus on a good introduction. This work, however,
does not shine much through to the final version. The engineering work did
persist. Here, Santanu contributed one of the main enablers of this research,
the initial labelled data extracted from CLAG compilations, which I later
extended to label StackOverflow posts. Santanu also helped with manual
evaluation of POSIT over the Linux Kernel Mailing list by being the second
manual rater. In the second iteration, Christoph Treude contributed his TaskNav
tool to the project which we used as a downstream tool example for POSIT.
Writing wise, the introduction belongs to Earl, while the other sections were
used as a learning opportunity for me and were co-written. Christoph helped
navigate TaskNav and contributed significantly to evaluation by being the second
rater for manual evaluations on TaskNav tasks. All authors have contributed
editorial passes over the paper.


\end{acknowledgements}

\setcounter{tocdepth}{2} 
% Setting this higher means you get contents entries for
%  more minor section headers.

\hypersetup{linkcolor = {black}}
\tableofcontents
\listoffigures
\listoftables
\hypersetup{linkcolor = {blue}}

