% Flags for content selection; 
% They should be before calls to \maketitle, \makedeclaration etc.
\setboolean{showconjoint}{false}
\setbool{useuclbanner}{true}
\uclbannerlocation{images/ucl-banner-a4.eps}

\title{Improving Software Project Health \\Using Machine Learning}
\author{Profir-Petru P\^ar\cb{t}achi}
\department{Department of Computer Science}


\extradeclaration{The work presented in this thesis %
is original work undertaken between September 2016 and July 2020 at %
University College London.\\%
\indent I list the papers that comprise this work in %
\Cref{chapter:introduction:sec:papers}. In the same section, I clearly detail %
my contributions towards each paper. They represent Chapers~\ref{chapter:am}, %
\ref{chapter:flexeme}, and \ref{chapter:posit} respectively.}

\maketitle
\makedeclaration

\begin{conjoint}

The thesis “Improving Project Health using Machine Learning” by Pârțachi
Profir-Petru, the undersigned, is composed of three main papers which were
written in collaboration with others. For each paper, I include the list of
authors herein and detail exactly my contributions. All authors have contributed
significantly to the writing of the papers; therefore, I detail only non-writing
work. Additionally, all work was done under the careful supervision of Earl
Barr, his insight has shaped all my work and our regular meetings were often
used to checkpoint and plan out my research tasks.

\noindent\textbf{Aide-mémoire: Improving a Project’s Collective Memory via Pull
Request-Issue Links} 

\noindent\emph{Authors: Profir-Petru Pârțachi, David R. White, Earl T. Barr}

\noindent\emph{Venue: Under submission to ACM Transactions on Software
Engineering and Methodology (TOSEM)}

\noindent I have determined and mined the dataset of GitHub projects. I have
implemented the proposed PR-Issue linker, performed the exploratory data
analysis and feature engineering as well as wrote the evaluation and result
analysis scripts. The experimental and EDA designs were done in close
collaboration with David White who made sure I perform these correctly and
guided me patiently.

\noindent\textbf{Flexeme: Untangling Commits Using Lexical Flows}

\noindent\emph{Authors: Profir-Petru Pârțachi, Santanu Kumar Dash, Miltose
Allamanis, Earl T. Barr}

\noindent\emph{Venue: Proceedings of 28th ACM Joint European Software
Engineering Conference and Symposium on the Foundations of Software Engineering,
(ESEC/FSE 2020)}

\noindent  I have implemented in full the construction of our new structure,
though the idea of the structure was from Earl Barr, and the details of the
specification were worked in close collaboration. I also implemented the
necessary methods to construct the dataset, reproduced previous work in the area
and implemented our proposed untangling algorithms as well as their evaluation
on the constructed dataset. The RefiNym implementation was provided by Santanu
Dash who helped me integrate it with Flexeme. The original PDG extraction
implementation for C\# code was provided by Miltose Allamanis. Santanu Dash and
I performed the manual evaluations.

\noindent\textbf{POSIT: Simultaneously Tagging Natural and Programming
Languages} 

\noindent\emph{Authors: Profir-Petru Pârțachi, Santanu Kumar Dash, Christoph
Treude, Earl T. Barr}

\noindent\emph{Venue: Proceedings of 42nd International Conference on Software
Engineering (ICSE ’20)}

\noindent I have implemented the preprocessing scripts, the Neural Network that
realises POSIT, the adaptation of the previous State-of-the-art to our problem,
and the necessary evaluation scripts. The Code Comments corpus was provided by
Santanu Dash. The original implementation of TaskNav was provided by Christoph
Treude. The informal proof of context-sensitivity of mixed-text was worked on in
close collaboration with Earl Barr without whom the proof would have not been
finished in a timely manner. Manual evaluation of POSIT was done together with
Santanu Dash, while manual evaluations of TaskNav augmented with POSIT were done
together with Christoph Treude.

\doublesignature{Earl T. Barr}

\end{conjoint}

\begin{abstract} % 300 word limit

In recent years systems that would previously live on different platforms
have been integrated under a single umbrella. The proliferation of GitHub,
which offers pull-requests, issue tracking and version history, and its
integration with other solutions such as Gerrit, or Travis, as well as the
response from competitors has led to leaner and faster development cycles.
This has also reduced the cost of entry and created large, publicly
accessible sources of source-code together with related project artefacts.

This shift in tooling has also facilitated a shift in development paradigms,
developers now prefer a continuous integration/continuous delivery
infrastructure. This has led to more projects adopting a more agile
development process. Developers often forgo tasks that may aid project
health so that they can instead travel light. However, project health
determines project success.  Project health encompasses traceability,
documentation, adherence to coding conventions, in short tasks that allow
for lower maintenance costs and higher accountability.

Simultaneously, this shift has allowed for the proliferation of Natural
Language or Natural Language and Formal Language textual artefacts which are
programmatically accessible. This suggests that approaches from Natural
Language Processing and Machine Learning are now feasible and indeed
desirable. This thesis aims to (semi-)automate tasks for this new paradigm
and its attendant infrastructure by bringing to the foreground the relevant
NLP and ML techniques.

Under this umbrella, I focus on three synergistic tasks from this domain: (1)
improving the issue-pull-request traceability, which can aid existing systems to
automatically curate the issue backlog as pull-requests are merged; (2)
untangling commits in a version history, which can aid the beforementioned
traceability task as well as improve the usability of determining a fault
introducing commit, or cherry-picking via tools such as git bisect; (3)
mixed-text parsing, to allow better API mining and open new avenues for
project-specific code-recommendation tools.	

\end{abstract}

\begin{impactstatement}

In the modern world, software is ubiquitous: from the personal computers most of
us use to the mission critical systems that require fine control. Software
quality is usually a result of a healthy software project, \ie it is not just a
function of the code, rather also of the process producing that code.
Further, projects are not just source-code; indeed, they contain a wealth of
documents written in English or other languages. Focusing strictly on
source-code tells us only half the story.

In this thesis, I propose the notion of project health; a notion that was
previously colloquially understood and that encompasses those processes that
help project succeed and flourish. Under this umbrella, I focus on three tasks
that could impede project success: pull-request-issue linking, commit separation
into atomic patches, and mixed-text segmentation and pre-processing. I focus on
tasks that would break assumptions made by researchers when proposing techniques
in the first two, while the latter presents a new way of handling data enabling
analysis which is aware of algorithmic and natural language information. By
borrowing techniques and methodology from Machine Learning, I propose prototype
systems to resolve these issues.

The work I present in this thesis follows the ethos of helping developers help
themselves and us. We, as researchers, help developers maintain project health
with its inherent benefits: easier onboarding, lower costs of maintenance \etc
This in turn can create better training data for researchers and may enable yet
better automation techniques and research avenues for Big Code.

\end{impactstatement}

\begin{acknowledgements}
    
A PhD is not an easy journey, but one I will look back on fondly. Along the way,
I have had the support of many for whom I wish to dedicate the following
paragraphs in thanks.

I would like to thank my supervisor, Dr Earl T. Barr, for the sage advice and
importantly patience with me especially as I was learning the art of
academic writing. I would also like to thank Earl for bearing with me as I fell
ill throughout the journey and offering support during those times. The
dialectic we had trying to formally prove a hypothesis with the clock ticking to
the deadline will be a fond memory for years despite the stress of the moment
then.

I would also like to thank Dr David R. White for the `Desk Compensation
Meetings' that set me on the path towards rigorous experimental design early in
my PhD. That formed the support beams for my empirical studies throughout my
doctoral work. I would also like to thank David for being there for me when I
needed someone to listen to me and helping navigate the academic landscape with
more confidence.

For the help and patience they showed, I would also like to thank my co-authors
during my PhD research. I would like to thank Dr Santanu Dash, for helping me
grok the rougher sides of Roslyn and enabling the work that followed from that.
I would also like to thank Santanu for helping me pursue an idea that later
became a paper by helping me with initial labelled data that would have been
impossible without his CLANG knowledge. His help and advice on the subtilties of
formal languages helped shape my work.

I extend similar gratitude to Miltose Allamanis and Christoph Treude, who have
made invaluable contributions and were a pleasure to work with as co-authors.
Miltose has helped me get a better start writing Machine Learning code, while
Christoph helped me better understand how to do qualitative and manual
quantitative analysis

I would like to thank my lab mates in CREST and SSE that made the journey more
fun: Leo Jeoffe, David Kelly, Bobby Bruce, DongGyun Han, Matheus Paix\~ao,
Carlos Gavidia, Giovani Guizzo, Vali Tawosi, Rebecca Mousa, Jie Zhang. The
informal chats were invaluable and they all helped me through the journey.
Discussing our work together helped me, and I hope them as well, better
understand how to do proper science, to better understand how and when
statistical techniques should be applied, how to formulate experiments, how to
thoroughly explore hypothesises. I also extend my gratitude to my close friend,
Tudor Haru\cb{t}a, with whom I oft shared my progress and who was always
available to chat when I needed to take my mind off of research.

Last but not by any means least, I would like to thank my family, my dear older
sister for always bearing with me and lending a ear to listen and a pair of eyes
to double check my English. My parents for supporting me through my PhD years
and even donating time on the home PC when I needed extra compute resources on a
tight deadline.

Our research is shaped by those in our network, those we talk to even causally,
thus the sum total of the contributions to this research is beyond any
quantifiable scope; I cannot hope to enumerate everyone who has contributed to
my work in the multitude of ways that I have been supported along on this
journey. To everyone, thank you.

\end{acknowledgements}

\setcounter{tocdepth}{2} 
\hypersetup{linkcolor = {black}}
\tableofcontents
\listoffigures
\listoftables
\hypersetup{linkcolor = {blue}}

