% UCL Thesis LaTeX Template
%  (c) Ian Kirker, 2014
% 
% This is a template/skeleton for PhD/MPhil/MRes theses.
%
% It uses a rather split-up file structure because this tends to
%  work well for large, complex documents.
% We suggest using one file per chapter, but you may wish to use more
%  or fewer separate files than that.
% We've also separated out various bits of configuration into their
%  own files, to keep everything neat.
% Note that the \input command just streams in whatever file you give
%  it, while the \include command adds a page break, and does some
%  extra organisation to make compilation faster. Note that you can't
%  use \include inside an \include-d file.
% We suggest using \input for settings and configuration files that
%  you always want to use, and \include for each section of content.
% If you do that, it also means you can use the \includeonly statement
%  to only compile up the section you're currently interested in.
% You might also want to put figures into their own files to be \input.

% For more information on \input and \include, see:
%  http://tex.stackexchange.com/questions/246/when-should-i-use-input-vs-include


% Formatting and binding rules for theses are here: 
%  https://www.ucl.ac.uk/students/exams-and-assessments/research-assessments/format-bind-and-submit-your-thesis-general-guidance

% This package goes first and foremost, because it checks all 
%  your syntax for mistakes and some old-fashioned LaTeX commands.
% Note that normally you should load your documentclass before 
%  packages, because some packages change behaviour based on
%  your document settings.
% Also, for those confused by the RequirePackage here vs usepackage
%  elsewhere, usepackage cannot be used before the documentclass
%  command, while RequirePackage can. That's the only functional
%  difference as far as I'm aware.
\RequirePackage[l2tabu, orthodox]{nag}

\PassOptionsToPackage{dvipsnames}{xcolor}


% ------ Main document class specification ------
% The draft option here prevents images being inserted,
%  and adds chunky black bars to boxes that are exceeding 
%  the page width (to show that they are).
% The oneside option can optionally be replaced by twoside if
%  you intend to print double-sided. Note that this is
%  *specifically permitted* by the UCL thesis formatting
%  guidelines.
%
% Valid options in terms of type are:
%  phd
%  mres
%  mphil
%\documentclass[12pt,phd,draft,a4paper,oneside]{ucl_thesis}
\documentclass[12pt,phd,a4paper,oneside]{ucl_thesis}

% Package configuration:
%  LaTeX uses "packages" to add extra commands and features.
%  There are quite a few useful ones, so we've put them in a 
%   separate file.
% -------- Packages --------

% This package just gives you a quick way to dump in some sample text.
% You can remove it -- it's just here for the examples.
\usepackage{blindtext}

% This package means empty pages (pages with no text) won't get stuff
%  like chapter names at the top of the page. It's mostly cosmetic.
\usepackage{emptypage}

% The graphicx package adds the \includegraphics command,
%  which is your basic command for adding a picture.
\usepackage{graphicx}

% The float package improves LaTeX's handling of floats,
%  and also adds the option to *force* LaTeX to put the float
%  HERE, with the [H] option to the float environment.
\usepackage{float}

% The amsmath package enhances the various ways of including
%  maths, including adding the align environment for aligned
%  equations.
\usepackage{amsmath}
\usepackage{amssymb}

% Use these two packages together -- they define symbols
%  for e.g. units that you can use in both text and math mode.
\usepackage{gensymb}
\usepackage{textcomp}
% You may also want the units package for making little
%  fractions for unit specifications.
%\usepackage{units}


% The setspace package lets you use 1.5-sized or double line spacing.
\usepackage{setspace}
\setstretch{1.5}

% That just does body text -- if you want to expand *everything*,
%  including footnotes and tables, use this instead:
%\renewcommand{\baselinestretch}{1.5}


% PGFPlots is either a really clunky or really good way to add graphs
%  into your document, depending on your point of view.
% There's waaaaay too much information on using this to cover here,
%  so, you might want to start here:
%   http://pgfplots.sourceforge.net/
%  or here:
%   http://pgfplots.sourceforge.net/pgfplots.pdf
%\usepackage{pgfplots}
%\pgfplotsset{compat=1.3} % <- this fixed axis labels in the version I was using

% PGFPlotsTable can help you make tables a little more easily than
%  usual in LaTeX.
% If you're going to have to paste data in a lot, I'd suggest using it.
%  You might want to start with the manual, here:
%  http://pgfplots.sourceforge.net/pgfplotstable.pdf
%\usepackage{pgfplotstable}

% These settings are also recommended for using with pgfplotstable.
%\pgfplotstableset{
%	% these columns/<colname>/.style={<options>} things define a style
%	% which applies to <colname> only.
%	empty cells with={--}, % replace empty cells with '--'
%	every head row/.style={before row=\toprule,after row=\midrule},
%	every last row/.style={after row=\bottomrule}
%}


% The mhchem package provides chemistry formula typesetting commands
%  e.g. \ce{H2O}
%\usepackage[version=3]{mhchem}

% And the chemfig package gives a weird command for adding Lewis 
%  diagrams, for e.g. organic molecules
%\usepackage{chemfig}

% The linenumbers command from the lineno package adds line numbers
%  alongside your text that can be useful for discussing edits 
%  in drafts.
% Remove or comment out the command for proper versions.
%\usepackage[modulo]{lineno}
% \linenumbers 


% Alternatively, you can use the ifdraft package to let you add
%  commands that will only be used in draft versions
%\usepackage{ifdraft}

% For example, the following adds a watermark if the draft mode is on.
%\ifdraft{
%  \usepackage{draftwatermark}
%  \SetWatermarkText{\shortstack{\textsc{Draft Mode}\\ \strut \\ \strut \\ \strut}}
%  \SetWatermarkScale{0.5}
%  \SetWatermarkAngle{90}
%}


% The multirow package adds the option to make cells span 
%  rows in tables.
\usepackage{multirow}


% Subfig allows you to create figures within figures, to, for example,
%  make a single figure with 4 individually labeled and referenceable
%  sub-figures.
% It's quite fiddly to use, so check the documentation.
%\usepackage{subfig}

% The natbib package allows book-type citations commonly used in
%  longer works, and less commonly in science articles (IME).
% e.g. (Saucer et al., 1993) rather than [1]
% More details are here: http://merkel.zoneo.net/Latex/natbib.php
%\usepackage{natbib}

% The bibentry package (along with the \nobibliography* command)
%  allows putting full reference lines inline.
%  See: 
%   http://tex.stackexchange.com/questions/2905/how-can-i-list-references-from-bibtex-file-in-line-with-commentary
\usepackage{bibentry} 

% The isorot package allows you to put things sideways 
%  (or indeed, at any angle) on a page.
% This can be useful for wide graphs or other figures.
%\usepackage{isorot}

% The caption package adds more options for caption formatting.
% This set-up makes hanging labels, makes the caption text smaller
%  than the body text, and makes the label bold.
% Highly recommended.
\usepackage[format=hang,font=small,labelfont=bf]{caption}

% If you're getting into defining your own commands, you might want
%  to check out the etoolbox package -- it defines a few commands
%  that can make it easier to make commands robust.
\usepackage{etoolbox}

% Needed for Author name, uses the correct comma below S and T for Romanian
\usepackage{combelow}  

% Add the papers as is, compiled on their own
\usepackage[enable-survey]{pdfpages}

% Subfigures, used in Aide-mémoire
\usepackage[caption=false,font=footnotesize,labelfont=sf,textfont=sf]{subfig}

% For code listing environments
\usepackage{listingsutf8}

% To import and not break relative links
\usepackage{import}

% Used in Global Macros for aesthetic et al. etc.
\usepackage{xspace}

% Used in macros for symbols and Profir's Name
\usepackage[utf8x]{inputenc} 

% for definition of new column type in AM
\usepackage{siunitx} 

% for new lines in table cells
\usepackage{makecell}

% aesthetically pleasing tables
\usepackage{booktabs}

% Allow Definition environments
\usepackage{amsthm}
\newtheorem{definition}{Definition}

% Switch to natbib to allow for citep and citet as used in papers
\usepackage{natbib}

% Some of the examples need tcolorbox
\usepackage{tcolorbox}

%allow valign of graphics
\usepackage[export]{adjustbox}

% For Flexeme's diagram
\usepackage{tikz}
\usetikzlibrary{calc,positioning}

% Used by some examples in Flexeme for specific Unicode
\usepackage[T1]{fontenc}

% Used for tables in Flexeme
\usepackage{tabularx}

% Used for the todo macro
\usepackage{ifthen}

% Used for better formatting of AM equations
\usepackage{multicol}


\newcommand{\cf}{\hbox{\emph{cf.}}\xspace}
\newcommand{\deletia}{\ldots [deletia] \ldots}
\newcommand{\etal}{\hbox{\emph{et al.}}\xspace}
\newcommand{\eg}{\hbox{\emph{e.g.}}\xspace}
\newcommand{\ie}{\hbox{\emph{i.e.}}\xspace}
\newcommand{\scil}{\hbox{\emph{sc.}}\xspace} %scilicet: it is permitted to know
\newcommand{\st}{\hbox{\emph{s.t.}}\xspace}
\newcommand{\wrt}{\hbox{\emph{w.r.t.}}\xspace}
\newcommand{\etc}{\hbox{\emph{etc.}}\xspace}
\newcommand{\viz}{\hbox{\emph{viz.}}\xspace} %videlicet: it is permitted to see
\newcommand{\infinity}{\infty}
\newcommand{\green}{PineGreen}
% Gotta have todos
\newcommand{\todo}[1]{\textcolor{red}{TODO: #1}}

% Missing unicode chars, other brokenness in ucs/inputenc {{{1
\DeclareUnicodeCharacter{183}{\cdot}						% ·
\DeclareUnicodeCharacter{931}{\ensuremath\Sigma}			% Σ
\DeclareUnicodeCharacter{9001}{\ensuremath\langle}			% 〈
\DeclareUnicodeCharacter{9002}{\ensuremath\rangle}			% 〉
\DeclareUnicodeCharacter{9608}{\ensuremath\blacksquare}		% █
\DeclareUnicodeCharacter{1013}{\in}							% ϵ
\DeclareUnicodeCharacter{8213}{---}							% ―

\PrerenderUnicode{ț}

\theoremstyle{definition}
\newtheorem{defn}{Definition}[section]
\newtheorem{conj}{Conjecture}

% Sets up links within your document, for e.g. contents page entries
%  and references, and also PDF metadata.
% You should edit this!
%%
%% This file uses the hyperref package to make your thesis have metadata embedded in the PDF, 
%%  and also adds links to be able to click on references and contents page entries to go to 
%%  the pages.
%%

% Some hacks are necessary to make bibentry and hyperref play nicely.
% See: http://tex.stackexchange.com/questions/65348/clash-between-bibentry-and-hyperref-with-bibstyle-elsart-harv
\usepackage{bibentry}
\makeatletter\let\saved@bibitem\@bibitem\makeatother
\usepackage[pdftex,hidelinks]{hyperref}
% Needed for sane references, used in papers, must be after hyperref
\usepackage[nameinlink]{cleveref}
% Again, after hyperlink by requirement, used in Flexeme
\usepackage[noend]{algorithmic} % may not be installed
\usepackage{algorithm}
\makeatletter\let\@bibitem\saved@bibitem\makeatother
\makeatletter
\AtBeginDocument{
    \hypersetup{
        pdfsubject={Thesis Subject},
        pdfkeywords={Thesis Keywords},
        pdfauthor={Author},
        pdftitle={Title},
    }
}
\makeatother
    


% And then some settings in separate files.
% These settings are from:
%  http://mintaka.sdsu.edu/GF/bibliog/latex/floats.html

% They give LaTeX more options on where to put your figures, and may
%  mean that fewer of your figures end up at the tops of pages far
%  away from the thing they're related to.

% Alters some LaTeX defaults for better treatment of figures:
% See p.105 of "TeX Unbound" for suggested values.
% See pp. 199-200 of Lamport's "LaTeX" book for details.

%   General parameters, for ALL pages:
\renewcommand{\topfraction}{0.9}	% max fraction of floats at top
\renewcommand{\bottomfraction}{0.8}	% max fraction of floats at bottom

%   Parameters for TEXT pages (not float pages):
\setcounter{topnumber}{2}
\setcounter{bottomnumber}{2}
\setcounter{totalnumber}{4}     % 2 may work better
\setcounter{dbltopnumber}{2}    % for 2-column pages
\renewcommand{\dbltopfraction}{0.9}	% fit big float above 2-col. text
\renewcommand{\textfraction}{0.07}	% allow minimal text w. figs

%   Parameters for FLOAT pages (not text pages):
\renewcommand{\floatpagefraction}{0.7}	% require fuller float pages
% N.B.: floatpagefraction MUST be less than topfraction !!
\renewcommand{\dblfloatpagefraction}{0.7}	% require fuller float pages

% remember to use [htp] or [htpb] for placement,
% e.g. 
%  \begin{figure}[htp]
%   ...
%  \end{figure} % For things like figures and tables
\bibliographystyle{plainnat}   % For bibliographies

% These control how many number sections your subsections will take
%    e.g. Section 2.3.1.5.6.3
%  and how many of those will get put into the contents pages.
\setcounter{secnumdepth}{3}
\setcounter{tocdepth}{3}


\begin{document}

\nobibliography*
% ^-- This is a dumb trick that works with the bibentry package to let
%  you put bibliography entries whereever you like.
% I used this to put references to papers a chapter's work was 
%  published in at the end of that chapter.
% For more information, see: http://stefaanlippens.net/bibentry

% If you haven't finished making your full BibTex file yet, you
%  might find this useful -- it'll just replace all your
%  citations with little superscript notes.
% Uncomment to use.
% \renewcommand{\cite}[1]{\emph{\textsuperscript{[#1]}}}

% At last, content! Remember filenames are case-sensitive and 
%  *must not* include spaces.
% Flags for content selection; 
% They should be before calls to \maketitle, \makedeclaration etc.
\setboolean{showconjoint}{false}
\setbool{useuclbanner}{true}
\uclbannerlocation{images/ucl-banner-a4.eps}

\title{Improving Software Project Health \\Using Machine Learning}
\author{Profir-Petru P\^ar\cb{t}achi}
\department{Department of Computer Science}


\extradeclaration{The work presented in this thesis %
is original work undertaken between September 2016 and July 2020 at %
University College London.\\%
\indent I list the papers that comprise this work in %
\Cref{chapter:introduction:sec:papers}. In the same section, I clearly detail %
my contributions towards each paper. They represent Chapers~\ref{chapter:am}, %
\ref{chapter:flexeme}, and \ref{chapter:posit} respectively.}

\maketitle
\makedeclaration

\begin{conjoint}
The thesis “Improving Project Health using Machine Learning” by Pârțachi
Profir-Petru, the undersigned, is composed of three main papers which were
written in collaboration with others. For each paper, I include the list of
authors herein and detail exactly my contributions. All authors have contributed
significantly to the writing of the papers; therefore, I detail only non-writing
work. Additionally, all work was done under the careful supervision of Earl
Barr, his insight has shaped all my work and our regular meetings were often
used to checkpoint and plan out my research tasks.

\begin{itemize}[leftmargin=*]
    \item[] \textbf{Aide-mémoire: Improving a Project’s Collective Memory via
Pull Request-Issue Links} 
    
    \noindent\emph{Authors: Profir-Petru Pârțachi, David R. White, Earl T. Barr}
    
    \noindent\emph{Venue: Submitted to ACM Transactions on Software Engineering
    and Methodology (TOSEM)}

    \noindent I have determined and mined the dataset of GitHub projects. I have
    implemented the proposed PR-Issue linker, performed the exploratory data
    analysis and feature engineering as well as wrote the evaluation and result
    analysis scripts. The experimental and EDA designs were done in close
    collaboration with David White who guided me patiently.

    \item[] \noindent\textbf{Flexeme: Untangling Commits Using Lexical Flows}
    
    \noindent\emph{Authors: Profir-Petru Pârțachi, Santanu Kumar Dash, Miltose
    Allamanis, Earl T. Barr}
    
    \noindent\emph{Venue: Proceedings of 28th ACM Joint European Software
    Engineering Conference and Symposium on the Foundations of Software
    Engineering, (ESEC/FSE 2020)}

    \noindent  I have implemented in full the construction of our new structure,
    though the idea of the structure was from Earl Barr, and the details of the
    specification were worked in close collaboration. I also implemented the
    necessary methods to construct the dataset, reproduced previous work in the
    area and implemented our proposed untangling algorithms as well as their
    evaluation on the constructed dataset. The RefiNym implementation was
    provided by Santanu Dash who helped me integrate it with Flexeme. The
    original PDG extraction implementation for C\# code was provided by Miltose
    Allamanis. Santanu Dash and I performed the manual evaluations.
    
    %%% !Remember to check if this is still needed if you edit text above it!
    \pagebreak
    
    \item[] \noindent\textbf{POSIT: Simultaneously Tagging Natural and
    Programming Languages} 
    
    \noindent\emph{Authors: Profir-Petru Pârțachi, Santanu Kumar Dash, Christoph
    Treude, Earl T. Barr}
    
    \noindent\emph{Venue: Proceedings of 42nd International Conference on
    Software Engineering (ICSE ’20)}

    \noindent I have implemented the preprocessing scripts, the Neural Network
    that realises POSIT, the adaptation of the previous State-of-the-art to our
    problem, and the necessary evaluation scripts. The Code Comments corpus was
    provided by Santanu Dash. The original implementation of TaskNav was
    provided by Christoph Treude. The informal proof of context-sensitivity of
    mixed-text was worked on in close collaboration with Earl Barr without whom
    the proof would have not been finished in a timely manner. Manual evaluation
    of POSIT was done together with Santanu Dash, while manual evaluations of
    TaskNav augmented with POSIT were done together with Christoph Treude.

\end{itemize}

\doublesignature{Earl T. Barr}
\end{conjoint}

\begin{abstract} % 300 word limit
In recent years, systems that would previously live on different platforms have
been integrated under a single umbrella. The increased use of GitHub, which
offers pull-requests, issue tracking and version history, and its integration
with other solutions such as Gerrit, or Travis, as well as the response from
competitors, created development environments that favour agile methodologies by
increasingly automating non-coding tasks: automated build systems, automated
issue triaging \etc In essence, source-code hosting platforms shifted to
continuous integration/continuous delivery(CI/CD) as a service. This facilitated
a shift in development paradigms, adherents of agile methodology can now adopt a
CI/CD infrastructure more easily. This has also created large, publicly
accessible sources of source-code together with related project artefacts:
GHTorrent and similar datasets now offer programmatic access to the whole of
GitHub.

Project health encompasses traceability, documentation, adherence to coding
conventions, tasks that reduce maintenance costs and increase accountability,
but may not directly impact features.  Over focus on health can slow velocity
(new feature delivery) so the Agile Manifesto suggests developers should travel
light --- forgo tasks focused on a project health in favour of higher feature
velocity. Obviously, injudiciously following this suggestion can undermine a
project's chances for success.

Simultaneously, this shift to CI/CD has allowed the proliferation of Natural
Language or Natural Language and Formal Language textual artefacts that are
programmatically accessible: GitHub and their competitors allow API access to
their infrastructure to enable the creation of CI/CD bots. This suggests that
approaches from Natural Language Processing and Machine Learning are now
feasible and indeed desirable. This thesis aims to (semi-)automate tasks for
this new paradigm and its attendant infrastructure by bringing to the foreground
the relevant NLP and ML techniques.

Under this umbrella, I focus on three synergistic tasks from this domain: (1)
improving the issue-pull-request traceability, which can aid existing systems to
automatically curate the issue backlog as pull-requests are merged; (2)
untangling commits in a version history, which can aid the beforementioned
traceability task as well as improve the usability of determining a fault
introducing commit, or cherry-picking via tools such as git bisect; (3)
mixed-text parsing, to allow better API mining and open new avenues for
project-specific code-recommendation tools.
\end{abstract}

\begin{impactstatement}
In the modern world, software is ubiquitous: from the personal computers most of
us use to the mission critical systems that require fine control. Software
quality is usually a result of a healthy software project, \ie it is not just a
function of the code, rather also of the process producing that code.
Further, projects are not just source-code; indeed, they contain a wealth of
documents written in English or other languages. Focusing strictly on
source-code tells us only half the story.

In this thesis, I propose the notion of project health; a notion that was
previously colloquially understood and that encompasses those processes that
help project succeed and flourish. Under this umbrella, I focus on three tasks
that could impede project success: pull-request-issue linking, commit separation
into atomic patches, and mixed-text segmentation and pre-processing. I focus on
tasks that would break assumptions made by researchers when proposing techniques
in the first two, while the latter presents a new way of handling data enabling
analysis which is aware of algorithmic and natural language information. By
borrowing techniques and methodology from Machine Learning, I propose prototype
systems to resolve these issues.

The work I present in this thesis follows the ethos of helping developers help
themselves and us. We, as researchers, help developers maintain project health
with its inherent benefits: easier onboarding, lower costs of maintenance \etc
This in turn can create better training data for researchers and may enable yet
better automation techniques and research avenues for Big Code. Further, all
tooling created during this thesis is Open Sourced and made available under the
MIT license for developer or researchers to use directly.
\end{impactstatement}

\begin{acknowledgements}   
A PhD is not an easy journey, but one I will look back on fondly. Along the way,
I have had the support of many for whom I wish to dedicate the following
paragraphs in thanks.

I would like to thank my supervisor, Dr Earl T. Barr, for the sage advice and
importantly patience with me especially as I was learning the art of
academic writing. I would also like to thank Earl for bearing with me as I fell
ill throughout the journey and offering support during those times. The
dialectic we had trying to formally prove a hypothesis with the clock ticking to
the deadline will be a fond memory for years despite the stress of the moment
then.

I would also like to thank Dr David R. White for the `Desk Compensation
Meetings' that set me on the path towards rigorous experimental design early in
my PhD. That formed the support beams for my empirical studies throughout my
doctoral work. I would also like to thank David for being there for me when I
needed someone to listen to me and helping navigate the academic landscape with
more confidence.

For the help and patience they showed, I would also like to thank my co-authors
during my PhD research. I would like to thank Dr Santanu Dash, for helping me
grok the rougher sides of Roslyn and enabling the work that followed from that.
I would also like to thank Santanu for helping me pursue an idea that later
became a paper by helping me with initial labelled data that would have been
impossible without his CLANG knowledge. His help and advice on the subtilties of
formal languages helped shape my work.

I extend similar gratitude to Miltose Allamanis and Christoph Treude, who have
made invaluable contributions and were a pleasure to work with as co-authors.
Miltose has helped me get a better start writing Machine Learning code, while
Christoph helped me better understand how to do qualitative and manual
quantitative analysis

I would like to thank my lab mates in CREST and SSE that made the journey more
fun: Leo Jeoffe, David Kelly, Bobby Bruce, DongGyun Han, Matheus Paix\~ao,
Carlos Gavidia, Giovani Guizzo, Vali Tawosi, Rebecca Mousa, Jie Zhang. The
informal chats were invaluable and they all helped me through the journey.
Discussing our work together helped me, and I hope them as well, better
understand how to do proper science, to better understand how and when
statistical techniques should be applied, how to formulate experiments, how to
thoroughly explore hypothesises. I also extend my gratitude to my close friend,
Tudor Haru\cb{t}a, with whom I oft shared my progress and who was always
available to chat when I needed to take my mind off of research.

Last but not by any means least, I would like to thank my family, my dear older
sister for always bearing with me and lending a ear to listen and a pair of eyes
to double check my English. My parents for supporting me through my PhD years
and even donating time on the home PC when I needed extra compute resources on a
tight deadline.

Our research is shaped by those in our network, those we talk to even causally,
thus the sum total of the contributions to this research is beyond any
quantifiable scope; I cannot hope to enumerate everyone who has contributed to
my work in the multitude of ways that I have been supported along on this
journey. To everyone, thank you.
\end{acknowledgements}

\setcounter{tocdepth}{2} 
\hypersetup{linkcolor = {black}}
\tableofcontents
\listoffigures
\listoftables
\hypersetup{linkcolor = {blue}}


\chapter{Introduction}
\label{chapter:introduction}

The software development process involves a multitude of aspects: the
specification and design, the architecture of the system, the code writing
itself, code version histories, issue tracking, code review. In an opensource
project external collaboration and their integration are also a part of it. Last
but not least, the developers themselves and the project culture further define
the development process. Some projects are vibrant; some moribund; others
implode after a few commits. \emph{Project health} determines project success.
Project health encompasses: technical, traceability, and social aspects of a
project. It is defined by the quality of code, documentation, version histories,
issue tracking, specifications and requirements. It is further defined by the
quality of the interlinking of these aspects. Finally, but not less important,
it is defined by the culture of the project, be that explicitly codified in a
file/manifesto or implicit in the interactions of developers. In this work, I
focus only on the first two aspects of project health: technical and
traceability. It is of no surprise that the artefacts produced by these
processes span a profusion of formats and both natural as well as programming
languages, even intermixed.

In recent years systems, that would previously live on different platforms have
been integrated under a single umbrella, thus the different aspects that can
impact project health can be centrally observed. The proliferation of GitHub,
which offers pull-requests, issue tracking and version history, and its
integration with other solutions such as Gerrit (for Code Reviews), Travis (for
continuous integration and delivery), as well as the response from competitors,
such as GitLab and Atlassian who offer equivalent offerings has led to leaner
and faster development cycles: GitHub advertises their CI offerings with a
promise to reduce time spent merging commits or debugging and increase time
spend writing code~\cite{GitHubCI}. This has also reduced the cost of entry and
created large, publicly accessible sources of source-code together with related
project artefacts, such as GHTorrent~\cite{GHTorrent},
CodeSearchNet~\cite{Husain2019} and similar datasets. Within this context,
developers can benefit from the automation of tasks, especially if they are oft
forgotten or ignored in favour of traveling light. 

This shift in tooling has also facilitated a shift in development paradigms,
developers now prefer a continuous integration/continuous delivery
infrastructure. This has led to more projects adopting a more agile development
process. Cleland-Huang~\etal~\cite{Cleland-Huang2014} and St{\aa}hl
\etal~\cite{Stahl2017} observe that, for traceability to be adopted, within an
agile framework, it must remain within the traveling light paradigm.
Simultaneously, the shift to continuous improvement/continuous development has
allowed for the proliferation of Natural Language or Natural Language and Formal
Language textual artefacts which are programmatically accessible. This suggests
that approaches from Natural Language Processing and Machine Learning are now
feasible and indeed desirable. This thesis aims to (semi-)automate tasks for
this new paradigm and its attendant infrastructure by bringing to the foreground
the relevant NLP and ML techniques.

\section{Problem Statements}
\label{chapter:introduction:sec:problem_statement}

Under Improving Project Health, I focus on three synergistic tasks: (1)
improving the issue-pull-request traceability, which can aid existing systems to
automatically curate the issue backlog as pull-requests are merged; (2)
untangling commits in a version history, which can aid the beforementioned
traceability task as well as improve the usability of determining a fault
introducing commit, or cherry-picking via tools such as git bisect; (3)
mixed-text parsing, to allow better API mining and open new avenues for
project-specific code-recommendation tools. 

Software development is a community task. The programming task itself is a
component of a much wider process. It is, however, a task that is viewed
exclusively when considered by developers at the cost of other administrative
tasks. This manifests itself as, for example, in the case of traceability, which
companies recognise as important, by organisations reaching an insufficient
level of traceability~\cite{mader2009motivation}. These administrative tasks
come with the promise of facilitating internal follow-ups, evaluation or
improvement of the development process~\cite{domges1998adapting}. Agile
practice, and especially Continuous Integration, by suggesting a lean approach
to documentation, reduces the direct applicability of existing approaches,
though work has been done to adapt to the new paradigm:
Ståhl~\etal~\cite{Stahl2017} provide the Eiffel framework, which allows
integrating different sources of traceability information in a light-weight
manner. Indeed, the conflict arises between the traditional assumptions of
Traceability and the agile team's tendencies of traveling quite
light~\cite{Cleland-Huang2014}. Indeed, \emph{traveling light} means forgoing
documentation, user stories not being punted to new sprints, or, more generally,
administrative tasks that distract from a developer coding being ignored. If
combined with backlog refinement~\cite{BacklogRefinement}, traveling light can
manifest itself as a form of project amnesia. This is a form of technical debt,
and will degrade project health in the long-run.

To help developers, tool smiths themselves require supporting artefacts and
utilities. One of these is access to high quality data that can be used both to
train a human intuition for a task as well as statistical approaches to enable
developers from less well maintained projects improve their situation. The
proliferation of web-sites such as GitHub, BitBucket/JIRA, GitLab, and others
that enable developers to have commits, issues, pull-request, code-reviews and
other development processes under one umbrella also opens a potential resource
to researchers and tool smiths that wish to study the properties of the
inter-relationships of such artefacts. In this context, traceability can help
developers maintain the collective memory of projects. When using such
resources, the promise of automatic triaging of issues as commits resolving them
are accepted into the mainline incentivises developers to maintain
code-review/issue to commit links. I have, however, observed a lack of such
links as far as GitHub is concerned, despite an encouragement of contributors to
do so in the contribution guidelines (\Cref{am:sec:eval:prelim}). Further,
developers only partially recording such link introduces bias in the datasets
researchers use.

\begin{tcolorbox}[title=Pull-request-Issue Traceability]
    Given a project, developers should receive traceability link suggestions
    when pertinent, \ie when they are already performing a task within which a
    link can be recorded. For a pull-request-issue traceability, suggestions are
    to be made at issue triage and PR submission.
\end{tcolorbox}

Additionally, links may not be clear if the fixes are presented as unfocused
patches, hiding bug fixing beyond many lines of refactoring.
Herzig~\etal~\cite{Herzig2016} (Herzig and Zeller~\cite{Herzig2013}) show that
tangled commits introduce bias in defect prediction. In our work, I aim to
provide a diff time solution for commit untangling to help developers reduce
this bias. The promise to developers is enabling tooling such as git bisect.

\begin{tcolorbox}[title=Commit Untangling]
    Given a commit, the patch should be decomposed into smaller patches such
    that each patch tackles a single task, \ie would be linked to a single issue
    or implements a single stakeholder concern. On atomic commits, this process
    should be the identity function.
\end{tcolorbox}

Despite the international nature of platforms such as GitHub, a significant
portion of software development information is in English. This leads to
projects in other natural languages to adopt English terms within their
processes as observed by Treude~\etal~\cite{Treude2015portuguese}. Further, such
natural language can mix any natural languages it may use with formal languages.
This same phenomenon is observable on mailing lists, such as the Linux Kernel
Mailing List, and developer fora, such as StackOverflow.  Further, despite
access to mark-up to help developers demarcate code from English, in practice,
even on StackOverflow submitters do forget to do so. This reduces the applicable
research tools which use such formatting cues.

\begin{tcolorbox}[title=Mixed-text Parsing]
    Given an artefact that freely mixes natural languages and formal languages,
    separate the languages. Further, for each token, provide the tag in the
    language from which the token is taken. In natural languages, this tags are
    part-of-speech tags; in formal languages, these tags are the AST tags of the
    parents of the terminals.
\end{tcolorbox}

\noindent The tasks range from more developer focused to more researcher
focused; however, all tasks aim to always provide a benefit to developers,
directly or indirectly.

Overall, the focus of this research focuses on the creation of tools and
meta-tools that (semi-)automate development tasks to enable project to remain
healthy. PR-issue traceability focuses directly on improving the development
process by reducing the administrative burden on the developers, while commit
untangling and mixed-text parsing focus on enabling a wider range of tools to be
applicable. The goal as a whole is an improvement of the development process by
allowing more time to be dedicated to the programming task rather than tasks
required to maintain the project health.

\section{Contributions}
\label{chapter:introduction:sec:contrib}

Under the three synergistic tasks, I have found the following ways of improving
the state-of-the-art, which can benefit both practitioners as well as
researchers; This thesis:

\begin{enumerate}
    \item Provides an online pull-request traceability link inference algorithm
    realised as Aide-mémoire. It solves an online, modern variant of the
    ``Missing Links'' problem proposed by Bachman~\etal~\cite{MissingLinks}. It
    departs from using commits in favour of pull-requests. Aide-mémoire suggests
    links when developers are already handling a PR or an issue.

    \item Forwards the state-of-the-art in commit untangling both in accuracy
    and in speed and thus enables the creation of better statistical tooling for
    developers indirectly, while directly allowing them to improve the health of
    their version histories. I realise this as Flexeme. Flexeme offers precise
    untangling suggestions within 10 seconds; fast enough to be viable in the
    code-review process.

    \item Formulates the mixed-text parsing problem, posing it as a simultaneous
    segmentation and tagging task. It proposes a solution, POSIT, which borrows
    from the Natural Language Processing and Linguistics literature that
    concerns itself with code-switching. It maps the borrowed results to the
    Software Engineering context and allows extending methods to languages other
    than English or resources that freely mix natural and formal languages.
\end{enumerate}

The solutions I propose can help developers directly, by improving the state of
the project artefacts, or indirectly by helping tool smiths. Further, these
solutions interact synergistically, the Flexeme and POSIT improve the quality of
the training data for Aide-mémoire, while also improving the general health of
developer version histories (Flexeme) or developer issue trackers and fora
(POSIT).

\section{List of Papers}
\label{chapter:introduction:sec:papers}

A long research project, like a PhD, is a close collaboration where it is
difficult, often impossible, to separate each collaborator's contributions. What
is clear is who took the lead on which aspects and tasks. The reader will also
notice a mixed use of `I' and `we', I continue to use `we' in the paper chapters
when referring to work done in close collaboration with my co-authors to
acknowledge their contribution. Further, the papers are presented as published
or submitted for review spare figures and tables that required editing to fit
into the thesis format. I now enumerate the tasks I lead in each of the papers
in this thesis.

\begin{itemize}[leftmargin=*]
    \item[]\noindent\textbf{1. Aide-mémoire: Improving a Project’s Collective Memory via Pull
Request-Issue Links} 

    \noindent\emph{Authors: Profir-Petru Pârțachi, David R. White, Earl T. Barr}

    \noindent\emph{Venue: Submitted to ACM Transactions on Software Engineering
    and Methodology (TOSEM)}

    \noindent I determined and mined the dataset of GitHub projects. I
    implemented the proposed PR-Issue linker, performed the exploratory data
    analysis and feature engineering as well as wrote the evaluation and result
    analysis scripts. The experimental and EDA designs were done in close
    collaboration with David White who guided me patiently.

    \item[]\noindent\textbf{2. Flexeme: Untangling Commits Using Lexical Flows}

    \noindent\emph{Authors: Profir-Petru Pârțachi, Santanu Kumar Dash, Miltos
    Allamanis, Earl T. Barr}

    \noindent\emph{Venue: Proceedings of 28th ACM Joint European Software
    Engineering Conference and Symposium on the Foundations of Software
    Engineering, (ESEC/FSE 2020)}

    \noindent I implemented in full the construction of our new structure,
    though the idea of the structure was from Earl Barr, and the details of its
    specification were worked out in close collaboration with him. I also
    implemented the necessary methods to construct the dataset, reproduced
    previous work in the area and implemented our proposed untangling algorithms
    as well as their evaluation on the constructed dataset. The RefiNym
    implementation was provided by Santanu Dash who helped me integrate it with
    Flexeme. The original PDG extraction implementation for C\# code was
    provided by Miltos Allamanis. Santanu Dash and I performed the manual
    evaluations.

    \item[]\noindent\textbf{3. POSIT: Simultaneously Tagging Natural and Programming
    Languages} 

    \noindent\emph{Authors: Profir-Petru Pârțachi, Santanu Kumar Dash, Christoph
    Treude, Earl T. Barr}

    \noindent\emph{Venue: Proceedings of 42nd International Conference on
    Software Engineering (ICSE ’20)}

    \noindent I implemented the preprocessing scripts, the Neural Network
    that realises POSIT, the adaptation of the previous state-of-the-art to our
    problem, and the necessary evaluation scripts. The Code Comments corpus was
    provided by Santanu Dash. The original implementation of TaskNav was
    provided by Christoph Treude. The informal proof of context-sensitivity of
    mixed-text was worked on in close collaboration with Earl Barr without whom
    the proof would have not been finished in a timely manner. The manual
    evaluation of POSIT was done together with Santanu Dash, while the manual
    evaluations of TaskNav augmented with POSIT were done together with
    Christoph Treude.
\end{itemize}

\section{Thesis Organisation}
\label{chapter:introduction:sec:organisation}

The remainder of the thesis is organised as follows.

\Cref{chapter:literature} first presents literature on the areas encompassed by
project health. It then focuses on the areas touched by the proposed tasks in
the order they are presented in the thesis. It first discusses traceability and
the ``Missing Links'' problem~\cite{MissingLinks}, it then focuses on the issues
caused by tangled commits and approaches to the commit untangling, it concludes
with an exploration of existing approaches to mixed-text: how do researchers
tackle code when applying natural language techniques and how the natural
language processing community handles the mixing of multiple natural languages.

\Cref{chapter:am} presents Aide-mémoire, a solution to online pull-request-issue
linking problem. It extends the ``Missing Links'' problem due to
Bachmann~\etal~\cite{MissingLinks} to pull-requests and proposes a solution that
can live along side the code-review process. Aide-mémoire is evaluated on an
extensive corpus of GitHub projects.

\Cref{chapter:flexeme} presents Flexeme, a solution to the commit untangling
problem. It shows that Flexeme improves the state-of-the-art while being fast
enough to be viable at during code-review. Flexeme is evaluated on an artificial
corpus that is rigorously constructed to mimic tangled commits
Herzig~\etal~\cite{Herzig2016} have observed developers make.

\Cref{chapter:posit} presents POSIT, where the mixed-text parsing problem is
introduced together with our proposed solution: POSIT. In it, we argue that the
mixed-text parsing problem is context-sensitive in the general case and hence
propose a distributional semantics approach, realised as a neural network, for
the problem. We show that POSIT can both directly improve software fora by
suggesting mixed code annotations, as well as aid other research tools: we show
POSIT to improve TaskNav's~\cite{Treude:2015:TTN:2819009.2819128} recall.

\Cref{chapter:conclusions} concludes the thesis and provides directions for
future work.
\chapter{Aide-mémoire: Improving a Project’s Collective Memory via Pull Request--Issue Links}
\label{chapter:am}

\paragraph{Paper Authors}\\
Profir-Petru~P\^ar\c{t}achi, Department of Computer Science, University College London, United Kingdom\\
David~R.~White, University of Sheffield, United Kingdom\\
Earl~T.~Barr, Department of Computer Science, University College London, United Kingdom

\subimport{../am-tex/}{../am-tex/paper_macros}

\paragraph{Abstract}
\subimport{../am-tex/}{../am-tex/abstract}

\subimport{../am-tex/}{../am-tex/introduction}
\subimport{../am-tex/}{../am-tex/example}
\subimport{../am-tex/}{../am-tex/implementation}
\subimport{../am-tex/}{../am-tex/feature_space}
\subimport{../am-tex/}{../am-tex/experiment}
\subimport{../am-tex/}{../am-tex/evaluation}
\subimport{../am-tex/}{../am-tex/related_work}
\subimport{../am-tex/}{../am-tex/conclusion}

\chapter{Flexeme: Untangling Commits Using Lexical Flows}
\label{chapter:flexeme}

% Copied here to avoid issues with the paper latex
\newcommand{\deltaPDG}{$\delta$-PDG\xspace}
\newcommand{\deltaPDGnCV}{$\delta$-PDG+CV\xspace}
\newcommand{\deltaPDGs}{$\delta$-PDGs\xspace}
\newcommand{\deltaPDGN}{$\delta$-NFG\xspace}
\newcommand{\deltaPDGNs}{$\delta$-NFGs\xspace}
\renewcommand{\algorithmicrequire}{\textbf{Input:}}
\renewcommand{\algorithmicensure}{\textbf{Output:}}


\newcommand{\ourapproach}{\textsc{flexeme}\xspace}
\newcommand{\Ourapproach}{\textsc{Flexeme}\xspace}
\renewcommand{\ourtool}{\textsc{heddle}\xspace}
\renewcommand{\Ourtool}{\textsc{Heddle}\xspace}
\newcommand{\projURL}{\url{https://github.com/PPPI/Flexeme}\xspace}

% I've moved our results here so they can be easily updated consistently...
\newcommand{\Ouraccuracy}{$0.81$\xspace}
\newcommand{\Ouraccuracyhigh}{$0.84$\xspace}
\newcommand{\OuraccuracyhighProject}{Nancy\xspace}
\newcommand{\Ourspeedup}{$32$\xspace}
\newcommand{\OurslowdownBarnett}{$9$\xspace}
\newcommand{\Ourabsolutespeedup}{$3'10''$\xspace}
\newcommand{\Ourruntime}{ten seconds\xspace}
\newcommand{\OurtoolsAtomicAccuracy}{$0.63$\xspace}
\newcommand{\OurtoolsAtomicLabelAccuracy}{$0.93$\xspace}

% Comparisons or other tools
\newcommand{\OurtoolVsHerzigAccuracy}{$0.14$\xspace}
\newcommand{\BarnettAccuracy}{$0.13$\xspace}
\newcommand{\HerzigConceptAccuracyDrop}{$0.07$\xspace}

% Scalability Results
\newcommand{\OurtoolsScalabilityRegression}{$t = 0.3371 - 0.0041 n + 0.0015 n^2$, $R^2=1.00$\xspace}
\newcommand{\HerzigScalabilityRegression}{$t = 0.8794 - 0.0528 n + 0.0019 n^2$, $R^2=0.99$\xspace}
\newcommand{\SpeedGainAtScale}{$68$ seconds\xspace}

% p-values
% Cross-tool
\newcommand{\HerzigDeltaOurtoolAccuracy}{$p<0.001$\xspace}
\newcommand{\HerzigDeltaOurtoolTime}{$p=0.51$\xspace}
% Cross-concepts
\newcommand{\BarnettConceptsAccuracy}{$p < 0.01$\xspace}
\newcommand{\HerzigConceptsAccuracy}{$p=0.28$\xspace}
\newcommand{\OurtoolConceptsAccuracy}{$p=0.76$\xspace}

\newcommand{\annequiv}{$\equiv_{a}$\xspace}
\newcommand{\nodeequiv}{$\equiv_{n}$\xspace}

\newcommand{\bra}{\langle}
\newcommand{\ket}{\rangle}

% ------------------------------------------------------- [ Colour Definitions ]
\definecolor{cb-green-sea}  {RGB}{  0, 146, 146}
\definecolor{cb-purple}     {RGB}{ 73,   0, 146}
\definecolor{cb-blue-sky}   {RGB}{109, 182, 255}
\definecolor{cb-burgundy}   {RGB}{146,   0,   0}

% Define diff language for lstlisting
\lstdefinelanguage{diff}{
  morecomment=[f][\color{cb-blue-sky}]{@@}, % group identifier
  morecomment=[f][\color{cb-burgundy}]-,    % deleted lines 
  morecomment=[f][\color{cb-green-sea}]+,   % added lines
  morecomment=[f][\color{cb-purple}]{---},  % Diff header lines (must appear after +,-)
  morecomment=[f][\color{magenta}]{+++},
}

\paragraph{Paper Authors}%
\begin{enumerate}
    \item[] Profir-Petru~P\^ar\c{t}achi, Department of Computer Science, University College London, United Kingdom
    \item[] Santanu~Kumar~Dash, University of Surrey, United Kingdom
    \item[] Miltiadis~Allamanis, Microsoft Research, Cambridge, United Kingdom
    \item[] Earl~T.~Barr, Department of Computer Science, University College London, United Kingdom
\end{enumerate}

\paragraph{Abstract}
\subimport{../untangle-tex/}{../untangle-tex/abstract}

\subimport{../untangle-tex/}{../untangle-tex/intro}
\subimport{../untangle-tex/}{../untangle-tex/example}
\subimport{../untangle-tex/}{../untangle-tex/definitions}
\subimport{../untangle-tex/}{../untangle-tex/implementation}
\subimport{../untangle-tex/}{../untangle-tex/evaluation}
\subimport{../untangle-tex/}{../untangle-tex/results}
\subimport{../untangle-tex/}{../untangle-tex/threats}
\subimport{../untangle-tex/}{../untangle-tex/rel_work}
\subimport{../untangle-tex/}{../untangle-tex/conclusion}
\chapter{POSIT: Simultaneously Tagging Natural and Programming Languages}
\label{chapter:posit}

%%%--------------------------------------------------------------------------%%%
%%% Macros from POSIT                                                        %%%
%%%--------------------------------------------------------------------------%%%

\renewcommand{\ourtool}{\textsc{POSIT}\xspace}
\renewcommand{\Ourtool}{\textsc{POSIT}\xspace}
\newcommand{\projurl}{\url{https://github.com/PPPI/POSIT}\xspace}

%%%--------------------------------------------------------------------------%%%
%%% END OF Macros from POSIT                                                 %%%
%%%--------------------------------------------------------------------------%%%

\paragraph{Paper Authors}
\begin{enumerate}
    \item[] Profir-Petru~P\^ar\cb{t}achi, Department of Computer Science, University College London, United Kingdom
    \item[] Santanu~Kumar~Dash, University of Surrey, United Kingdom
    \item[] Christoph~Treude, University of Adelaide, South Australia, Australia
    \item[] Earl~T.~Barr, Department of Computer Science, University College London, United Kingdom
\end{enumerate}

\paragraph{Abstract ---}
\subimport{../code_pos_tex/}{../code_pos_tex/abstract}

\subimport{../code_pos_tex/}{../code_pos_tex/intro}
\subimport{../code_pos_tex/}{../code_pos_tex/motivating_example}
\subimport{../code_pos_tex/}{../code_pos_tex/problem_statement}
\subimport{../code_pos_tex/}{../code_pos_tex/implementation}
\subimport{../code_pos_tex/}{../code_pos_tex/evaluation}
\subimport{../code_pos_tex/}{../code_pos_tex/application}
\subimport{../code_pos_tex/}{../code_pos_tex/discussion}
\subimport{../code_pos_tex/}{../code_pos_tex/rel_work}
\subimport{../code_pos_tex/}{../code_pos_tex/conclusion}
\chapter{Conclusions}
\label{chapter:conclusions}

\todo{We conclude on how the three papers advance the state of the art. How further advances could be made and future directions for project health.}

\section{Summary of Contributions}

\todo{summarise key takeaways of the three papers, sum them up under a single umbrella}

We proposed the first online PR-Issue linker, advanced the state of the art for commit untangling, and open the door to mixed-text analysis.

\section{Summary of Future Work}

\todo{more fluff}

\textbf{Mixed-text comprehension}
POSIT is a step along the path to mixed-text comprehension. It offers the first building blocks, an equivalent for Part of Speech tags for mixed-text. More sophisticated models can build upon this to aid comprehension and parsing of mixed text such that meaning is carried over from the Natural Language channel to the algorithmic channel.

\textbf{Graph Neural Network Approaches for Graph Similarity}
In Flexeme, we make use of graph kernels for graph Similarity computation; however recent research in Graph Neural Networks show promising results, and future work may focus on replacing our graph kernels for a neural network implementation if a suitable training corpus is present.

\textbf{Multi-version PDG Static Analysis}
In Flexeme, we propose a novel structure, the \deltaPDGN. While we demonstrate its use for commit untangling, this same structure could be employed for multi-version static analysis when the analysis depends on Program Dependency Trees. Alternatively, one could consider version sensitive slicing.
% You could separate these out into different files if you have
%  particularly large appendices.

% Actually generates your bibliography. The fact that \include is 
% the last thing before this ensures that it is on a clear page.
\bibliography{full_bib}

% All done. \o/
\end{document}
