\chapter{Supervised Learning:\\Random Forests and Mondrian Forests}
\label{appendix:supervised}

In the context of Machine Learning, Supervised Learning tasks itself with
learning a mapping $x \rightarrowtail y: \mathbb{R}^n \rightarrow Y$ when we
observe pairs: $\{(x^{(1)}, y^{(1)}),\dots,(x^{(n)}, y^{(n)})\}$. If $Y$ is a
continuous set, such as $\mathbb{R}$, then we are solving a regression task,
whereas if $Y$ is discrete, such as $\{c_1, \dots, c_k\}$, we are solving a
classification task. Supervised learning in either case either learns a decision
boundary, such as for Regression models or Support Vector Machines, or learns a
probability distribution over the data for each class, for example in Na\"ive
Bayes or Gaussian Discriminant Analysis. In the remainder of this chapter, I
will first cover some preliminary notions, then I will use them to build up to
Random Forest and Mondrian Forests classification, the main models used in
\Cref{chapter:am}.

\section{Preliminaries}
\label{appendix:supervised:prelim}

\begin{definition}[\textbf{Hypothesis and Hypothesis Space}]
    A hypothesis, noted $h_{\theta}$, is the particular trained model that we
    choose, the model output is represented by $h_{\theta}(x^{(i)})$ for a
    particular $x^{(i)}$. The hypothesis space $\mathcal{H}$ represents the
    space of all functions $h_{\theta}$ that our model can approximate.
\end{definition}
%
\noindent Of note is the use of $\theta$ as a subscript. This makes the
dependence of the hypothesis on the model parameters and hyperparameters
explicit.

\begin{definition}[\textbf{Loss Function}]
    A loss function $L: Y \times Y \rightarrow \mathbb{R}$ is a mapping that
    takes the model prediction and the actual value and outputs how different
    they are. It measures how correct/wrong a model is at a particular point.
    
\end{definition}

\begin{definition}[\textbf{Cost Function}]
    The cost function $J$ is used to model the performance of a model and is
    related to the loss function as follows:
    \begin{align}
        J(\theta) = \sum_{i=1}^{n}L(h_{\theta}(x^{(i)}), y^{(i)})
    \end{align}
\end{definition}

\begin{definition}[\textbf{Likelihood}]
    The likelihood function $\mathcal{L}(\theta)$ is a measure of the models'
    goodness of fit. It represents the join probability of the sample as a
    function of the model parameters. When training a model, often we want to
    maximise the the likelihood function. Often for numerical convivence its
    log,\ie $\log(\mathcal{L}(\theta))$, is used instead of the likelihood
    function.
\end{definition}

\noindent Learning is often formulated in terms of minimising $J(\theta)$ or
maximising $\mathcal{L}(\theta)$.

\section{Decision Trees and Random Forests}
\label{appendix:supervised:randomforests}

The first model of interest to us is the Random Forest classifier. To understand it, we must first describe decision trees.

\begin{definition}[\textbf{Decision Tree}]
    A decision tree is a non-parametric tree model where leaves represent the
    values in $Y$ and the non-terminal nodes are a predicate $\pi$ over the
    input $x$, such that we continue down the left child when $\pi$ is true and
    down the right child otherwise. The goal is to learn simple decision rules
    from the data.
\end{definition}

To learn a decision tree, we iteratively pick a subset of features from the
input vector $\mathbf{x}$. This choice is made based on a notion of local
optimality: maximising information gain or minimising Gini impurity; a globally
optimal solution is known to be NP-Complete~\cite{laurent1976constructing}. How
subsets are chosen differs by algorithm, for example C4.5 first iterates through
each feature and picks the one that provides the best information gain when
splitting $\mathbf{y}$ based on it~\cite{quinlan2014c4}. The next step is to
then recurse on each subset until we either reach a depth limit or we have split
on all attributes.

To go from a Decision Tree to Random Forests we need to apply bagging on both
features and training data. The first step in training a forest of classifiers,
is to create $B$ subsamples from our training data by repeatedly sampling with
replacement from it. We can then train a decision tree on each of the $B$
subsamples. We could stop here, and simply average the responses (continuous),
or majority vote (classification); however, to obtain a Random Forest, we should
also perform the other bagging step. When training a decision tree on a of the
$B$ subsamples, we should also sample a subset of features that this decision
tree can use.

\section{Mondrian Forests}
\label{appendix:supervised:mondrian}

\todo{KD-trees and their generalisation}
