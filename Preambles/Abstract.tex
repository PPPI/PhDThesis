In recent years, systems that would previously live on different platforms have
been integrated under a single umbrella. The increased use of GitHub, which
offers pull-requests, issue tracking and version history, and its integration
with other solutions such as Gerrit, or Travis, as well as the response from
competitors created development environments that favour agile methodologies by
increasingly automating non-coding tasks: automated build systems, automated
issue triaging \etc This in turn can favour leaner and faster development
cycles. Further, by offering the automation as a service, the cost of entry for
projects to make use of it has been reduced. In turn this has created large,
publicly accessible sources of source-code together with related project
artefacts: GHTorrent and similar datasets can now offer programmatic access to
the whole of GitHub.

This shift in tooling has also facilitated a shift in development paradigms,
Adherents of agile methodology can now adopt a continuous integration/continuous
delivery infrastructure more easily. However, the Agile Manifesto suggests
developers should oft forgo tasks that may aid project health in favour of a
higher velocity by traveling light. Still, project health could impact project
success in the long-term. Project health encompasses traceability,
documentation, adherence to coding conventions, in short tasks that allow for
lower maintenance costs and higher accountability.

Simultaneously, this shift to continuous integration/continuous development has
allowed the proliferation of Natural Language or Natural Language and Formal
Language textual artefacts that are programmatically accessible: GitHub and
their competitors allow API access to their infrastructure to enable the
creation of CI/CD bots. This suggests that approaches from Natural Language
Processing and Machine Learning are now feasible and indeed desirable. This
thesis aims to (semi-)automate tasks for this new paradigm and its attendant
infrastructure by bringing to the foreground the relevant NLP and ML techniques.

Under this umbrella, I focus on three synergistic tasks from this domain: (1)
improving the issue-pull-request traceability, which can aid existing systems to
automatically curate the issue backlog as pull-requests are merged; (2)
untangling commits in a version history, which can aid the beforementioned
traceability task as well as improve the usability of determining a fault
introducing commit, or cherry-picking via tools such as git bisect; (3)
mixed-text parsing, to allow better API mining and open new avenues for
project-specific code-recommendation tools.