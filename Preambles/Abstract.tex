In recent years, systems that would previously live on different platforms have
been integrated under a single umbrella. The proliferation of GitHub, which
offers pull-requests, issue tracking and version history, and its integration
with other solutions such as Gerrit, or Travis, as well as the response from
competitors has led to leaner and faster development cycles. This has also
reduced the cost of entry and created large, publicly accessible sources of
source-code together with related project artefacts such as GHTorrent and
similar datasets.

This shift in tooling has also facilitated a shift in development paradigms,
developers now prefer a continuous integration/continuous delivery
infrastructure. This has led to more projects adopting a more agile development
process. Developers often forgo tasks that may aid project health so that they
can instead travel light. However, project health determines project success.
Project health encompasses traceability, documentation, adherence to coding
conventions, in short tasks that allow for lower maintenance costs and higher
accountability.

Simultaneously, this shift has allowed the proliferation of Natural Language or
Natural Language and Formal Language textual artefacts which are
programmatically accessible. This suggests that approaches from Natural Language
Processing and Machine Learning are now feasible and indeed desirable. This
thesis aims to (semi-)automate tasks for this new paradigm and its attendant
infrastructure by bringing to the foreground the relevant NLP and ML techniques.

Under this umbrella, I focus on three synergistic tasks from this domain: (1)
improving the issue-pull-request traceability, which can aid existing systems to
automatically curate the issue backlog as pull-requests are merged; (2)
untangling commits in a version history, which can aid the beforementioned
traceability task as well as improve the usability of determining a fault
introducing commit, or cherry-picking via tools such as git bisect; (3)
mixed-text parsing, to allow better API mining and open new avenues for
project-specific code-recommendation tools.