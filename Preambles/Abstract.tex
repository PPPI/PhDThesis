In recent years, systems that would previously live on different platforms have
been integrated under a single umbrella. The increased use of GitHub, which
offers pull-requests, issue tracking and version history, and its integration
with other solutions such as Gerrit, or Travis, as well as the response from
competitors, created development environments that favour agile methodologies by
increasingly automating non-coding tasks: automated build systems, automated
issue triaging \etc In essence, source-code hosting platforms shifted to
continuous integration/continuous delivery(CI/CD) as a service. This facilitated
a shift in development paradigms, adherents of agile methodology can now adopt a
CI/CD infrastructure more easily. This has also created large, publicly
accessible sources of source-code together with related project artefacts:
GHTorrent and similar datasets now offer programmatic access to the whole of
GitHub.

Project health encompasses traceability, documentation, adherence to coding
conventions, tasks that reduce maintenance costs and increase accountability,
but may not directly impact features.  Over focus on health can slow velocity
(new feature delivery) so the Agile Manifesto suggests developers should travel
light --- forgo tasks focused on a project health in favour of higher feature
velocity. Obviously, injudiciously following this suggestion can undermine a
project's chances for success.

Simultaneously, this shift to CI/CD has allowed the proliferation of Natural
Language or Natural Language and Formal Language textual artefacts that are
programmatically accessible: GitHub and their competitors allow API access to
their infrastructure to enable the creation of CI/CD bots. This suggests that
approaches from Natural Language Processing and Machine Learning are now
feasible and indeed desirable. This thesis aims to (semi-)automate tasks for
this new paradigm and its attendant infrastructure by bringing to the foreground
the relevant NLP and ML techniques.

Under this umbrella, I focus on three synergistic tasks from this domain: (1)
improving the issue-pull-request traceability, which can aid existing systems to
automatically curate the issue backlog as pull-requests are merged; (2)
untangling commits in a version history, which can aid the beforementioned
traceability task as well as improve the usability of determining a fault
introducing commit, or cherry-picking via tools such as git bisect; (3)
mixed-text parsing, to allow better API mining and open new avenues for
project-specific code-recommendation tools.