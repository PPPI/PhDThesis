A PhD is not an easy journey, but one I will look back on fondly. Along the way,
I have had the support of many for whom I wish to dedicate the following
paragraphs in thanks.

I would like to thank my supervisor, Dr Earl T. Barr, for the sage advice and
importantly patience with me especially as I was learning the art of
academic writing. I would also like to thank Earl for bearing with me as I fell
ill throughout the journey and offering support during those times. The
dialectic we had trying to formally prove a hypothesis with the clock ticking to
the deadline will be a fond memory for years despite the stress of the moment
then.

I would also like to thank Dr David R. White for the `Desk Compensation
Meetings' that set me on the path towards rigorous experimental design early in
my PhD. That formed the support beams for my empirical studies throughout my
doctoral work. I would also like to thank David for being there for me when I
needed someone to listen to me and helping navigate the academic landscape with
more confidence.

For the help and patience they showed, I would also like to thank my co-authors
during my PhD research. I would like to thank Dr Santanu Dash, for helping me
grok the rougher sides of Roslyn and enabling the work that followed from that.
I would also like to thank Santanu for helping me pursue an idea that later
became a paper by helping me with initial labelled data that would have been
impossible without his CLANG knowledge. His help and advice on the subtilties of
formal languages helped shape my work.

I extend similar gratitude to Miltos Allamanis and Christoph Treude, who have
made invaluable contributions and were a pleasure to work with as co-authors.
Miltos has helped me get a better start writing machine learning code, while
Christoph helped me better understand how to do qualitative and manual
quantitative analysis.

I would like to thank my lab mates in CREST and SSE that made the journey more
fun: Leo Jeoffe, David Kelly, Bobby Bruce, DongGyun Han, Matheus Paix\~ao,
Carlos Gavidia, Giovani Guizzo, Vali Tawosi, Rebecca Mousa, Jie Zhang. The
informal chats were invaluable and they all helped me through the journey.
Discussing our work together helped me, and I hope them as well, better
understand how to do proper science, to better understand how and when
statistical techniques should be applied, how to formulate experiments, how to
thoroughly explore hypothesises. I also extend my gratitude to my close friend,
Tudor Haru\cb{t}a, with whom I oft shared my progress and who was always
available to chat when I needed to take my mind off of research.

Last but not by any means least, I would like to thank my family, my dear older
sister for always bearing with me and lending a ear to listen and a pair of eyes
to double-check my English. My parents for supporting me through my PhD years
and even donating time on the home PC when I needed extra compute resources on a
tight deadline.

Our research is shaped by those in our network, those we talk to even casually,
thus the sum total of the contributions to this research is beyond any
quantifiable scope; I cannot hope to enumerate everyone who has contributed to
my work in the multitude of ways that I have been supported along on this
journey. To everyone, thank you.