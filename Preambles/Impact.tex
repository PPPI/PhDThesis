In the modern world, software is ubiquitous: from the personal computers most of
us use to the mission critical systems over which we require fine control.
Software quality is usually a result of a healthy software project, \ie it is
not just a function of the code, rather also of the process producing that code.
Further, projects are not just source-code; indeed, they contain a wealth of
documents written in English or other languages. Focusing strictly on
source-code tells us only part the story.

In this thesis, I propose the notion of project health; a notion that was
previously colloquially understood and that encompasses those processes that
help project succeed and flourish. Under this umbrella, I focus on three tasks
that could impede project success: pull-request-issue linking, commit separation
into atomic patches, and mixed-text segmentation and pre-processing. I focus on
tasks that would break assumptions made by researchers when proposing techniques
in the first two, while the latter presents a new way of handling data enabling
analysis which is aware of algorithmic and natural language information. By
borrowing techniques and methodology from Machine Learning, I propose prototype
systems to resolve these issues.

The work I present in this thesis follows the ethos of helping developers help
themselves and us. We, as researchers, help developers maintain project health
with its inherent benefits: easier onboarding, lower costs of maintenance, \etc
This, in turn, can create better training data for researchers and may enable
yet better automation techniques and research avenues for big code. Further, all
tooling created during this thesis is open sourced and made available under the
MIT license for developer or researchers to use directly.